\section{Methods}

\textbf{Lesions:} Ibotenic acid was injected bilaterally in 11 Long-Evans rats (ages from 83 to 141 days; 9 females, 2 males), on 3 injection sites at 2 depths (\SI{-1.5}{\milli\meter} and \SI{-0.75}{\milli\meter} from the surface of the brain). At each depth we injected a total amount of \SI{82.8}{\nano\liter} using a microinjector (Drummond Nanoject II, \SI{9.2}{\nano\liter} per injection, 9 injections per depth). The coordinates for each site, in \si{\milli\meter} with respect to Bregma, were: +1.0 AP / 2.0 ML; +1.0 AP / 4.0 ML; +3.0 AP / 2.0 ML. Five other animals were used as sham controls (age-matched controls; 3 females, 2 males), subject to the same intervention, but where ibotenic acid was replaced with physiological saline. Six additional animals were left as wildtype, no-surgery, controls (age-matched controls; 6 females).

\textbf{Recovery period:} After the surgeries, animals were given a minimum of one week (up to two weeks) recovery period in isolation. After this period, animals were handled every day for a week, after which they were paired again with their age-matched control to allow for social interaction during the remainder of the recovery period. In total, all animals were allowed at least one full month of recovery before they were first exposed to the behaviour assay.

\textbf{Histology:} Animals were perfused (peri?) intracardially with 4\% paraformaldehyde in phosphate buffer saline (PBS) and brains were post-fixed for at least \SI{24}{\hour} in the same fixative. Serial coronal sections (\SI{100}{\micro\meter}) were Nissl-stained and imaged for identification of lesion boundaries.

\textbf{Behaviour:} The animals were kept in a state of water deprivation for \SI{20}{\hour} before each daily session. During the session the animal was placed inside a behaviour box for \SI{30}{\minute}, where it could collect water by shuttling back and forth between two reward ports (Island Motion Corporation, USA). To do this, animals had to cross a \SI{48}{\centi\meter} obstacle course composed of eight \SI{2}{\centi\meter} aluminum steps spaced by \SI{4}{\centi\meter} (Figure \ref{fig:assay}A). The structure of the behaviour assay and each step in the obstacle course was built out of Bosch Rexroth aluminum structural framing, \SI{20}{\milli\meter} series. For every trial, rats were delivered a \SI{20}{\micro\liter} drop of water. At the end of the day, rats were given free access to water for \SI{10}{\minute} before initiating the next deprivation period.

A motorized brake allowed us to lock or release each step in the obstacle course (Figure \ref{fig:assay}B). The shaft of each of the obstacles was coupled to an acrylic piece used to control the rotational stability of each step. In order to lock a step in a fixed position, two servo motors are actuated to press against the acrylic piece and hold it in place. Two other acrylic pieces were used as stops to ensure a maximum rotation angle of approximately +/- \ang{100}. Two small nuts were attached to the bottom of each step to work as light weights that give the obstacles a tendency to return to their original flat configuration. In order to ensure that noise from servo motor actuation cannot be used as a cue to tell the animal about the state of each step, the motors are always set to press against an acrylic piece, either the piece that keeps the step stabilized, or the acrylic stops. At the beginning of each trial, the motors go through a randomized sequence of positions in order to mask information about state transitions and also to ensure the steps are reset to their original configuration. Control of the motors was done using a Motoruino board (Artica, PT) along with a custom workflow written in the Bonsai visual programming language \cite{Lopes2015a}.

Animals were run on \SI{30}{\minute} sessions six days of the week from Monday to Saturday, with a day of free access to water on Sunday. Before the water deprivation begins, animals were run on a single habituation session where they were placed in the box for a period of \SI{15}{\minute}. Animals were run on the following sequence of conditions over the course of a month (see also Figure \ref{fig:assay}A): day 0, habituation to the box; day 1-4, all the steps were fixed in a stable configuration; day 5, 20 trials of the stable configuration, after which the two center steps are made free to rotate; day 6-10, the center two steps remain free to rotate; day 11, 20 trials of the partially unstable configuration, after which the two center steps are again fixed in a stable state; day 12, all the steps were fixed in a stable configuration; day 13-16, the state of the center two steps was randomized on a trial-by-trial basis to be either stable or free to rotate; day 17-18, 20 trials of stable configuration, followed by the randomized protocol for the center two steps; day 19-22, the state of the six center steps was randomized trial-by-trial in such a way that on each trial either two rails are picked at random to be free to rotate or with a low probability all rails are stable; day 23, after 20 trials of randomized permutations, all rails are made free to rotate; day 24, all rails are free to rotate.

\textbf{Behaviour Data:} The performance of the animals was recorded using a high-speed and high-resolution video camera (PointGrey Flea3 FL3-U3-13S2M-CS, 1280x960 @ 120 fps). Behaviour was run in the dark, under infrared illumination using super-bright LED strips. Tracking of the nose was achieved by background subtraction and connected component labeling of segmented image elements, of which we then selected the furthermost point along the major axis of the ellipse best fit to the largest segmented object.

\textbf{Video Classification:} Ethogram classification was done on a frame-by-frame analysis of the high-speed video aligned on first contact with the manipulated rail. The frame of first contact was defined as the first frame in which there is noticeable movement of the rail caused by animal contact. Three main categories of behaviour were observed to follow the first contact: compensation, investigation and halting. Behaviour sequences were first classified as belonging to one of these categories and their onsets and offsets determined by the following criteria. Compensation behaviour is defined by rapid, whole-body (?) movement that results in an adaptive postural correction to the perturbation. Onset of this behaviour is defined by the first frame in which there is visible rapid contraction of the body musculature following first contact. Investigation behaviour consists of periods of targeted interaction with the rails, often involving manipulation of the freely moving obstacle with the forepaw. The onset of this behaviour is defined by the animal orienting its head down to one of the manipulated rails, followed by subsequent interaction. Halting behaviour is characterized by a period in which the animals stop their ongoing motor program, and maintain the same body posture for several seconds, with occasional movements of the head, but without orienting specifically to the manipulated rails. They tend to look up and to the sides as if in a state of confusion (too vague?), or uncertainty about what to do next. Onset of this behaviour is defined by the moment where locomotion and other motor activities besides movement of the head come to a stop. A human classifier blind to the lesion condition was given descriptions of each of these three main categories of behaviour and asked to note onsets and offsets of each behaviour throughout the video.

\textbf{Data Analysis:} Analysis was run using the NumPy scientific computing package for the Python programming language and custom software. We need more details of specific analysis techniques. Maybe.
