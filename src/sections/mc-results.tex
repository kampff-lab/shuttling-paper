\section{Results}

To investigate whether the intact motor cortex is required for the robust control of movement in response to unexpected perturbations, we designed a reconfigurable dynamic obstacle course where individual steps can be made stable or unstable on a trial-by-trial basis (Figure \ref{fig:assay}, also see Methods). In this assay, rats are motivated to go back and forth across the obstacles, in the dark, in order to collect water rewards. We specifically designed the assay such that modifications to the physics of the obstacles could be made covertly. In this way, the animal has no explicit information about the state of the steps until he actually walks over them. Water deprived animals were trained daily for 4 weeks, throughout which they encountered increasingly challenging states of the obstacle course. Our goal was to characterize precisely the conditions under which motor cortex becomes necessary for the control of movement, and this motivated us to introduce an environment with graded levels of uncertainty.

We compared the performance of 22 animals, 11 with bilateral lesions to the primary and secondary forelimb motor cortex and 11 age and gender matched controls (5 sham surgery, 6 wild-types). Animals were given ample time to recover, 4 weeks post-surgery, in order to specifically isolate behaviours that are chronically impaired in animals lacking the functions enabled by motor cortical structures. Histological examination of the lesions revealed significant variability in the extent of damaged areas (Figure \ref{fig:histology}), which was in some cases traced back to spurious mechanical blockage of the injection pipette during lesion induction. Nevertheless, volume reconstruction of the serial sections allowed us to accurately quantify the size of each lesion and use these values to compare observed behavioural effects as a function of lesion size.

During the first sessions in the ``stable'' environment, all animals, both lesions and controls, quickly learned to shuttle across the obstacles, achieving stable, skilled performance after a few days of training (Figure \ref{fig:learning}A). Even though the distance between steps was fixed for all animals, the time taken to adapt the crossing strategy was similar irrespective of body size. When first encountering the obstacles, animals adopted a cautious gait, investigating the location of the subsequent obstacle with their whiskers, stepping with the leading forepaw followed by a step to the same position with the trailing paw (Video \ref{vid:learning}: ``First Leftwards Crossing''). However, over the course of only a few trials, all animals exhibited a new strategy of ``stepping over'' the planted forepaw to the next obstacle, suggesting an increased confidence in their movement strategy in this novel environment (Video \ref{vid:learning}: ``Second Leftwards Crossing'').

This more confident gait developed into a coordinated locomotion sequence after a few additional training sessions (Video \ref{vid:learning}: ``Later Crossing''). The development of the ability to move confidently and quickly over the obstacle course was observed in both lesion and control animals (Video \ref{vid:learning-matrix}). Surprisingly, we did not observe noticeable deficits in paw placement in the lesioned animals compared to controls (Figure \ref{fig:learning}C) as previously reported following motor cortical lesion in the rat \cite{Metz2002} and pyramidal tract section in the cat \cite{Liddell1944}.

In addition to the excitotoxic lesions, in three animals we performed larger frontal cortex aspiration lesions in order to determine whether the remaining trunk and hindlimb representations were necessary to navigate the stable obstacle course. Also, in order to exclude the involvement of other corticospinal projecting regions in the parietal and rostral visual areas \cite{Miller1987}, we included three additional animals which underwent even more extensive cortical lesion procedures (see Methods). In these larger cortical lesions, recovery was found to be overall slower than in lesions limited to the motor cortex and animals required isolation and extended care during the recovery period.

Nevertheless, when tested in the shuttling assay, the overall performance of these animals was indistinguishable from controls and other lesions (Figure \ref{fig:learning}B, Video \ref{vid:learning-matrix}). Only in the animal with the largest frontoparietal lesion, which extended all the way to rostral visual cortex (Extended Lesion F), did we observe an immediate and obvious deficit in paw placement upon the first encounter with the obstacles. However, over the course of ten repeated trials in that same session, this impairment recovered dramatically, up to the point where the animal was moving as efficiently as controls or other lesioned animals (Video \ref{vid:decorticate-habituation}). The ability of this animal to improve its motor control strategy in such a short period of time seems to indicate the presence of motor learning and not simply an increase in confidence with the new environment. The initial movement defect seems to be consistent with results from sectioning the entire pyramidal tract in cats \cite{Liddell1944}, and as observed in these studies, quickly disappeared over the course of a few trials.

In subsequent training sessions we progressively increased the difficulty of the obstacle course, by progressively making more steps in the course unstable. The goal was to compare the performance of the two groups as a function of difficulty. Surprisingly, both lesion and control animals were able to improve their performance by the end of each training stage even for the most extreme condition where all steps were made unstable (Figure \ref{fig:learning}A, Video \ref{vid:conditions}). This seems to indicate that the ability of these animals to fine-tune their motor performance in a challenging environment remained intact.

One noticeable exception was the animal with the largest ibotenic acid lesion. This animal, following exposure to the first unstable protocol, was unable to bring himself to cross the obstacle course (Video \ref{vid:jpak20}). Some other control and lesioned animals experienced a similar form of distress following exposure to the first change in the environment, but eventually all managed to start crossing the obstacles again over the course of a single session. In order to test whether this was due to some kind of motor disability, we lowered the difficulty of the protocol for this one animal until he was able to cross again. Following a random permutation protocol where any two single steps were released randomly, this animal was able to get itself to cross a single released obstacle placed in any location of the assay. After this success, he eventually learned to cross the highest difficulty level in the assay in about the same time as all the other animals, which convinced us that there was indeed no lasting motor execution or learning deficit, and that the disability must have been due to some other unknown (cognitive) factor. 

Having established that the overall motor performance of these animals was similar across all conditions, we next asked whether there was any difference in the strategy used by the two groups of animals to cross the obstacles. We noticed that during the first week of training, the posture of the animals when stepping on the obstacles changed significantly over time (Figure \ref{fig:posture}B,C). Specifically, the centre of gravity of the body was shifted further forward and higher in later days, in a manner proportional to performance. However, after the obstacles changed to the unstable state, we observed an immediate and persistent adjustment of this posture, with animals assuming a lowered centre of gravity as they resumed shuttling across the assay (Figure \ref{fig:posture}B,C). Interestingly, we also noticed that a group of animals adopted a different strategy. Instead of lowering their centre of gravity, they either kept it unchanged or shifted it even more forward and performed a jump over the affected obstacles (Figure \ref{fig:jumping}A,B). These two strategies were remarkably consistent across the two groups of animals but there was no correlation between the strategy used and the degree of motor cortical lesion (Figure \ref{fig:posture}D-F, \ref{fig:jumping}C). In fact, we found the best predictor of whether an animal was a ``jumper'' to be the body weight of the animal (Figure \ref{fig:jumping}C).

During the two days where the stable state of the environment was reinstated, the posture of the animals was gradually restored to pre-manipulation levels (Figure \ref{fig:posture}B,C), although often at a slower rate than the transition from stable to unstable. Again, this postural adaptation was independent of the presence or absence of forepaw motor cortex.

We next looked in detail at the days where the state of the obstacle course was randomized on a trial-by-trial basis. This stage of the protocol is particularly interesting as it reflects a situation where the environment has a persistent degree of uncertainty. For this analysis, we were forced to exclude the animals that employed a jumping strategy, as their experience with the manipulated obstacles was the same irrespective of the state of the world. First, we repeated the same posture analysis comparing all the stable and unstable trials in the random protocol in order to control whether there was any cue in our motorized setup that the animals might be using to gain information about the current state of the world. There was no significant effect between stable and unstable trials on the posture of the animal (Figure \ref{fig:random}A). However, by classifying the trials on the basis of past trial history, a significant effect on posture was obtained (Figure \ref{fig:random}B), which became more pronounced the more consistent the immediate history was (Figure \ref{fig:random}C). This suggests that the animals adjust their body posture when stepping on the affected obstacles on the basis of their current expectation about the state of the world, which is updated by the previously experienced states. Surprisingly, this effect again did not depend on the presence or absence of frontal motor cortical structures (Figure \ref{fig:random}D-F).

Finally, we decided to test whether general motor performance was affected by the randomized state of the obstacles. If the animals do not know what state the world will be in, then there will be an increased challenge to their stability when they cross over the unstable obstacles, possibly demanding a quick change in strategy when they learn whether the world is stable or unstable. In order to evaluate the dynamics of crossing, we measured the speed at which the animals were moving at each point in the assay, aligned on the first manipulated step. Consistent with the past history effect discussed above, there was also a consistent difference in the speed with which the animals approached the obstacles, depending on the state of the world during the previous trial (figure?). In order to normalize for this absolute change in speed we used the average speed with which the animal approached the step as a baseline and subtracted it from the overall profile for each trial. In order to summarize the difference in crossing speeds, we subtracted the average speed in unstable trials from the average speed in stable trials at each position in the assay and computed the sum of this measure for each animal (Figure \ref{fig:speed}). Interestingly, two of the largest lesions appeared to be significantly slowed down on unstable trials, while controls and the smallest lesions instead tended to accelerate after encountering an unstable obstacle. However, the overall effect for lesions versus controls is not statistically significant ($p = 0.14$).

Nevertheless, we were intrigued by this observation and decided to investigate in detail the single moment in the assay where the effect of a perturbation should be the most salient. In the random protocol, even though the state of the world is unpredictable, the animals know that the obstacles might become unstable. However, the very first time the environment becomes unstable, the collapse of the obstacles is completely unexpected and demands an entirely novel motor response.

A detailed analysis of the responses to the first collapse of the steps revealed a striking difference in the strategies deployed by the lesion and control animals. Upon the first encounter with the manipulated steps, we observed three types of behavioural responses from the animals (Video \ref{vid:manipulation-strategies}): investigation, in which the animals immediately stop their progression and orient towards, whisk and physically manipulate the altered obstacle; compensation, in which the animals rapidly adjust their behaviour to negotiate the unexpected obstacle; and halting, in which the ongoing motor program ceases and the animals' behaviour simply comes to a stop for several seconds. Remarkably, these responses depended on the presence or absence of motor cortex (Figure \ref{fig:ethogram}). All animals with significant motor cortical lesions, upon their first encounter with the novel environmental obstacle, halted for several seconds, whereas animals with an intact motor cortex, or with vestigial lesions, were able to rapidly react with either an investigatory or compensatory response (Video \ref{vid:manipulation-small},\ref{vid:manipulation-large}).

The response of animals with extended lesions was even more striking: in two of these animals, there was a failure to recognize that a change had occurred at all (Video \ref{vid:manipulation-decorticate-oblivious}). Instead, they kept walking across the now released steps as best they could for several trials, never stopping to assess the new situation. Over time, one of them gradually realized the change and stopped his progression, while the second one only realized the change after tripping and inadvertently hitting the steps with its snout. This was the first time we ever observed this behaviour, as all animals with or without cortical lesions always displayed a clear switch in behavioural state following the first encounter with the manipulation. The remaining animals all halted their progression following the collapse of the obstacles, in a way similar to the large lesions (Video \ref{vid:manipulation-decorticate-halting}).
