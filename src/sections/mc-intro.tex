\section{Introduction}

The involvement of the brain and spinal cord in motor control has been recognized since the earliest known clinical records of head and spinal injuries, dating back to ancient Egypt \citep{Louis1994,VanMiddendorp2010}. However, the mechanism used by the nervous system to generate movement was not fully appreciated until Galvani first reported his famous experiments on \textit{animal electricity} \citep{Galvani1791}. By isolating the sciatic nerve and gastrocnemius muscle in the frog, Galvani clearly demonstrated in a series of stimulation experiments that an electrical process, contained entirely within the biology of the frog's leg, was responsible for the spontaneous generation of muscle contractions. This would lead to the discovery and physiological characterization of the nerve impulse, the action potential, that travels across the nerve to initiate muscle movement \citep{DuBois-Reymond1843,Bernstein1868,Schuetze1983}. The success of these seminal experiments immediately raised a fundamental question regarding nerve conduction: if spontaneous muscle contraction is generated by nerve impulses transmitted throughout the nervous system, how is this transmission coordinated in order to generate the complex patterns of muscle activity observed in natural behaviour?

\subsection{Discovery of the motor cortex}

In search of answers to this question, researchers next turned to the brain, the seat of anatomical convergence of the nervous system. Following Galvani's footsteps, several attempts were made to stimulate the cerebral cortex electrically, but with little success \citep{Gross2007}. It wasn't until the 1870s that the first indications of a direct involvement of the cortex in the production of movement came to light, around the time when Hughlings Jackson undertook his studies on epileptic convulsions \citep{Jackson1870}. He observed that in some patients the fits would start by a deliberate spasm on one side of the body, and that different body parts would become systematically affected one after the other. He connected the orderly march of these spasms to the existence of localized lesions in the \emph{post-mortem} brain of his patients and hypothesized that the origin of these fits was uncontrolled excitation caused by local changes in cortical grey matter \citep{Jackson1870}. In that same year, Fritsch and Hitzig published their famous study demonstrating that it is possible to elicit movements by direct stimulation of the cortex in dogs \citep{Fritsch1870}. Furthermore, stimulation of different parts of the cortex produced movement in different parts of the body \citep{Fritsch1870}. It appeared that the causal mechanism for epileptic convulsions predicted by Hughlings Jackson had been found, and with it a possible explanation for how the intact brain might control movement. The cerebral cortex was already considered at the time to be the seat of reasoning and sensation, so if activity over this so-called \emph{motor cortex} was able to exert direct control over the musculature of the body, then it might, in the normal brain, be the area that connects volition to muscles \citep{Fritsch1870}.

\subsection{The Goltz-Ferrier debates}

David Ferrier, a Scottish neurologist deeply impressed by the ideas of Hughlings Jackson and by the positive results of Fritsch and Hitzig's experiments, proceeded to reproduce and expand on their observations with comprehensive stimulation studies showing how activity in the motor cortex was sufficient to produce a large variety of movements across a wide range of mammalian species \citep{Ferrier1873}. Meanwhile, other researchers across Europe such as Goltz and Christiani were facing a dilemma: in many of the so-called ``lower mammals'' massive lesions of the cerebral cortex failed to demonstrate any visible long-term impairments in the motor behaviour of animals \citep{James1885,Goltz1888}.  These two lines of inquiry first clashed at the seventh International Medical Congress held in London in August 1881, where Goltz of Strassburg and Ferrier of London presented their results in a series of debates on the localization of function in the cerebral cortex \citep{Phillips1984,Tyler2000}.

Goltz assumed a clear anti-localizationist position. He advanced that it was impossible to produce a complete paresis of any muscle, or complete dysfunction of any perception, by destruction of any part of the cerebral cortex, and that he found mostly deficits of general intelligence in his dogs \citep{Tyler2000}. Following Goltz's presentation, Ferrier emphasized the danger of generalizing from the dog to animals of other orders (e.g. man and monkey). He then proceeded to exhibit his own lesion results by means of antiseptic surgery in the monkey, describing how a circumscribed unilateral lesion of the motor cortex produced complete contralateral paralysis of the leg. He also produced a striking series of microscopic sections of Wallerian degeneration \citep{Waller1850} of the ``motor path'' from the cortex to the contralateral spinal cord, the crossed descending projections forming the pyramidal corticospinal tract \citep{Tyler2000}.

The debates concluded with the public demonstration of live specimens: a dog with large lesions to the parietal and posterior lobes from Goltz; and from Ferrier, a hemiplegic monkey with a unilateral lesion to the motor cortex of the contralateral side. As predicted, Goltz's dog showed a clear ability to locomote and avoid obstacles and to make use of its other basic senses, while displaying peculiar deficits of intelligence such as failing to respond with fear to the cracking of a whip or ignoring tobacco smoke blown in its face. On the other hand, Ferrier's monkey appeared severely hemiplegic, in a condition similar to human stroke patients. After the demonstrations, the animals were killed and their brains removed. Preliminary observations revealed that the lesions in Goltz's dog were less extensive than expected, particularly on the left hemisphere. Ferrier's lesions on the other hand were precisely circumscribed to the contralateral motor cortex. These demonstrations secured the triumph of Ferrier, who went on to firmly establish the localizationist approach to neurology and the idea of a somatotopic arrangement over the motor cortex.

The Goltz-Ferrier debates had far-reaching implications throughout the entire research community of the time, and the basic dilemma that was presented has sparked controversy and confusion for over a hundred years since \citep{Phillips1984,Lashley1924,DeBarenne1933,Tyler2000,Gross2007}. In the meantime, views of motor cortex have evolved to suggest it plays a role in ``understanding'' the movements of others \citep{Rizzolatti2004}, imagining one's own movements \citep{Porro1996}, or in learning new movements \citep{Kawai2015}, but where are we today regarding its role in directly controlling movement?

\subsection{Stimulating motor cortex causes movement; motor cortex is active during movement}

Motor cortex is still broadly defined as the region of the cerebral hemispheres from which movements can be evoked by low-current stimulation, following Fritsch and Hitzig's original experiments in 1870 \citep{Fritsch1870}. Stimulating different parts of the motor cortex elicits movement in different parts of the body, and systematic stimulation surveys have revealed a topographical representation of the entire skeletal musculature across the cortical surface \citep{Leyton1917, Penfield1937, Neafsey1986}. Electrophysiological recordings in motor cortex have routinely found correlations between neural activity and many different movement parameters, such as muscle force \citep{Evarts1968}, movement direction \citep{Georgopoulos1986}, speed \citep{Schwartz1993}, or even anisotropic limb mechanics \citep{Scott2001} at the level of both single neurons \citep{Evarts1968,Churchland2007} and populations \citep{Georgopoulos1986,Churchland2012}. Determining what exactly this activity in motor cortex controls \citep{Todorov2000} has been further complicated by studies using long stimulation durations in which continuous stimulation at a single location in motor cortex evokes complex, multi-muscle movements \citep{Graziano2002,Aflalo2006}. However, as a whole, these observations all support the long standing view that activity in motor cortex is involved in the direct control of movement.

\subsection{Motor cortex lesions produce different deficits in different species}

What types of movement require motor cortex? In humans, a motor cortical lesion is devastating. Permanent injury to the frontal lobes of the brain by stroke or mechanical means is often followed by weakness or paralysis of the limbs in the side of the body opposite to the lesion \citep{Louis1994}. Although the paretic symptoms have a tendency to recover partially, especially with training and rehabilitation, permanent movement deficits and loss of muscle control in the affected limbs is the common prognosis; movement is permanently and obviously impaired \citep{Laplane1977,Kwakkel2003}. In non-human primates, similar gross movement deficits are observed after lesions, albeit transiently \citep{Leyton1917,Travis1955}. The longest lasting effect of a motor cortical lesion is the decreased motility of distal forelimbs, especially the control of individual finger movements required for precision skills \citep{Leyton1917,Darling2011}. But equally impressive is the extent to which other movements fully recover, including the ability to sit, stand, walk, climb and even reach to grasp, as long as precise finger movements are not required \citep{Leyton1917,Darling2011,Zaaimi2012}. In non-primate mammals, the \emph{absence} of lasting deficits following motor cortical lesion is even more striking. Careful studies of skilled reaching in rats have revealed an impairment in paw grasping behaviours \citep{Whishaw1991,Alaverdashvili2008a}, comparable to the long lasting deficits seen in primates, but this is a limited impairment when compared to the range of movements that are preserved \citep{Whishaw1991,Kawai2015}. In fact, even after complete decortication, rats, cats and dogs retain a shocking amount of their movement repertoire \citep{Goltz1888,Bjursten1976,Terry1989}. If we are to accept the simple hypothesis that motor cortex is the structure responsible for ``voluntary movement production'', then why is there such a blatant difference in the severity of deficits caused by motor cortical lesions in humans versus other mammals? With over a century of stimulation and electrophysiology studies clearly suggesting that motor cortex is involved in many types of movement, in all mammalian species, how can these divergent results be reconciled?

\subsection{The role of the corticospinal tract}

It must have felt uncanny to those early researchers to find that surface stimulation of the cortex produces discrete muscle responses, in a way so similar to what Galvani did with the frog's leg. Indeed, Sherrington himself conveys the feeling clearly in the opening of his seminal lecture on the motor cortex \cite[p.271]{Sherrington1906}, confessing ``that although it is not surprising that such territorial subdivision of function should exist in the cerebral cortex, it is surprising that by our relatively imperfect artifices for stimulation we should be able to obtain clear evidence thereof.''

Of course, it did not go unnoticed that this fact might be due to the massive projection from cortex to the spinal cord, which had been fully traced by Ludwig Türck only twenty years before Fritsch and Hitzig's experiment \citep{Nathan1955}. This corticospinal tract was found to originate in the anterior regions of the cerebral cortex and terminate directly in the lateral columns of the spinal cord after decussating (i.e. crossing over) at the level of the brainstem's \emph{medulla oblongata}. The existence of this corticospinal pathway presented compelling anatomical evidence of the means by which the motor cortex might be able to exert a direct influence on movement by electrical conduction of nerve impulses, but the role of this connection remained elusive.

\subsection{The effects of lesions in the corticospinal tract}

In the wake of the Goltz-Ferrier debates, investigations of the role of the direct corticospinal descending pathway were conducted in multiple animal species. Sherrington himself started out his work by tracing spinal cord degeneration over large periods of time (up to 11 months) following cortical lesions in Goltz's dogs \citep{Langley1884,Sherrington1885}. He confirmed that many of the properties of the corticospinal tract in the primate held for the dog, and furthermore became one of the first to observe the presence of a degenerated ``re-crossed'' pyramidal tract that travels down the cord ipsilateral to the side of the lesion \citep{Sherrington1885}. These fibers would later come to be called the ipsilateral, ventral corticospinal tract, and have since been found and described in most mammalian species as forming roughly 10\% of the entire corticospinal projections \citep{Kuypers1981,Brosamle2000,Lacroix2004}. However, he also had the chance during this time to observe first hand the negative effects of corticospinal degeneration following lesion, which had been previously reported by Goltz and others in a variety of non-primate specimens. In his own words:

\blockquote[{\protect\cite[p.189]{Sherrington1885}}]{That the pyramidal tracts are in the dog requisite for volitional~impulses to reach limbs and body seems negatived by the fact that the animal can run, leap, turn to either side, use neck and jaws, \&c. with ease and success after nearly, if not wholly, complete degeneration of these tracts on both sides. Further, after complete degeneration of one pyramid, there is in the dog no obvious difference between the movements of the right and left sides.}

Interestingly, he does note that \enquote{defect of motion is observable only as a clumsiness in execution of fine movements} \citep{Sherrington1885}. These observations once again stood out in stark contrast with lesion experiments reported by Ferrier in the monkey, where cauterization of specific motor cortical areas produced complete and persistent paralysis of the corresponding body parts \citep{Ferrier1884}.

Years later, Sherrington would come back to the motor cortex with a new set of studies on stimulation and ablation of the precentral region \citep{Grunbaum1903,GrahamBrown1913,Leyton1917}. In these studies together with Gr\"unbaum, Sherrington targeted motor cortical lesions to the excitable area of the arm or the leg and tracked the recovery of the animals over time. Following the initial paresis and loss of muscle control they observed dramatic recovery of most skilled motor acts, such as peeling open a banana or climbing cages \citep{Leyton1917}. In order to test whether the recovery process was due to cortical reorganization, they systematically stimulated the areas adjacent to the lesion as well as the motor cortex of the opposite hemisphere, but failed to evoke movements in the affected limb \citep{Leyton1917}, as would be expected if commands were traveling down the corticospinal tract in spared regions. Furthermore, subsequent ablation of those areas failed to produce any new impairments in the recovered limb, leaving Sherrington and his colleagues at a loss to find the locus of recovery \citep{Leyton1917}.

Confused by these results, which they thought ``caused concern to, students of cerebral physiology'', Glees and Cole introduced a set of more quantitative behavioural assays in the hope of tracking in detail the recovery of motor control \citep{Glees1950,Cole1952}. They studied the behaviour of monkeys solving various puzzle boxes following successive circumscribed lesions to the thumb, index and arm areas of the motor cortex. As Sherrington reported, there was a quick recovery after an initial period of paralysis and loss of motor control. However, even though the monkeys fully recovered their ability to skillfully open the puzzle box, some subtle movement deficits and paresis in the control of fine movements of the digits was reported to persist \citep{Glees1950}. When stimulating motor cortical areas surrounding the circumscribed lesions, they were able to evoke movements in the impacted digits and reinstate the paretic symptoms after further ablation \citep{Glees1950}. This suggested the hypothesis that surrounding areas of the motor cortex could undergo reorganization following the lesion. However, an important difference to emphasize between these experiments and those of Sherrington is the fact that only relatively circumscribed motor cortical regions were removed in each surgery, whereas in the original Sherrington study the entire elbow, wrist, index, thumb and remaining digit motor areas were excised at once \citep{Leyton1917}, most likely causing degeneration of the entire corticospinal pathway for the affected limb. The presence of an intact corticospinal tract, excitability of movements to low-current stimulation and transient paretic symptoms following ablation thus seem to go hand in hand.

In the hopes of clarifying the confusion of exactly which movements were controlled by cortex, other studies focused on lesions restricted to the corticospinal tract, using both unilateral and bilateral section at the level of the medullary pyramids \citep{Tower1940,Lawrence1968,Lawrence1968a}. The goal was to isolate the effects of all the individual descending pathways to the spinal cord and resolve once and for all the question of whether the corticospinal tract of the motor cortex was the source of all ``voluntary'' movements. Sarah Tower was the first to describe in detail the results of unilateral and bilateral pyramidotomy in primates, with and without lesion of the motor cortex \citep{Tower1940}. She summarized the condition as ``hypotonic paresis'', characterized by a loss of skeletal muscle tone and depression of the vasomotor system, along with general weakening of the reflexes involving the affected limb segments. Although all discrete usage of the hand and digits was eliminated, she did emphasize the clear presence of voluntary movements in the various purposeful compensations produced by the animals to deal with the affliction. Tower attributed these compensations to the preserved capacities of brainstem circuits.

A more definitive study to dissociate the effects of direct corticospinal and indirect brainstem descending pathways was conducted by Lawrence and Kuypers, and presented in their now classical publications \citep{Lawrence1968,Lawrence1968a}. Using the Kl\"uver board, a task where monkeys have to pick morsels of food from differently sized round holes, they observed that while normal monkeys routinely pick up the food by pinching individual bits with their fingers, monkeys with bilateral corticospinal lesions were mostly unable to perform this precise pincer movement, and instead employed coarser compensatory clasping strategies to retrieve the food \citep{Lawrence1968}. In addition, lesioned monkeys were consistently reported to be somewhat slower and less agile than normal animals. However, most of their overall movement repertoire was surprisingly preserved. Their final conclusions fit remarkably well with the initial observations of Sherrington in the dog, suggesting that the corticospinal pathways superimpose speed and agility on subcortical mechanisms, and provide the capacity for fractionation of movements such as independent finger movements \citep{Lawrence1968}. These observations recapitulate the effects of motor cortical lesions reported by Sherrington, but remain at odds with the primary role assigned to motor cortex, and the direct corticospinal tract, with the control of all voluntary movements.

\subsection{There are anatomical differences in corticospinal projections between primates and other mammals}

In primates, the conspicuous effects of motor cortical lesion can also be induced by sectioning the corticospinal tract, the direct monosynaptic projection that connects motor cortex, and other cortical regions, to the spinal cord \citep{Tower1940,Lawrence1968}. In monkeys, and similarly in humans, this pathway has been found to directly terminate on spinal motor neurons responsible for the control of distal muscles \citep{Leyton1917,Bernhard1954} and is also thought to support the low-current movement responses evoked by electrical stimulation of the cortex, as evidenced by the increased difficulty in obtaining a stimulation response following section at the level of the medulla \citep{Woolsey1972}.

However, the corticospinal tract is by no means the only pathway from cortex to movement (Figure \ref{fig:descendingTaxa}). Motor cortex targets many other brain regions that can themselves generate movement. In fact, this specialized connection from telencephalon to spinal cord appeared only recently in vertebrate evolution \citep{TenDonkelaar2009}, and was further elaborated to include a direct connection from cortex to motor neurons only in some primate species and other highly manipulative mammals such as raccoons \citep{Heffner1983}. In all other mammals, including cats and rats, the termination pattern of the corticospinal tract largely avoids the motor neuron pools in ventral spinal cord and concentrates instead on intermediate zone interneurons and dorsal sensory neurons \citep{Kuypers1981,Yang2003}. Why then is there such a large dependency on this tract for human motor control? One possibility is that the rubrospinal tract---a descending pathway originating in the brainstem and terminating in the intermediate zone---is degenerated in humans compared to other primates and mammals \citep{Nathan1955,Nathan1982}, and is thought to play a role in compensating for the loss of the corticospinal tract in non-human species \citep{Lawrence1968a,Zaaimi2012}.

It thus seems likely that most mammals rely on ``indirect'' pathways to convey cortical motor commands to muscles. These differences in anatomy might explain the lack of conspicuous, lasting movement deficits following motor cortical lesion in non-primates, but leaves behind a significant question: what is the motor cortex actually controlling in all these other mammals?

\subsection{What is the role of motor cortex in non-primate mammals?}

In the rat, a large portion of cortex is considered ``motor'' based on anatomical \citep{Donoghue1982}, stimulation \citep{Donoghue1982,Neafsey1986} and electrophysiological evidence \citep{Hyland1998}. However, the most consistently observed long-term motor control deficit following motor cortical lesion has been an impairment in supination of the wrist and individuation of digits during grasping, which in turn impairs reaching for food pellets through a narrow vertical slit \citep{Whishaw1991,Alaverdashvili2008a}. Despite the fact that activity in rodent motor cortex has been correlated with movements in every part of the body (not just distal limbs) \citep{Hill2011,Erlich2011}, it would appear we are led to conclude that this large high-level motor structure, with dense efferent projections to motor areas in the spinal cord \citep{Kuypers1981}, basal ganglia \citep{Turner2000,Wu2009}, thalamus \citep{Lee2008}, cerebellum \citep{Baker2001} and brainstem \citep{Jarratt1999}, as well as to most primary sensory areas \citep{Petreanu2012,Schneider2014}, evolved simply to facilitate more precise wrist rotations and grasping gestures. Maybe we are missing something. Might there be other problems in movement control that motor cortex is solving, but that we may be overlooking with our current assays?

\subsection{A role in modulating the movements generated by lower motor centres}

The idea that the descending cortical pathways superimpose speed and precision on an existing baseline of behaviour has been suggested by lesion work in the primate \citep{Lawrence1968a}, but has been investigated much more thoroughly in the context of studies on the neural control of locomotion in cats. These studies have suggested that the corticospinal tract can play a role in the \emph{adjustment} of ongoing movements, modulating the activity and sensory feedback in spinal circuits in order to adapt a lower movement controller to challenging conditions.

It has been known for more than a century that completely decerebrate cats are capable of sustaining the locomotor rhythms necessary for walking on a flat treadmill utilizing only spinal circuits \citep{GrahamBrown1911}. In addition, there is a general capacity for spinal circuits to modulate network activity with incoming sensory input in order to coordinate and switch between different responses, even during specific phases of movement \citep{Forssberg1975}. Brainstem and midbrain circuits are sufficient to initiate the activity of these spinal central pattern generators \citep{Grillner1973}, so what exactly is the contribution of motor cortex to the control of locomotion? Single-unit recordings of pyramidal tract neurons (PTNs) from cats walking on a treadmill have shown that a large proportion of these neurons are locked to the step cycle \citep{Armstrong1984a}. However, we know from the decerebrate studies that this activity is not necessary for the basic locomotor pattern. What then is its role?

Lesions of the lateral descending pathways (containing corticospinal and rubrospinal projections) produce a long term impairment in the ability of cats to step over obstacles \citep{Drew2002}. Recordings of PTN neurons during locomotion show increased activity during these visually guided modifications to the basic step cycle \citep{Drew1996}. These observations suggest that motor cortex neurons are necessary for precise stepping and adjustment of ongoing locomotion to changing conditions. However, long-term effects seem to require complete lesion of \emph{both} the corticospinal and rubrospinal tracts \citep{Drew2002}. Even in these animals, the voluntary act of stepping over an obstacle does not disappear entirely, and moreover, they can adapt to changes in the height of the obstacles \citep{Drew2002}. Although they never regain the ability to gracefully clear an obstacle, these animals still adjust their stepping height when faced with a higher obstacle in such a way that would have allowed them to comfortably clear the lower obstacle \citep{Drew2002}. Furthermore, deficits caused by lesions restricted to the pyramidal tract seem to disappear over time \citep{Liddell1944}, and are most clearly visible only the first time an animal encounters a new obstacle \citep{Liddell1944}.

The view that motor cortex in non-primate mammals is principally responsible for adjusting ongoing movement patterns generated by lower brain structures is appealing. What is this modulation good for? What does it allow an animal to achieve? How can we assay its necessity?

\subsection{Towards a new teleology; new experiments required}

It should now be clear that the involvement of motor cortex in the direct control of all ``voluntary movement'' is human-specific. There is a role for motor cortex across mammals in the control of precise movements of the extremities, especially those requiring individual movements of the fingers, but these effects are subtle in non-primate mammals. Furthermore, what would be a devastating impairment for humans may not be so severe for mammals that do not depend on precision finger movements for survival. Therefore, generalizing this specific role of motor cortex from humans to all other mammals would be misleading. We could be missing another, more primordial role for this structure that predominates in other mammals, and by doing so, we may also be missing an important role in humans.

The proposal that motor cortex induces modifications of ongoing movement synergies, prompted by the electrophysiological studies of cat locomotion, definitely points to a role consistent with the results of various lesion studies. However, in assays used thus far, the ability to modify ongoing movement generally recovers after a motor cortical lesion. What are the environmental situations in which motor cortical modulation is most useful?

Cortex has long been proposed to be the structure responsible for integrating a representation of the world and improving the predictive power of this representation with experience \citep{Barlow1985,Doya1999}. If motor cortex is the means by which these representations can gain influence over the body, however subtle and ``modulatory'', can we find situations (i.e. tasks) in which this cortical control is required?

The necessity of cortex for various behavioural tasks has been actively investigated in experimental psychology for over a century, including the foundational work of Karl Lashley and his students \citep{Lashley1921a,Lashley1950a}. In the rat, large cortical lesions were found to produce little to no impairment in movement control, and even deficits in learning and decision making abilities were difficult to demonstrate consistently over repeated trials. However, Lashley did notice some evidence that cortical control may be involved in postural adaptations to unexpected perturbations \citep{Lashley1921a}. These studies once again seem to recapitulate the two most consistent observations found across the entire motor cortical lesion literature in non-primate mammals since Hitzig \citep{Fritsch1870}, Goltz \citep{Goltz1888}, Sherrington \citep{Sherrington1885} and others \citep{Oakley1979,Terry1989}. One, direct voluntary control over movement is most definitely not abolished through lesion; and two, certain aspects of some movements are definitely impaired, but only under certain challenging situations. The latter are often reported only anecdotally. It was this collection of intriguing observations in animals with motor cortical lesions that prompted us to expand the scope of standard laboratory tasks to include a broader range of motor control challenges that brains encounter in their natural environments.

\section{Experiment Introduction}

In the natural world, an animal must be able to adapt locomotion to any surface, not only in anticipation of upcoming terrain, but also in response to the unexpected perturbations that often occur during movement. This allows animals to move robustly through the world, even when navigating a changing environment. Testing the ability of the motor system to generate a robust response to an unexpected change can be difficult as it requires introducing a perturbation without cueing the animal about the altered state of the world. Marple-Horvat and colleagues built a circular ladder assay for cats that was specifically designed to record from motor cortex during such conditions \citep{Marple-Horvat1993}. One of the modifications they introduced was to make one of the rungs of the ladder fall unexpectedly under the weight of the animal. When they recorded from motor cortical neurons during the rung drop, they noticed a marked increase in activity, well above the recorded baseline from normal stepping, as the animal recovered from the fall and resumed walking. However, whether this increased activity of motor cortex was necessary for the recovery response has never been assayed.

\begin{featurebox}
\caption{Some cautionary remarks on lesion techniques}

The original methods used to induce a permanent lesion to the motor cortex were very crude, often involving gross mechanical aggression to the neural tissue by using surgical knife cuts or ablation by water-jet, aspiration, and thermo- or electrocoagulation. These methods are still widely used in lesion studies for their simplicity and bluntness, but have the disadvantage of making it hard to limit the lesion to a single area because of possible damage to subcortical areas or the destruction of fibers of passage. These limitations made it more difficult to interpret the effects of cortical lesions, and eventually led to the development of new techniques designed to work around such problems. Chemical injections of neurotoxic compounds such as ibotenic acid or kainic acid aim to increase selectivity of the lesion by limiting damage to neural cell bodies in the target area while leaving the fibers of passage intact \citep{Schwarcz1979}. Photothrombosis \citep{Watson1985} or devascularization by pial stripping \citep{Meyer1971} aim to reproduce the effects of clinical stroke while avoiding extension of the lesion to subcortical areas as much as possible.

The early studies of Broca localizing the function of articulate language to a specific region in the cerebral hemispheres \citep{Broca1861} established a long tradition of correlating the location of surgical brain injury with detailed analysis of any subsequent behavioural deficits. This method is not without its difficulties. The problems of plasticity and diaschisis will forever complicate conclusions based on injury and manipulation of nervous tissue \citep{Lashley1933}. Many recent methods for reversible chemical or optogenetic inactivation of the cortex have been proposed to improve statistical power of behavioural assessments \citep{DeFeudis1980,Dong2010,Guo2015}. Unfortunately, given that the cortex maintains a tight balance of excitation and inhibition during normal functioning and is also densely interconnected with the rest of the brain, the effects of such transient manipulations are prone to cause multiple downstream effects that can confound inferences about behavioural relevance \citep{Otchy2015}. In this respect, they are similar to stimulation experiments in that they are very useful in determining that two areas are connected in a circuit, but not necessarily what the connection means. Of course, permanent lesions themselves can induce plasticity changes in the function of downstream and upstream circuits. The expectation, however, is that such changes represent a homeostatically stable state of the system, allowing simultaneous investigation of the limits of recovery, as well as the kinds of problems for which a fully intact structure is definitely required.
\end{featurebox}
