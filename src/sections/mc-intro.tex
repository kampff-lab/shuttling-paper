\section{Introduction}

Since its discovery 150 years ago, the role of the motor cortex has been a topic of controversy and confusion \cite{Lashley1924}. Here we report our efforts to piece together a teleology for cortical motor control. It is a story about what motor cortex may be good for, rather than how it works. Motor cortex has been implied in ``understanding'' the observed movements of others \cite{Rizzolatti2004}, imagining one's own movements, or in learning new movements, but here we will focus on its role in directly controlling movement.

Motor cortex is still broadly defined as the region of the cerebral hemispheres from which movements can be evoked by low-current surface stimulation, following Fritsch and Hitzig's original experiments in 1870 \cite{Fritsch1870}. Stimulation of different parts of the motor cortex elicit movements in different parts of the body, establishing a topographical representation covering the entire skeletal musculature across cortical surface \cite{Leyton1917,Penfield1937,Neafsey1986}. Electrophysiological recordings have found correlations of motor cortical activity with all kinds of movement parameters including muscle force, speed or target endpoint, at both single neuron and population levels \cite{Georgopoulos1986}. There is a long standing debate regarding what, exactly, does motor cortex control, muscles or movements \cite{Todorov2000}. In fact, recent stimulation studies have even suggested that the entire natural repertoire of animal behaviour is represented across the surface of motor cortex \cite{Graziano2002,Aflalo2006}. Together, these observations have led to the long standing idea that motor cortex is the part of the brain responsible for the control of voluntary movements.

However, there is a nagging dilemma in the field of cortical motor control which is equally old. In humans, the result of motor cortical stroke can be devastating, resulting in permanent loss of control of muscle movements or even paralysis; motor control is permanently and obviously impaired. In non-human primates, the primary reported effect of motor cortical lesion is a clear loss of motility in distal forelimb, especially in the control of fine, individual finger movements, which are required for the execution of precision skills \cite{Leyton1917,Darling2011}. But equally impressive is the extent to which other behaviours fully recover, including the ability to sit, stand, walk, climb and even reach accurately for food, as long as precise finger movements are not required \cite{Leyton1917,Darling2011}. In other mammals, the absence of lasting effects following motor cortical lesion is even more dramatic. Careful studies of skilled reaching in the rodent have revealed subtle impairments in grasping behaviours \cite{Alaverdashvili2008a}, comparable to the loss of precision finger movements in primates, but this loss pales in comparison to the range of movements that \emph{are} preserved. In fact, even following complete decortication, rats and cats retain a shocking amount of their original behavioural repertoire \cite{Bjursten1976,Terry1989}. If we accept the simple hypothesis that motor cortex is the structure responsible for ``voluntary movement production'', then why is there such a huge difference in the effects of motor cortical lesions between humans and other mammals? Do non-human mammals only utilize cortical motor control for precision movements involving the distal musculature? The results of the numerous stimulation and electrophysiology studies do seem to suggest the involvement of this structure with all kinds of movements, across all mammalian species, so how can these two disparate views be rectified?

A partial explanation for this dilemma has been found in the anatomy of the descending cortical motor pathways. In primates, the conspicuous effects of motor cortical lesion can be reproduced by complete section of the pyramidal tract, the direct monosynaptic projection that links motor cortex, and other cortical regions, to the spinal cord \cite{Tower1940,Lawrence1968}. This corticospinal tract is thought to support the low-current movement responses caused by electrical stimulation in the cortex, as observed by the increased difficulty in obtaining stimulation effects following section of the tract at the level of the medulla \cite{Woolsey1972}. In the monkey, and similarly in man, this fibre system has been found to directly terminate on spinal motor neurons responsible for the control of distal musculature \cite{Leyton1917,Bernhard1954}. However, in all other mammals, including cats and rats, the termination pattern of the pyramidal tract in the cord largely avoids these ventral motor neuron pools and concentrates instead on intermediate zone interneurons \cite{Kuypers1981,Yang2003}. Furthermore, the rubrospinal tract, an important descending pathway that overlaps with the corticospinal terminations in the intermediate zone, is much degenerated in humans when compared to primates and other mammals \cite{Square1982}, and is thought to play a role in compensating for the loss of the pyramidal tract \cite{Lawrence1968a}.

These differences in anatomy might explain the lack of noticeable, lasting movement deficits in non-primates, but begs the question of what motor cortex is good for in all these other mammals, including our oldest known common ancestors. In the rat, a large swath (what percentage?) of cortex is considered ``motor'' based on anatomical \cite{Donoghue1982}, stimulation \cite{Donoghue1982,Neafsey1986} and electrophysiological grounds. However, the most consistently observed long-term effect following motor cortical lesion has been an impairment in supination of the wrist which impairs reaching for food pellets through a narrow vertical slit. Are we to believe that this massive high-level motor control structure, with dense efferent projections not only to the spinal cord but to basal ganglia, cerebellum and other brainstem movement related areas, as well as to most primary sensory areas, has evolved simply to allow more precise wrist rotations? What did this cortical structure influence about movement and behaviour in our primordial relatives? If nothing, then why did it evolve? Are there other functions in the control of movement underlying motor cortical circuits and its descending projections that we may be missing with our current assays?

A different perspective on this issue has emerged from studies in the neural control of locomotion, particularly in cats, which suggest that the corticospinal tract may play a role in the adjustment of ongoing movements that are themselves generated by lower motor systems. In this view, rather than motor cortex assuming direct control over muscle movement, it might instead modulate the activity and sensory feedback in spinal circuits in order to adapt the movement to changing conditions. The idea that the descending pathways that make up the lateral system can superimpose speed and precision on an existing baseline of behaviour was also suggested by the work in primates, but has been investigated most thoroughly in the context of cat locomotion.

It has been known for more than a century that completely decerebrate cats are still capable of sustaining the basic locomotor rhythm necessary for walking on a flat treadmill utilizing only spinal circuits \cite{GrahamBrown1911}. Brainstem circuits are sufficient to initiate the activity of these central patttern generators \cite{Grillner1973}, so what are the contributions of motor cortex to the control of locomotion? Single-unit recordings of pyramidal tract neurons (PTNs) during treadmill locomotion have shown that a large proportion of these neurons are locked to the step cycle \cite{Armstrong1984a}. However, we know from the decerebrate studies that this activity is not required for the basic movement pattern, so what is its functional role?

Lesions of the lateral descending pathways have revealed a long term impairment in the ability to step over obstacles \cite{Drew2002}. Recordings of PTN neurons during locomotion show increased activity during execution of these visually guided modifications to the step cycle \cite{Drew1996}. This led to the suggestion that motor cortex neurons are necessary for precise stepping and for superimposing movement modifications on top of ongoing muscle synergies. However, long-term effects seem to require complete lesion of \emph{both} the corticospinal and rubrospinal tracts \cite{Drew2002}. Even in these animals, the voluntary act of stepping over an obstacle does not disappear entirely and in fact it can even be adapted to changes in the height of the obstacles \cite{Drew2002}. Specifically, even though animals never regain the ability to gracefully clear an obstacle without touching it, they seem to adjust their stepping height to a higher obstacle in such a way that would allow them to clear the lower obstacles. On the other hand, effects of lesions restricted to the pyramidal tract seem to disappear over time \cite{Liddell1944}, and are more clearly visible only the first time an animal encounters the obstacles \cite{Liddell1944}.

What can we take away from all these studies? It is becoming clear that the involvement of motor cortex in the control of all ``voluntary movement'' is human-specific. There is a clear role of motor cortex across mammals in the control of precise movements of the extremities, especially those requiring breaking apart lower movement patterns and synergies such as individual movements of the fingers, but these effects are subtle in non-primate mammals. Is this really the only reason why we have a motor cortex? In other mammals many other precise movements do not depend on motor cortex for their execution. Thus, generalizing this specific role of motor cortex from humans to all other mammals can be misleading. We may be missing some other more primordial roles for this structure that predominate in other mammals, and by doing so, we may also be missing something important for humans.

The suggestion that motor cortex produces modifications to ongoing movement patterns, as suggested by the electrophysiological studies of cat locomotion, definitely points to a satisfying role that could potentially rectify the results of lesions. However, it seems that even these modifications can generally recover from a motor cortical lesion. What exactly does cortex add to movement? When would such cortical control be required?

Cortex has long been thought to be the fundamental structure for integrating a representation about the contextual state of the world and for understanding how it works from experience. If motor cortex is the means by which these representations can gain influence over the body, can we find situations where they would be required for adaptive behaviour? Cortical control over behaviour has been actively investigated in experimental psychology by the foundational work of Karl Lashley and many of his students. In the rat, large cortical lesions were found to produce only slight impairments in movement control, and even deficits in learning and decision making abilities were hard to demonstrate consistently over repeated trials. There is some evidence that cortical control may be involved in postural adaptations to unexpected situations \cite{Lashley1921a}, but we may need to expand the scope of laboratory tasks to address a broader range of ``natural'' problems that brains would normally encounter in order to clarify the cortical contributions to movement in non-primate mammals. Specifically, we may just be missing some obvious roles because our assessment is focused on tasks in which humans are impaired following motor cortical lesion.

In the following, we report an experiment that was designed precisely to simultaneously assay the role of motor cortex in the direct control of movement as well as provide opportunities to expose the animals to more natural and challenging situations. The results of this experiment have led us to reformulate a new hypothesis for motor cortical control that we outline and address in the discussion.

\section{Experiment Introduction}

In the natural world it is important to be able to adapt our locomotion to any given surface, not only in anticipation of the movement conditions that it will afford, but also in response to both expected and unexpected perturbations that may happen in the course of movement. This allows animals to move robustly even in the face of changing environmental conditions. Testing this kind of robust control during locomotion can be difficult as it requires introducing a mechanical perturbation without queueing the animal about the altered state of the world. Marple-Horvat and colleagues built a circular ladder specifically designed to record from motor cortex during exactly such conditions \cite{Marple-Horvat1993}. One of the modifications they introduced was to make one of the rungs of the ladder fall unexpectedly under the weight of the animal. When they now looked at the activity of motor cortical neurons during the rung drop, they noticed an increased pattern of activity, above the recorded baseline from normal stepping, as the animal recovered from the fall and resumed walking. However, whether this increased activity of motor cortex was necessary for the recovery response has never been assayed.

To test whether the intact motor cortex is required for robust control, we designed a reconfigurable dynamic obstacle course where individual steps can be made stable or unstable on a trial-by-trial basis. In this assay, rats are motivated to go back and forth across the obstacles, in the dark, in order to collect water rewards. We specifically designed the assay such that modifications to the physics of the obstacles could be made covertly. In this way, the animal has no explicit information about the state of the steps until he actually walks over them. Our goal was to characterize precisely the conditions under which motor cortex becomes necessary for the control of movement, which motivated us to introduce an environment with graded levels of uncertainty.

In this experiment, we assessed the role of motor cortical structures by making targeted lesions to areas responsible for forepaw control. Assaying the relevance of a brain structure has long involved surgical lesions of that structure and subsequent detailed analysis of any behavioural deficits. This method is not without its difficulties. The problems of plasticity and diaschisis seem to forever plague our conclusions based on injury and manipulations of nervous tissue. Many recent methods for reversible chemical or optogenetic inactivation of the cortex have been proposed to improve statistical power of behavioural assessments. Unfortunately, given that the cortex maintains a tight balance of excitation and inhibition during its functioning, the effects of such transient manipulations are prone to cause multiple downstream effects that can confound inferences about behavioural relevance \cite{Otchy2015}. In this respect, they are similar to stimulation experiments in that they are very useful in determining that two areas are connected in a circuit, but not necessarily what the connection means. Of course, permanent lesions themselves can induce plasticity changes in the function of downstream and upstream circuits but such changes should represent a homeostatically stable state of the system. In this way, we hope to simultaneously probe the limits of recovery, as well as the kinds of problems for which a fully intact structure is definitely required.

We first describe the assay and our detailed methods, and proceed to describe the behaviour of animals faced with different degrees of uncertainty in this motor control problem. We then describe and discuss the differences in the behaviour of animals with and without motor cortex under different situations and conclude with a discussion of the implications of the results in the context of our understanding of motor cortex, ultimately proposing a new (more robust) role for motor cortex in non-primate mammals.
