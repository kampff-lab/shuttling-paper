\section{Introduction}

Since its discovery 150 years ago, the role of motor cortex has been a topic of controversy and confusion \cite{Lashley1924}. Here we report our efforts to establish a teleology for cortical motor control. Motor cortex may play roles in ``understanding'' the movements of others \cite{Rizzolatti2004}, imagining one's own movements, or in learning new movements \cite{Kawai2015}, but here we will only focus on its role in directly controlling movement.

\subsubsection*{Stimulating motor cortex causes movement; motor cortex is active during movement}

Motor cortex is broadly defined as the region of the cerebral hemispheres from which movements can be evoked by low-current stimulation, following Fritsch and Hitzig's original experiments in 1870 \cite{Fritsch1870}. Stimulating different parts of the motor cortex elicits movement in different parts of the body, and systematic stimulation surveys have revealed a topographical representation of the entire skeletal musculature across the cortical surface \cite{Leyton1917, Penfield1937, Neafsey1986}. Electrophysiological recordings in motor cortex have routinely found correlations between neural activity and many different movement parameters, such as muscle force \cite{Evarts1968}, movement direction \cite{Georgopoulos1986}, speed \cite{Schwartz1993}, or even anisotropic limb mechanics \cite{Scott2001} at the level of both single neurons \cite{Evarts1968,Churchland2007} and populations \cite{Georgopoulos1986,Churchland2012}. Determining what exactly this activity in motor cortex controls \cite{Todorov2000} is further complicated by studies using long stimulation durations in which continuous stimulation at a single location in motor cortex evokes complex, multi-muscle movements \cite{Graziano2002,Aflalo2006}. However, as a whole, these observations all support the long standing view that activity in motor cortex is involved in the direct control of movement.

\subsubsection*{Motor cortex lesions produce different deficits in different species}

What types of movement require motor cortex? In humans, a motor cortical lesion is devastating, resulting in the loss of muscle control or even paralysis; movement is permanently and obviously impaired \cite{Laplane1977}. In non-human primates, similar gross movement deficits are observed after lesions, albeit transiently \cite{Leyton1917}. The longest lasting effect of a motor cortical lesion is the decreased motility of distal forelimbs, especially control of the individual finger movements required for precision skills \cite{Leyton1917,Darling2011}. But equally impressive is the extent to which other movements fully recover, including the ability to sit, stand, walk, climb and even reach to grasp, as long as precise finger movements are not required \cite{Leyton1917,Darling2011,Zaaimi2012}. In non-primate mammals, the absence of lasting deficits following motor cortical lesion is even more striking. Careful studies of skilled reaching in the rodent have revealed an impairment in paw grasping behaviours \cite{Alaverdashvili2008a}, comparable to the long lasting deficits seen in primates, but this is a limited impairment when compared to the range of movements that \emph{are} preserved. In fact, even after complete decortication, rats, cats and dogs retain a shocking amount of their movement repertoire \cite{Bjursten1976,Terry1989}. If we are to accept the simple hypothesis that motor cortex is the structure responsible for ``voluntary movement production'', then why is there such a blatant difference in the severity of deficits caused by motor cortical lesions in humans versus other mammals? With over a century of stimulation and electrophysiology studies clearly suggesting that motor cortex is involved in many types of movement, in all mammalian species, how can these divergent results be aligned?

\subsubsection*{There are anatomical differences in corticospinal projections between primates and other mammals}

In primates, the conspicuous effects of motor cortical lesion can also be produced by sectioning the pyramidal tract, the direct monosynaptic projection that links motor cortex, and other cortical regions, to the spinal cord \cite{Tower1940,Lawrence1968}. The corticospinal tract is thought to support the low-current movement responses evoked by electrical stimulation in the cortex, as evidenced by the increased difficulty in obtaining a stimulation response following section at the level of the medulla \cite{Woolsey1972}. In the monkey, and similarly in man, this fibre system has been found to directly terminate on spinal motor neurons responsible for the control of distal musculature \cite{Leyton1917,Bernhard1954}. However, in all other mammals, including cats and rats, the termination pattern of the pyramidal tract in the cord largely avoids these ventral motor neuron pools and concentrates instead on intermediate zone interneurons and dorsal sensory neurons \cite{Kuypers1981,Yang2003}. Furthermore, the rubrospinal tract, a descending pathway originating in the brainstem and terminating in the intermediate zone, is degenerated in humans compared to other primates and mammals \cite{Square1982}, and is thought to play a role in compensating for the loss of the pyramidal tract in non-human species \cite{Lawrence1968a,Zaaimi2012}. These differences in anatomy might explain the lack of conspicuous, lasting movement deficits in non-primates, but leaves behind a significant question: what is the motor cortex actually controlling in all these other mammals?

\subsubsection*{What is the role of motor cortex in lower mammals?}

In the rat, a large portion of cortex is considered ``motor'' based on anatomical \cite{Donoghue1982}, stimulation \cite{Donoghue1982,Neafsey1986} and electrophysiological evidence \cite{Hyland1998}. However, the most consistently observed long-term deficit following motor cortical lesions has been an impairment in supination of the wrist and individuation of digits during grasping, which in turn impairs reaching for food pellets through a narrow vertical slit \cite{Alaverdashvili2008a}. Despite the fact that activity in rodent motor cortex has been correlated with movements of the entire skeletal musculature (not just distal limbs) \cite{Hill2011,Erlich2011}, it appears we must conclude that this large high-level motor structure, with dense efferent projections to motor areas in the spinal cord, basal ganglia, thalamus \cite{Lee2008}, cerebellum and brainstem, as well as to most primary sensory areas \cite{Petreanu2012,Schneider2014}, evolved simply to facilitate more precise wrist rotations and grasping gestures. Or maybe we are missing something. Might there be other problems in movement control that motor cortex is solving, but that we may be missing with our current assays?

\subsubsection*{A role in modulating the movements generated by lower motor centres}

A different perspective on motor cortex emerged from studying the neural control of locomotion, particularly in cats, suggesting that the corticospinal tract plays a role in the \emph{adjustment} of ongoing movements that are generated by lower motor systems. In this view, rather than motor cortex assuming direct control over muscle movement, it modulates the activity and sensory feedback in spinal circuits in order to adapt a lower movement controller to challenging conditions. The idea that the descending cortical pathways superimpose speed and precision on an existing baseline of behaviour was also suggested by lesion work in primates \cite{Lawrence1968a}, but has been investigated most thoroughly in the context of cat locomotion.

It has been known for more than a century that completely decerebrate cats are capable of sustaining the basic locomotor rhythm necessary for walking on a flat treadmill utilizing only spinal circuits \cite{GrahamBrown1911}. Brainstem circuits are sufficient to initiate the activity of these spinal central pattern generators \cite{Grillner1973}, so what exactly is the contribution of motor cortex to the control of locomotion? Single-unit recordings of pyramidal tract neurons (PTNs) during treadmill locomotion have shown that a large proportion of these neurons are locked to the step cycle \cite{Armstrong1984a}. However, we know from the decerebrate studies that this activity is not necessary for the basic movement pattern; what is its role?

Lesions of the lateral descending pathways (containing corticospinal and corticobulbar projections) produce a long term impairment in the ability of cats to step over obstacles \cite{Drew2002}. Recordings of PTN neurons during locomotion show increased activity during execution of these visually guided modifications to the step cycle \cite{Drew1996}. These observations suggested that motor cortex neurons are necessary for precise stepping and for superimposing movement modifications on top of an ongoing movement pattern. However, long-term effects seem to require complete lesion of \emph{both} the corticospinal and rubrospinal tracts \cite{Drew2002}. Even in these animals, the voluntary act of stepping over an obstacle does not disappear entirely and it can adapt to changes in the height of the obstacles \cite{Drew2002}. Specifically, even though animals never regain the ability to gracefully clear an obstacle without touching it, they still seem to adjust their stepping height to a high obstacle in such a way that would allow them to comfortably clear lower obstacles \cite{Drew2002}. On the other hand, deficits caused by lesions restricted to the pyramidal tract seem to disappear over time \cite{Liddell1944}, and are most clearly visible only the first time an animal encounters a new obstacle \cite{Liddell1944}.

The view that motor cortex in non-primate mammals is principally responsible for adjusting ongoing movement patterns generated by lower brain structures is appealing. What is this modulation good for? What does it allow an animal to achieve? How can we assay its necessity?

\subsubsection*{Towards a new teleology; new experiments required}

It should now be clear that the involvement of motor cortex in the control of all ``voluntary movement'' is human-specific. There is a role for motor cortex across mammals in the control of precise movements of the extremities, especially those requiring individual movements of the fingers, but these effects are subtle in non-primate mammals. Furthermore, what would be a severe impairment for humans may not be so devastating for mammals that do not depend on precision finger movements for survival. Therefore, generalizing this specific role of motor cortex from humans to all other mammals would be misleading. We could be missing another, more primordial, role for this structure that predominates in other mammals, and by doing so, we may also be missing another important role in humans.

The proposal that motor cortex induces modifications of ongoing movement patterns, prompted by the electrophysiological studies of cat locomotion, definitely points to a role consistent with the results of various lesion studies. However, in assays used, the ability to modify ongoing movement generally recovers after a motor cortical lesion. What are the environmental siutations in which motor cortical modulation is most useful?

Cortex has long been proposed to be the structure responsible for integrating a representation of the world and improving the predictive power of this representation with experience (citation). If motor cortex is the means by which these representations can gain influence over the body, however subtle and ``modulatory'', can we find situations (i.e. tasks) in which this cortical control is required?

The necessity of cortex for various behavioural tasks has been actively investigated in experimental psychology for over a century, including the foundational work of Karl Lashley and his students \cite{Lashley1950a}. In the rat, large cortical lesions were found to produce little to no impairment in movement control, and even deficits in learning and decision making abilities were difficult to demonstrate consistently over repeated trials. However, Lashley also noted some evidence that cortical control may be involved in postural adaptations to unexpected situations \cite{Lashley1921a}. This collection of intriguing observations and anecdotes of behaviour in animals with motor cortical (Cite: goltz-ferrier) lesions has prompted us to expand the scope of standard laboratory tasks to include a broader range of motor control problems that brains encounter in their natural environments.

In the following, we report an experiment that was designed to provide controlled exposure of animals to more naturally challenging environments. The results of this experiment have led us to formulate a new teleology for cortical motor control that we will present in the discussion.

\section{Experiment Introduction}

In the natural world, an animal must be able to adapt locomotion to any surface, not only in anticipation of upcoming terrain, but also in response to the unexpected perturbations that often occur during movement. This allows animals to move robustly through the world, even when navigating a changing environment. Testing the ability of the motor system to generate a robust response to an unexpected change can be difficult as it requires introducing a perturbation without cueing the animal about the altered state of the world. Marple-Horvat and colleagues built a circular ladder assay for cats that was specifically designed to record from motor cortex during such conditions \cite{Marple-Horvat1993}. One of the modifications they introduced was to make one of the rungs of the ladder fall unexpectedly under the weight of the animal. When they recorded from motor cortical neurons during the rung drop, they noticed a marked increase in activity, well above the recorded baseline from normal stepping, as the animal recovered from the fall and resumed walking. However, whether this increased activity of motor cortex was necessary for the recovery response has never been assayed.
