\section{Introduction}

Here we will discuss the role of motor cortex in controlling movement.  Motor cortex may play a role in "understanding" the observed movements of others, imaging your own movements, or in learning new movements, but here we focus on its role in controlling movement. How does neural activity in cortical motor areas influence movement of the body? Are there specific types of movements for which motor cortex is required? Are there behaviours that an animal cannot perform without motor cortex? Motor cortex is active during movement, stimulating motor cortex ellicits movevement, and lesions to motor cortex (in some mammals) cause lasting deficits in motor control. Its role seems obvious: motor cortex is the part of cortex that produces movement of the body. Yet, it is not so simple.

We were motivated to open this discussion based on the following observation: when a human motor cortex is damaged, by stroke or other lesion, then a paitient suffers severe and conspicuous deficits in movement. However, when lower mammals (rats, cats, dogs, and monkeys) suffer similar damage, then the effect on movement is much less obvious. In fact, rats without motor cortex, or even most of cortex, do not show lasting problems in movement...as assayed with a battery of standard behavioural tests. This presents a dillemma. Although it is clear that motor cortex is important for movement in humans (e.g. humans with lesions of secondary motor areas called "Broca's area" lose, and may never recover, the ability to speak), but thus is much less clear in other mammals. Is there a general "role" for motor cortex? We will try to describe such a general role, and to provide an explanation for the apparent human-rat discrepancy, but to do so, we must first review a century of evidence that has led to our current views about what motor cortex is and what it is good for. 

This discrepancy was apparent from the beginning. 

Our review will begin with the discovery of motor cortex. In the 1880s, a lovely series of experiments isolated a region of the cortex that, when stimulated, could evoke movements. However, lesioning this area produced confusing results (and they still do). Monkeys with lesions showed human-like gross motor impariments, but dogs with similar lesion extents recovered remarkably well. These seemingly conflicting results were largely ignored...primates were more closely related to humans, so the basic premise that a localized part of cortex moves "us" both was accepted, and it's just different in dogs. What does dog (or cat and rat) motor cortex do? An open question.

We all evoled from a little rodent-like creature. It had an expanded telencephalon and likely a cortical organization, which began to expand, ultimately becoming 70% of the human brain. What did this cortical structure influe nce about movement and behaviour in thes eprimoridial relatives? If nothing, then why did it evolve? 

We then survey studies of motor cortex anatomy and its output projections. It was 

We will discuss the physiology of motor cortex, both recording experiments that have identified a number

However, we will priminarily focus on experiments that disrupt the function of motor cortex and then chacraterize changes in behaviour. These experiments have involved a range of lesion size, form thos targeting "primary" motor cortical areas to complete decortications (i.e. total removal of cortex). These results reveal a clear pattern. As one transcends the "mammalian hieracrchy" the overt, conspicuous deficits in behaviour resulting from damage to motor cortex increase. Humans not only need motor cortex to spear, they are the only species that needs motor cortex to walk! Rats seem fine, cats and dogs as well, whereas primates are more obviously impaired, yet do show impresisve recovery. This graded importance of motor cortex for movement immediately suggests that motor cortex plays an increasingly "important" role in higer, more encephalized, mammals. However, as we shall see

comment on dextererity and volutnary control/


After review our past, we will revisit the dilemma of motro cortex from another perspective. What WOULD we need a motor cortex for? What are the HARD motor control probelms that we ecounter in the environment for which a higher motor control system would be needed? Traditional assays of behaviour, and the deficits that results of lesions to motor cortex, do not necessarly test the range of problems animals would have encountered in their envieoment,mes...amd would have required elegant solutions to. We will discuss the ecological dilemmas of animals by caharcateizing the nature of problems faced by those scientists trying to build new machines for the natural enviornment, robotics. Here the probelsm are clear...and they are not the ones for which we have ascribed a role to our brain's highest motor control system. Moving with precision and "fine dexterity" is not hard for modern robots. We rely on precisely this skill in all industrila robotics, many of which vastly exceed human level performance. Yet, these robots are not wandering around amongst us. Why? They fall over.

Here is the dilemma, and here is a new probelm...do they fit tgether? Have they been addrssed before. 

We will next outline the histriy of robustness and its cortical requirement. 

Then we will propose and run a simple experiment. How do rats repsond to unexpected problems. We wil present a new assay, and assess the role of motor cortex by making targeted lesions to areas responsible for forepaw control. These results suggest a ...but will require an entirely new kind of assays for motor cortical function. Thus we will conclude with some pontificatiin. What does this new framework for otor cortex (and even cortex in general) mean for futue neuoscience? What does it mean for the evoiltuon of mammalian brains?

There is a lot to say, and we will say it, but mind you, this is just the beginning. We wre likely wrong and major and minor points...but we want a more "rubust" understanding of cortex. So we start a conversation.



box: What do we mean by "Motor Cortex"?



box: What do we mean be teleology?

