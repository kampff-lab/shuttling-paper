\section{Review}

In the latter half of the 19th century, Hughlings Jackson, a prominent physician often considered the father of neurology, underwent a series of studies on chronic epileptic convulsions \cite{Jackson1870}. He observed that in some patients an epileptic fit would start by deliberate spasm on one side of the body, and that different body parts would become systematically affected one after the other. In his 1870 publication, he connected the orderly march of these spasms to the existence of localized cortical lesions in the \emph{post-mortem} brain of his patients, and advanced the bold proposition that the origin of these fits was uncontrolled spreading excitation caused by local changes in cortical grey matter \cite{Jackson1870}. This hypothesis was controversial since around that time the entire surface of the cortex was viewed by physiologists as being largely inexcitable \cite{Gross2007}.

Nevertheless, later that same year, Fritsch and Hitzig demonstrated for the first time that it was possible to elicit movements by direct electrical stimulation of the cerebral cortex in dogs \cite{Fritsch1870}. Furthermore, stimulation of different parts of the cortex produced movement in different parts of the body on the contralateral side to the stimulated hemisphere. The correspondence was stable over time, with excitable areas found to be restricted mostly to the anterior regions of the cortex. This single finding sparked a revolution in cortical neurophysiology and neurology. Not only the causal mechanism for epileptic convulsions predicted by Hughlings Jackson had been found, but with it came a possible explanation for how the normal brain might control movement. If a localized map of the body was laid out across the cortical surface, then activation of this \emph{motor cortex} might be the way by which voluntary movement commands are translated into overt motor behaviour \cite{Fritsch1870}. This hypothesis was also consistent with observations of long-term hemiparesis and hemiplegia in human stroke patients.

David Ferrier, a Scottish neurologist deeply impressed by the ideas of Hughlings Jackson and by the positive results of Fritsch and Hitzig's experiments, proceeded to reproduce and expand on their observations across a wide range of mammalian species \cite{Ferrier1873}, producing some of the first comprehensive cortical stimulation maps in primates showing the somatotopic organization of motor cortex. However, while Ferrier's stimulation studies demonstrated how activity in the motor cortex was sufficient to produce a large variety of movements, other researchers like Goltz were reporting how massive cortical lesions failed to demonstrate any visible long-term impairments in the motor behaviour of animals \cite{Goltz1888}. These two lines of inquiry first clashed at the seventh International Medical Congress held in London in August 1881, where both Goltz and Ferrier presented their results in a series of debates on the localization of function in the cerebral cortex \cite{Tyler2000}.

Goltz assumed a clear anti-localizationist position. He advanced that it was impossible to produce a complete paresis of any muscle, or complete dysfunction of any perception, by destruction of any part of the cerebral cortex, and that he found mostly deficits of general intelligence in his dogs \cite{Tyler2000}. Following Goltz's presentation, Ferrier emphasized the danger of generalizing from the dog to animals of other orders (e.g. man and monkey). He then proceeded to exhibit his own lesion results by means of antiseptic surgery in the monkey, describing how a circumscribed unilateral lesion of the motor cortex produced complete contralateral paralysis of the leg. He also produced a striking series of microscopic sections of Wallerian degeneration \cite{Waller1850} of the ``motor path'' from the cortex to the contralateral spinal cord, the crossed descending projections forming the pyramidal corticospinal tract \cite{Tyler2000}.

The debates concluded with the public demonstration of live specimens: a dog with large lesions to the parietal and posterior lobes from Goltz; and from Ferrier, a hemiplegic monkey with a unilateral lesion to the motor cortex of the contralateral side. As predicted, Goltz's dog showed a clear ability to locomote and avoid obstacles and to make use of its other basic senses, while displaying peculiar deficits of intelligence such as failing to respond with fear to the cracking of a whip or ignoring tobacco smoke blown to its face. On the other hand, Ferrier's monkey showed up severely hemiplegic, in a condition similar to human stroke patients. After the demonstrations, the animals were killed and their brains removed. Preliminary observations revealed that the lesions in Goltz's dog were less extensive than expected, particularly on the left hemisphere. Ferrier's lesions on the other hand were precisely circumscribed to the contralateral motor cortex. These demonstrations secured the triumph of Ferrier, who went on to firmly establish the localizationist approach to neurology and the idea of a somatotopic organization of the motor cortex translating voluntary commands into motor actions.

