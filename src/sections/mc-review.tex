\section{Review}

In the latter half of the 19th century, Hughlings Jackson, a prominent physician often considered the father of neurology, underwent a series of studies on chronic epileptic convulsions \cite{Jackson1870}. He observed that in some patients an epileptic fit would start by deliberate spasm on one side of the body, and that different body parts would become systematically affected one after the other. In his 1870 publication, he connected the orderly march of these spasms to the existence of localized cortical lesions in the \emph{post-mortem} brain of his patients, and advanced the bold proposition that the origin of these fits was uncontrolled spreading excitation caused by local changes in cortical grey matter \cite{Jackson1870}. This hypothesis was controversial at the time, as the entire surface of the cortex was viewed by physiologists as being completely inexcitable \cite{Gross2007}.

Nevertheless, later that same year, Fritsch and Hitzig demonstrated for the first time that it was possible to elicit movements by direct electrical stimulation of the surface of the cerebral cortex in dogs \cite{Fritsch1870}. Furthermore, they found that stimulation of different parts of the cortex produced movement in different parts of the body on the contralateral side to the stimulated hemisphere. The correspondence between stimulated cortical points and the part of the body that was moved was stable over time, forming a topographically organized representation of the body, with excitable areas found to be restricted mostly to anterior regions of the cortex. This single finding sparked a revolution in cortical neurophysiology and neurology. Not only the causal mechanism for epileptic convulsions predicted by Hughlings Jackson had been found, but with it came a possible explanation for how the normal brain might control movement. If a localized map of the body was laid out across the cortical surface, then activation of this \emph{motor cortex} might be the way by which voluntary movement commands are translated into overt motor behaviour \cite{Fritsch1870}. This hypothesis was also consistent with observations of long-term hemiparesis and hemiplegia in human stroke patients.

David Ferrier, a Scottish neurologist deeply impressed by the ideas of Hughlings Jackson and by the positive results of Fritsch and Hitzig's experiments, outlined an ambitious research programme to reproduce and expand these observations across a wide range of mammalian species \cite{Ferrier1873}. He produced some of the first comprehensive cortical stimulation maps in primates showing the somatotopic organization of the motor cortex and convincingly established the existence of an excitable area as a hallmark of mammalian cortex. However, while Ferrier's stimulation studies demonstrated how activity in the motor cortex was sufficient to produce a large variety of movements, other researchers like Goltz were reporting how massive cortical lesions failed to demonstrate any visible long-term impairments in the motor behaviour of animals \cite{Goltz1888}. These two lines of inquiry first clashed at the seventh International Medical Congress held in London in August 1881, where both Goltz and Ferrier presented their results in a series of debates on the localization of function in the cerebral cortex \cite{Tyler2000}.

Goltz assumed a clear anti-localizationist position. He advanced that it was impossible to produce a complete paresis of any muscle, or complete dysfunction of any perception, by destruction of any part of the cerebral cortex, and that he found mostly deficits of general intelligence in his dogs \cite{Tyler2000}. Following Goltz's presentation, Ferrier emphasized the danger of generalizing from the dog to animals of other orders (e.g. man and monkey). He then proceeded to exhibit his own lesion results by means of antiseptic surgery in the monkey, describing how a circumscribed unilateral lesion of the motor cortex produced complete contralateral paralysis of the leg. He also produced a striking series of microscopic sections tracing a direct descending ``motor path'' from the cortex to the contralateral spinal cord, crossing at the medullary pyramids in the brainstem \cite{Tyler2000}.

The debates concluded with the public demonstration of live specimens: a dog with large bilateral lesions to the parietal and posterior lobes from Goltz; and from Ferrier, a hemiplegic monkey with a unilateral lesion to the motor cortex of the contralateral side. As predicted, Goltz's dog showed a clear ability to locomote and avoid obstacles and to make use of its other basic senses, while displaying peculiar deficits of intelligence such as failing to respond with fear to the cracking of a whip or ignoring tobacco smoke blown to its face. On the other hand, Ferrier's monkey showed up severely hemiplegic, in a condition similar to human stroke patients. After the demonstrations, the animals were killed and their brains removed. Preliminary observations revealed that the lesions in Goltz's dog were less extensive than expected, particularly on the left hemisphere. Ferrier's lesions on the other hand were precisely circumscribed to the contralateral motor cortex. These demonstrations secured the triumph of Ferrier, who went on to firmly establish the localizationist approach to neurology and the idea of a somatotopic organization of the motor cortex translating voluntary commands into motor actions.

Despite being remembered as a victory for Ferrier, the dilemma outlined by Goltz's dogs persists to this day. If motor cortex drives voluntary behaviour in mammalian cortex, why do we see so little impairment in the movements of non-primate mammals with large (motor) cortical lesions? This observation has been reproduced for more than a hundred years in dogs, cats, rats and other species. In fact, Charles Sherrington, throughout his programme to dissect out the circuits underlying spinal reflexes, demonstrated how complex motor responses can be coordinated, sustained, and even modulated appropriately by sensory input even in the complete absence of the brain \cite{Sherrington1906}, so clearly not all movements depend on the presence of the motor cortex. If we are to understand motor cortex as a controller, what kind of movements does it control?

Some explanation for this dilemma was advanced by anatomical studies complementing the lesion and stimulation results. The fibre tract passing through the medullary pyramids was found in man and monkey all throughout the dorsal and lateral parts of the internuncial zone in the spinal cord. Also, in fibres originating from the pre-central gyrus, or motor cortex, these projections were found to target mostly the motor neuron pools in the ventral horn, overlapping with some subcortical descending pathways, such as the rubrospinal tract. Conversely, the targets of other subcortical descending projections to the spinal cord such as the vestibulospinal, tectospinal and reticulospinal tracts are limited to the nucleus proprius in the dorsal horn. While this division into a lateral and medial system mostly holds for other non-primate mammals, the exuberant targeting of motor neuron pools in the ventral horn seems to be a specialization limited to certain primate sub-species. This specialization, combined with a degeneration of the rubrospinal tract in humans, can account for a significant part of the dramatic difference in the effects of lesions to the control of extremities observed in human stroke patients versus cortical lesions in non-human mammals.

The anatomy of the descending corticospinal projections also seems to account for the observed movement responses following surface stimulation of the cortex across mammalian species. Specifically, if the corticospinal fibres are completely sectioned, effects of stimulation can be suppressed. Furthermore, lesions to parts of the cortex that do not contribute to the corticospinal tract will not be followed by any obvious paresis or plegia. This seems to indicate that the transient effects of motor cortical lesions are due to shock, or loss of tonic activity, in the spinal cord from pyramidal tract degeneration. This transient shock is observed to a greater or lesser extent in all mammalian species following loss of the pyramidal tract. Another possibility that is advanced for the motor recovery following lesions is ``vicariation of function'' by other cortical areas, whereby these areas would take over the role of motor cortex following insult or injury to this part of the brain. The two main difficulties for this approach are the irreversible degeneration of the pyramidal tract and the lack of effect of further lesions following complete removal of the motor cortex. Long-term impairments of such lesions have been more difficult to ascertain, but have been roughly described as a ``clumsiness'' in motor movements, which impairs execution of precise movements, including the individual control of digits and altered reflexes.

Interestingly, the most commonly reported behavioural and physiological effects of motor cortical lesions seem to depend in their greatest part on the corticospinal tract, as evidenced by the fact that experiments sectioning only this fibre system are sufficient to reproduce all of the main observed impairments. However, there is a lot of additional anatomy to explain, including descending projections from motor cortex to other subcortical structures involved in movement control such as the basal ganglia, brainstem and cerebellum, as well as recently identified projections to the major sensory processing structures, such as the auditory and visual cortices. Modulation of thalamic inputs seems to be another major motif of motor cortical efferent pathways. In fact, motor cortex stands very high up in the classical van Hessen hierarchy of cortical processing modules, indicating its perceived anatomical role as a major modulatory structure in the brain. This seems at odds with the naive perception of ``primary'' motor cortex as a peripheral motor controller.

Cortical electrophysiology was introduced as a way to try and clarify these dilemmas. The hope was that if the activity of the brain is measured while animals are moving, it might be possible to parcel out the unique contributions of motor cortex to the control of movement. However, recordings of neural activity during movement were unable to recapitulate the species difference and behaviour effects observed from the lesions. The activity across the motor cortex seems to correlate with most observable parameters of the ongoing behaviour of the animal, whether or not the behaviour requires the presence of motor cortex for its execution. In cat locomotion studies, for example, it is possible to isolate individual pyramidal tract neurons in the motor cortex whose activity is time-locked to a specific phase of the step cycle, while the animal is stepping on a flat treadmill. However, even in a decerebrate cat this behaviour can be initiated and sustained entirely by brainstem and spinal cord circuits. If motor cortex is not necessary to produce motor behaviour in this case, what is the functional role of this activity?