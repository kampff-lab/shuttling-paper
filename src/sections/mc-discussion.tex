\section{Experiment Discussion}

Consistent with the previous locomotion studies outlined in the introduction, we did not observe any noticeable effects in movement execution over stable environments in animals with motor cortical lesions. Even when exposed to much harder, unstable, environments, all animals were able to learn an efficient control strategy for locomotion that was able to overcome even the most challenging obstacles, with or without motor cortex. These movement strategies seem to include a preparatory component that takes into account the expected state of the world the animal will encounter. Surprisingly, these preparatory responses did not depend on the presence of frontal cortical structures in these animals.

It was only when the environment did not conform to expectation and demanded a rapid response and change in strategy, that a difference between the two groups of animals could be seen. Animals with extensive damage to the motor cortex were unable to rapidly deploy a change in strategy. Rather, they halted their progression for several seconds, apparently unable to decide on a suitable next strategy. In an ecological situation, such hesitation could easily prove fatal.

\section{Discussion}

Is this the kind of problem worthy of higher level motor cortical control? One of the most commonly emphasized mammalian traits is the vast range of environmental niches that they are able to occupy. This adaptive flexibility requires naturally the ability to evaluate and cope with several unexpected situations that have possibly never even been encountered exactly by any other individual of the same species. Still, mammals are able to thrive in such environments. This success requires, more than precision, resourcefulness and dexterity, the ability to quickly come up with a motor solution for any situation and in any condition \cite{Bernstein1996}. As Bernstein suggested, such dexterity is different from simple harmony in movements or motor precision. It is really required only when there is \enquote{a conglomerate of unexpected, unique complications in the external situations, in a quick succession of motor tasks that are all unlike each other} \cite{Bernstein1996}.

It becomes clear then why the effects of motor cortical lesions have been so hard to report consistently: assays of motor behaviour typically rely on the assessment of situations that are repeated over many trials to allow for improved statistical treatment. It is not easy to constantly produce novel situations that animals have never encountered before, and even if it was, the challenges in analysing these unique situations would certainly be considerable.

Deploying a rapid motor response such as grabbing the handrail when slipping off the stairs or reaching to stop the fall of a cup of coffee that unexpectedly falls from the table is not exactly the kind of feat for which we praise our athletes and sports champions. However, if we consider how research into the field of building intelligent machines has progressed so far, it would not be the first time that we find ourselves humbled by how hard the problems that we consider trivial really are. Vision for example, remains an impressively hard problem for a machine to solve at human-level performance, whereas we can now routinely build computers that are able to beat our best minds at intellectual games such as Checkers \cite{Samuel1959}, Chess \cite{Lai2015} and Go \cite{Silver2016}. In fact, it is harder to get a robot to actually ``see'' the state of the board game and use its own hands to grab the pieces and move them, than to actually compute the best strategy for winning the game.

Conversely, in the field of robotics, feats of precision and fine movements are not the major sources of difficulty. For many years industrial robots have been pushing past human performance in both accuracy and speed of movement execution. More recently, using reinforcement learning techniques, they are even able to learn such movement strategies automatically, given a human-defined goal and many repeated trials for fine-tuning. What are then the hard problems for robotic motor control? Clearly most robots are still confined to industrial factories and controlled spaces. The reason is that as soon as a mobile robot needs to walk on land, a vast number of non-linear, previously unknown situations quickly present themselves. Especially for bipedal platforms, the non-linearities in dealing with these unexpected situations generalize extremely poorly from the statistical machine learning models that can be trained in controlled situations. There is a reason why flying drones took off much sooner than terrestrial robots: they face a much more linear, and predictable, control problem. And yet, even though we now have autopilots for every stage of flight, we still demand a human at the helm of every aeroplane, even with all the risks associated with the human condition. One reason for this that seems relatively consensual is that we simply do not trust machines to face unexpected situations on which they have never been trained before. We don't yet trust a machine to make a quick assessment of a novel situation and realize the implications of a change in context. And perhaps unsurprisingly, context can have dramatic implications on the effects of motor control execution that go way beyond precision in fine movements.

Clearly our results are insufficient to make any outstanding and final conclusions at this point. The challenge of analysing behaviour in unique situations is considerable and indeed places an unfair burden on the familiar statistical methods that have been used for its evaluation. However, we believe that the development of new tools and assays to tackle these challenges will point towards new directions of research that, as we observed already in this simple experiment, can provide fundamental insight into critical problems of motor control that may until now have been largely ignored. Ultimately, they may also help to elucidate and better clarify other symptoms of motor cortical injury that continue to elude our understanding of the role of this structure: unsupported falls are one of the leading causes of injury and death in patients surviving motor cortical stroke \cite{Jacobs2014}.
