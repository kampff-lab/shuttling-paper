\section{Experiment Discussion}

In this experiment, we assessed the role of motor cortical structures by making targeted lesions to areas responsible for forelimb control \cite{Kawai2015,Otchy2015}. Consistent with previous studies, we did not observe any conspicuous deficits in movement execution for rats with bilateral motor cortex lesions when negotiating a stable environment. Even when exposed to a sequence of unstable obstacles, animals were able to learn an efficient strategy for crossing these more challenging environments, with or without motor cortex. These movement strategies also include a preparatory component that might reflect the state of the world an animal expected to encounter. Surprisingly, these preparatory responses also did not require the presence of motor cortex.

It was only when the environment did not conform to expectation, and demanded a rapid adjustment, that a difference between the lesion and control groups was obvious. Animals with extensive damage to the motor cortex did not deploy a change in strategy. Rather, they halted their progression for several seconds, unable to robustly respond to the new motor challenge. In an ecological setting, such hesitation could easily prove fatal.

\section{Extended Discussion}

Is ``robust control'' a problem worthy of high level cortical input? Recovering from a perturbation -- to maintain balance or minimize the impact of a fall -- is a role normally assigned to our lower level postural control systems. The corrective responses embedded in our spinal cord \cite{Sherrington1893b,Sherrington1910}, brainstem \cite{Arshian2014} and midbrain \cite{Grillner1973} are clearly important components of this stabilizing network, but are they sufficient to maintain robust movement in the dynamic environments that we encounter on a daily basis? Some insight into the requirements for a robust control system can be gained from engineering attempts to build robots that navigate in natural environments.

In the field of robotics, feats of precision and fine movement control (the most commonly prescribed role for motor cortex), are not a major source of difficulty. Industrial robots have long since exceeded human performance in both accuracy and execution speed \cite{Senoo2009}. More recently, using reinforcement learning methods, they are now able to automatically learn efficient movement strategies, given a human-defined goal and many repeated trials for fine-tuning \cite{Coates2008}. What then are the hard problems in robotic motor control? Why are most robots still confined to factories, i.e. controlled, predictable environments? The reason is that as soon as a robot encounters natural terrain, a vast number of previously unknown situations arise. The resulting ``perturbations'' are dealt with poorly by the statistical machine learning models that are currently used to train robots in controlled settings.

Let’s consider a familiar example: You are up early on a Sunday morning and head outside to collect the newspaper. It is cold out, so you put on a robe and some slippers, open the front door, and descend the steps leading down to the street in front of your house. Unbeknownst to you, a thin layer of ice has formed overnight and your foot is now quickly sliding out from underneath you. You are about to fall. What do you do? Well, this depends. Is there a railing you can grab to catch yourself? Were you carrying a cup of coffee? Did you notice the frost on the lawn and step cautiously, anticipating a slippery surface? Avoiding a dangerous fall, or recovering gracefully, requires a rich knowledge of the world -- knowledge that is not immediately available to spinal or even brainstem circuits. This rich context relevant for robust movement is readily available in cortex, and cortex alone.

Imagine now that you are tasked with building a robot to collect your morning newspaper. This robot, in order to avoid a catastrophic and costly failure, would need to have all of this contextual knowledge as well. It would need to know about the structure of the local environment (e.g. hand railings that can support its weight), hot liquids and their viscosities, and even the correlation of frozen dew with icy surfaces. To be a truly robust movement machine, a robot must \emph{understand} the physical structure of the world.

Reaching to stop a fall while holding a cup of coffee is not exactly the kind of feat for which we praise our athletes and sports champions, and this might explain why the difficulty of such ``feats of robustness'' are often overlooked. However, it would not be the first time that we find ourselves humbled by the daunting complexity of a problem that we naively assumed was ``trivial''. Vision, for example, has remained an impressively hard task for a machine to solve at human-level performance, yet it was originally proposed as an undergraduate summer project \cite{Papert1966}. Perhaps a similar misestimate has clouded our designation of the hard motor control problems worthy of cortical input.

Inspired by the challenges confronting roboticists, as well as our rodent behavioural results, we are now in a position to posit a new role for motor cortex.

\subsubsection*{A primordial role for motor cortex}

We are seeking a role for motor cortex in non-primate mammals, animals that do not require this structure for overt movement production. The struggles of roboticists highlight the difficulty of building movement systems that robustly adapt to unexpected perturbations, and the results we report in this study suggest that this is, indeed, the most conspicuous deficit for rats lacking motor cortex. So let us propose that, in rodents, motor cortex is primarily responsible for extending the robustness of the subcortical movement systems. It is not required for control in stable, predictable, non-perturbing environments, but instead specifically exerts its influence when unexpected challenges arise. This, we propose, was the original selective pressure for evolving a motor cortex, and thus, its primordial role. This role persists in all mammals, mediated via a modulation of the subcortical motor system (as is emphasized in studies of cat locomotion), and has evolved in primates to include direct control of the skeletal musculature. Our proposal of a ``robust'' teleology for motor cortex has a number of interesting implications.

\subsubsection*{Implications for non-primate mammals}

One of the most impressive traits of mammals is the vast range of environmental niches that they occupy. While most other animals adapt to change over evolutionary time scales, mammals excel in their flexibility, quickly evaluating and responding to unexpected situations, and taking risks even when faced with challenges that have never been previously encountered \cite{Spinka2001}. This success requires more than precision, it requires resourcefulness -- the ability to quickly come up with a motor solution for any situation and under any condition \cite{Bernstein1996}. The Russian neurophysiologist Bernstein referred to this ability with an unconventional definition of ``dexterity'', which he considered to be distinct from a simple harmony and precision of movements. In his words, dexterity is required only when there is \enquote{a conglomerate of unexpected, unique complications in the external situations, [such as] in a quick succession of motor tasks that are all unlike each other} \cite{Bernstein1996}.

If Bernstein’s ``robust dexterity'' is the primary role for motor cortex, then it becomes clear why the effects of lesions have thus far been so hard to characterize: assays of motor behaviour typically evaluate situations that are repeated over many trials in a stable environment. Such repeated tasks were useful, as they offer improved statistical power for quantification and comparison. However, we propose that these conditions specifically exclude the scenarios for which motor cortex originally evolved. It is not easy to repeatedly produce conditions that animals have not previously encountered, and the challenges in analysing these unique situations are considerable.

The assay reported here represents our first attempt at such an experiment, and it has already revealed that such conditions may indeed be necessary to isolate the role of motor cortex in rodents. We thus propose that neuroscience should pursue similar assays, emphasizing unexpected perturbations and novel challenges, and we have developed new hardware and software tools to make their design and implementation much easier \cite{Lopes2015a}.

\subsubsection*{Implications for primate studies}

In contrast to other mammals, primates require motor cortex for the direct control of movement. However, do they also retain its role in generating robust responses? The general paresis, or even paralysis, that results from motor cortical lesions in primates obscures the involvement of cortex in directing rapid responses to perturbations. Yet there is evidence that a role in robust control is still present in primates, including humans. For example, stroke patients with partial lesions to the distributed motor cortical system will often recover the ability to move the affected musculature. However, even after recovering movement, stroke patients are still prone to severe impairments in robust control -- unsupported falls are one of the leading causes of injury and death in patients surviving motor cortical stroke \cite{Jacobs2014}. We thus suggest that stroke therapy, currently focused on regaining direct movement control, should also consider strategies for improving robust responses.

Even if we acknowledge that a primordial role of motor cortex is still apparent in primate movement control, it remains to be explained why the motor cortex of these species acquired direct control of basic movements in the first place. This is an open question, but the consequences are intriguing.

What happens when cortex acquires direct control of movement? First, it must learn how to use this influence, bypassing or modifying lower movement controllers. While functional corticospinal tract connections may be established prenatally \cite{Eyre2000}, effective development of corticospinal dependent movements that override the lower motor system can take much longer in primates and seems to follow the maturation period of corticospinal termination patterns \cite{Lawrence1976}. Humans require years of practice to produce and refine simple locomotion and grasping \cite{Thelen1985,VonHofsten1989}, motor behaviours that are available to other mammals almost immediately after birth. This may be the cost of giving cortex direct control of movement -- it takes more time to figure out how to move the body -- but what is the benefit? 

A robust control system (i.e. cortex), in full command of the body, is capable of negotiating ever more difficult environmental niches. Primates may have been able to ascend trees and negotiate their precarious branches, and thus avoid less ``dexterous'' predators, by relying on more direct motor cortical control. However, the implications of this cortical ``take-over'' are potentially even more profound. 

If cortex has direct control of overt movements, then observing the behaviour of a primate constitutes direct observation of commands arising in cortex (vs. subcortical pattern generators). In other words, when you watch a primate move you are directly observing cortical commands. If a species reliant on direct cortical control happened to live in social groups, then members of this group would have a uniquely efficient means of observing the state of cortex in a conspecific; what was happening in cortex would be directly observable in the movements of each individual. The implications of establishing an efficient means of ``communicating'' the state of one cortex to another are simply exhilarating to imagine...

\subsubsection*{Some preliminary conclusions}

Clearly our results are insufficient to draw any final conclusion, but that is not our main goal. We present these experiments to support and motivate our attempt to distil a long history of research, and ultimately suggest a new approach to investigating the role of motor cortex. This approach most directly applies to studies of non-primate mammals. There is now a host of techniques to monitor and manipulate cortical activity during behaviour in these species, but we propose that we should be monitoring and manipulating activity during behaviours that actually require motor cortex.

This synthesis also has implications for engineers and clinicians. We suggest that acknowledging a primary role for motor cortex in robust control, a problem still daunting to robotics engineers, can guide the development of new approaches for building intelligent machines, as well as new strategies to assess and treat patients with motor cortical damage. We concede that our results are still naïve, but propose that the implications are worthy of further consideration.

