\section{Experiment Discussion}

Consistent with previous studies, we did not observe any conspicuous deficits in movement execution for rats with bilateral motor cortex lesions when negotiating a stable environment. Even when exposed to much unstable obstacles, all animals were able to learn an efficient strategy for crossing these more challenging environments, with or without motor cortex. These movement strategies include a preparatory component that seems to reflect the state of the world an animal expects to encounter. Surprisingly, these preparatory responses also did not require the presence of motor cortex.

It was only when the environment did not conform to expectation, and demanded a rapid adjustment, that a difference between the lesion and control groups was obvious. Animals with extensive damage to the motor cortex did not deploy a change in strategy. Rather, they halted their progression for several seconds, unable to robustly respond to the new motor challenge. In an ecological setting, such hesitation could easily prove fatal.

\section{Discussion}

Is “robust control” a problem worthy of high level motor cortical input? Recovering from a perturbation, to maintain balance or minimize the impact of a fall, is a role normally assigned to our low level posture control system. The corrective responses embedded in our spinal cord networks (cite: knee jerk) and brainstem (cite: vestibular) are clearly important components of this stabilizing process, but are they sufficient to maintain robust movement within the complex environments we encounter, and elegantly negotiate, on a daily basis? Some insight into the requirements for designing a robust control system can be gained from engineers’ trying to build robots that can navigate in natural environments.

In the field of robotics, feats of precision and fine movement control (the most commonly prescribed role for motor cortex), are not a major source of difficulty. Industrial robots have long since exceeded human performance in both accuracy and execution speed. More recently, using reinforcement learning methods, they are now able to learn efficient movement strategies, given a human-defined goal and many repeated trials for fine-tuning. What then are the hard problems in robotic motor control? Most robots are still confined to factories, i.e. controlled, predictable environments. However, as soon as a mobile robot needs to negotiate natural terrain, a vast number of previously unknown situations arise. The non-linearities inherent to these unexpected states and resulting “perturbations” are dealt with poorly by the statistical machine learning models that are currently used to train robots in controlled settings.

Let’s consider a familiar example: You are up early on a Sunday morning and head outside to collect the newspaper. It’s cold out, so you put on a robe and some slippers, open the front door, and descend the steps leading down to the street in front of your house. Unbeknownst to you, a thin sheet of ice has formed overnight and your foot is now quickly sliding out from underneath you. You are about to fall. What do you do? Well, this depends. Is there a railing you can grab to catch yourself? Were you carrying a cup of coffee? Did you notice the frost on the lawn and step more cautiously, anticipating a slippery surface? Avoiding a dangerous fall, or recovering gracefully, requires a rich knowledge of the world, knowledge that is not immediately available to spinal or even brainstem circuits. This rich context relevant for movement is available to cortex, and cortex alone.

Imagine you were tasked with building a robot to collect your morning newspaper. This robot, in order to avoid a catastrophic and costly failure would need to have all of this contextual knowledge as well. It would ned to know about the structure of the local environment (hand railings that can support its weight), hot liquids and their viscosities, and even the correlation of frozen dew with icy surfaces. To be a truly robust movement machine, a robot must understand the world. This remains the hard problem for artificial intelligence, not playing chess or go [cite], and this may be a fitting role for our highest movement controller, motor cortex.

Given that clever posture corrections are not the sort of feats for which we praise our athletes and sports champions, it is not surprising that the difficulty of this problem has been under appreciated. It also would not be the first time that we find ourselves humbled by a problem that we initially considered trivial. Vision, for example, remains an impressively hard task for a machine to solve at human-level performance, yet it was originally proposed as an undergraduate summer project (cite: Minsky).

A primordial role for motor cortex:

We are seeking a role for motor cortex in lower (non-primate) mammals, animals that do not appear to require this structure for overt movement production. The struggles of roboticists highlight the challenge of building movement systems that robustly adapt to unexpected perturbations and the results we report in this study suggest that this is, indeed, the most conspicuous deficit of rats lacking motor cortex. So let us propose that, in rodents, motor cortex is primarily responsible for extending the robustness of the sub-cortical movement systems. It is not required for control in stable, predictable, non-perturbing environments, but instead specifically exerts its influence when unexpected challenges arise. This, we propose, was the original selective pressure for evolving a motor cortex, and thus its primordial role. This role persists in lower mammals and has been elaborated upon in primates. Our proposal of a “robust” teleology for motor cortex has a number of interesting implications.


Implications for lower mammals:

One of the most impressive traits of mammals is the vast range of environmental niches that they occupy [cite]. This flexibility requires the ability to quickly evaluate and respond to unexpected situations, many of which have never been previously encountered. Nonetheless, mammals are able to thrive in such environments. This success requires more than precision, it requires robustness, the ability to quickly come up with a passable motor solution for any situation, in any condition \cite{Bernstein1996}. Bernstein referred to this ability with his unconventional definition of “dexterity”, distinct from a simple harmony in precise hand movements. Such dexterity is required when there is \enquote{a conglomerate of unexpected, unique complications in the external situations, in a quick succession of motor tasks that are all unlike each other} \cite{Bernstein1996}.

If “robust dexterity” is the primary role for motor cortex, then it is clear why the effects of lesions have thus far been so hard to characterize: assays of motor behaviour typically rely on assessment in situations that are repeated over many trials, in the same environment. Such repeated tasks were necessary, as they offer improved statistics for quantification and comparison. However, these conditions may specifically exclude the scenarios for which motor cortex originally evolved. However, it is not easy to repeatedly produce situations that animals have not previously encountered, and even if it was, the challenges in analysing these unique situations are considerable.

The assay reported here represents our first attempt at such an experiment, and it has already revealed that such conditions may indeed by necessary to isolate the role of motor cortex. We thus propose that neuroscience should pursue similar assays, emphasizing perturbations and novel challenges, and we have worked on developing new hardware and software tools to make their design much easier (cite: Bonsai).



Implications for primates:

In contrast to lower mammals, primates rely on motor cortex for the direct control of movement. However, do they also retain its role in generating robust responses? The general paresis, or even paralysis, that results from motor cortical lesions in primates will trivially obscure any of its involvement in directing rapid responses to perturbations. Yet there is still evidence that a role in robust control is also present in primates, including humans. For example, stroke patients with partial lesions to the distributed motor cortical system will often recover the ability to move the affected bid parts. However, even after recovering movement, stroke patients are still prone to devastating errors in robust control. In fact, unsupported falls are one of the leading causes of injury and death in patients surviving motor cortical stroke \cite{Jacobs2014}. We thus suggest that stroke therapy, currently focused on regaining direct movement control, should also consider strategies for improving robustness.

Even if we acknowledge that a primordial role of motor cortex is still apparent in primate movement control, it remains to be explained why the motor cortex of these species has acquired control of basic movements. This remains an open question, but the consequences are intriguing.


What happens when cortex acquires direct control of movement? First, it must learn how to use this influence, which may explain the extended developmental times required for primates to produce basic motor control. Humans babies require years of practice to produce simple locomotion and grasping [cite], motor behaviours that are available to lower mammals almost immediately after birth. This may be the cost of giving cortex control, it takes time to figure out how to move the body, so what is the benefit? 

A robust control system (i.e. cortex), in full command of the body, is capable of negotiating more difficult environmental niches. Primates may have been able to ascend trees, and avoid their less “dexterous” predators, with a feat of motor cortical control. However, the implications of this cortical “take-over” are potentially even more profound. 

If cortex has direct control of overt movements, then observing the behaviour of a primate constitutes direct observation of commands arising in cortex(vs. sub-cortical pattern generators). In other words, when you watch a primate move you are directly observing cortical commands. Now, if a species reliant on direct cortical control happened to live in social groups, then members of this group would have a uniquely efficient means of observing the state of cortex in a conspecific; what was going on in cortex would be directly observable in the movements of that individual. The implications of establishing an efficient means of “communicating” the state of one cortex to another are simply exhilarating to imagine…



Some Conclusions:

Clearly our results are insufficient to draft a final conclusion, but that is not our goal. We present our experiments to support and motivate our effort to distil a long history of research, and ultimately suggest a new approach to investigating the role of motor cortex. This approach most directly applies to studies of lower mammals, for which we now have a host of techniques to monitor and manipulate cortical activity during behaviour. However, we propose that we should be monitoring and manipulating activity during behaviours that actually require motor cortex. 

This synthesis also has implications for human studies. We suggest that acknowledging a primary role for motor cortex in robust control, a problem still daunting to robotics engineers, can guide the development of new approaches to building intelligent machines, as well as new strategies to assess and treat patients with motor cortical damage. We concede that our results are still naïf, but propose that the implications our worthy of your consideration.

