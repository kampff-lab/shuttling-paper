
\begin{figure}
\centering
\includestandalone{figures/assay}
\caption{\textbf{An obstacle course for rodents.} A) Schematic of the different conditions in the shuttling protocol. B) Schematic of the step locking mechanism.}
\label{fig:assay}
\end{figure}

\begin{figure}
\centering
\includestandalone{figures/histology}
\caption{\textbf{Histological analysis of lesion size.} A) Representative example of bilateral ibotenic acid lesion of primary and secondary motor cortex. B) Distribution of lesion volumes in the left and right hemispheres for individual animals. A lesion was considered ``large'' if the volume was above \SI{5}{\milli\meter\cubed} on both hemispheres. C) Super-imposed reconstruction stacks for all the large lesions. D) Super-imposed reconstruction stacks for all the small lesions.}
\label{fig:histology}
\end{figure}

\begin{figure}
\centering
\includestandalone{figures/learning}
\caption{\textbf{Both lesions and controls achieve similar performance in the different protocol stages.} A) Average time required to collect the next reward across sessions. B) Average time required to cross the obstacles across sessions. C) Average time required to cross the obstacles for the first five trials of the first ``habituation'' session.}
\label{fig:learning}
\end{figure}

\begin{figure}
\centering
\includestandalone{figures/posture}
\caption{\textbf{Rats adapt their postural approach to the obstacles after a change in physics.} A) Schematic of postural analysis image processing. A ``step'' is detected when a paw of the animal activates the blue ROI. At that moment, we extract the position of the animal's nose (marked in red; see methods). B) The average position of the nose across sessions where the environment was stable and C) after the environment changed to unstable. The asterisk indicates the average in the first 20 trials before the environment changed from stable to unstable. D) Distribution of nose positions for all control animals in the last two days of the stable and unstable protocol stages. Respectively for E) large lesions and F) small lesions.}
\label{fig:posture}
\end{figure}

\begin{figure}
\centering
\includestandalone{figures/jumping}
\caption{\textbf{Animals adopted one of two strategies for dealing with the manipulated obstacles.} A) Example average projection of posture image stacks for stable (green) and unstable (red) sessions for two non-jumper and two jumper animals. B) Average nose trajectories for individual animals in the unstable condition. C) Correlation of the probability of skipping the center two steps with the weight of the animal. The color of each dot indicates whether the animal was a control or a lesion.}
\label{fig:jumping}
\end{figure}

\begin{figure}
\centering
\includestandalone{figures/random}
\caption{\textbf{Animals adjust their posture on a trial-by-trial basis to the expected state of the world.} A) Distribution of nose positions for all animals in stable (blue) and unstable (red) trials of the randomized protocol (see methods). B) Distribution of nose positions for trials in which the last trial was stable (blue) or unstable (red). C) Respectively for trials where the last two trials were stable (blue) or unstable (red). D-F) Same data as in C) split by the control and lesion groups.}
\label{fig:random}
\end{figure}

\begin{figure}
\centering
\includestandalone{figures/speed}
\caption{\textbf{Uncertainty in the expected state of the world causes the animals to rapidly adjust their movement trajectory.} A) Example average speed profile for stable (blue) and unstable (red) trials in the randomized sessions of a control animal (see text). B) Respectively for one of the largest lesions. C) Summary of the average difference between the speed profiles for stable and unstable trials across the two groups of animals.}
\label{fig:speed}
\end{figure}

\begin{figure}
\centering
\includestandalone{figures/ethogram}
\caption{\textbf{First response to an unexpected change in the environment.} A) Response types observed across individuals upon first encountering an unpredicted shift in the state of the centre obstacles. B) Ethogram of the behavioral responses classified according to the three criteria described in A) and aligned (0.0) on first contact with the newly manipulated obstacle. Black dashes indicate when the animal flicks its ears. White indicates the animal has left the obstacle course.}
\label{fig:ethogram}
\end{figure}
