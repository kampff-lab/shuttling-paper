
\begin{figure}
\centering
\includestandalone{figures/assay}
\caption{An obstacle course for rodents. (\textbf{A}) Schematic of the apparatus and summary of the different conditions in the behaviour protocol. Animals shuttle back and forth between two reward ports at either end of the enclosure. (\textbf{B}) Schematic of the locking mechanism that allows each individual step to be made stable or unstable on a trial-by-trial basis. (\textbf{C}) Example video frame from the behaviour tracking system. Coloured overlays represent regions of interest and feature traces extracted automatically from the video.}
\label{fig:assay}
\end{figure}

\begin{figure}
\centering
\includestandalone{figures/histology}
\caption{Histological analysis of lesion size. (\textbf{A}) Representative example of bilateral ibotenic acid lesion of primary and secondary forelimb motor cortex. (\textbf{B}) Distribution of lesion volumes in the left and right hemispheres for individual animals. A lesion was considered ``large'' if the total lesion volume was above \SI{15}{\milli\meter\cubed}. (\textbf{C}) Super-imposed reconstruction stacks for all the small lesions. (\textbf{D}) Super-imposed reconstruction stacks for all the large lesions.}
\label{fig:histology}
\end{figure}

\begin{figure}
\centering
\includestandalone{figures/learning}
\caption{Overall performance on the obstacle course is similar for both lesion and control animals. (\textbf{A}) Average time to cross the obstacles on the different protocol stages for forelimb motor cortex lesions and control animals. Each set of three coloured bars represents the average performance on a single session. Asterisks indicate sessions where there was a change in assay conditions during the session (see text). In these transition sessions, the average performance on the 20 trials immediately preceding the change is shown to the left of the solid vertical line whereas the performance on the remainder of that session (after the change) is shown to the right. 


(New Figure!)
(\textbf{B}) Average time required to cross the obstacles in the stable condition for extended lesions. Performance of the other groups is shown for comparison. (\textbf{C}) Average number of slips per crossing in early versus late sessions of the stable condition. Animals adopting a jumping strategy (both lesions and controls; see text) were not included in this analysis as jumping increased the probability of incorrect paw placement independently of proficiency.}
\label{fig:learning}
\end{figure}

\begin{figure}
\centering
\includestandalone{figures/posture}
\caption{Rats adapt their postural approach to the obstacles after a change in physics. (\textbf{A}) Schematic of postural analysis image processing. The position of the animal's nose is extracted whenever the paw activates the ROI of the first manipulated step (see methods). (\textbf{B}) The horizontal position, i.e. progression, of the nose in single trials for one of the control animals stepping across the different conditions of the shuttling protocol. (\textbf{C}) Average horizontal position of the nose across the different protocol stages for both lesion and control animals. Asterisks indicate the average nose position on the 20 trials immediately preceding a change in protocol conditions (see text). (\textbf{D}) Distribution of nose positions for control animals in the last two days of the stable (blue) and unstable (orange) protocol stages, respectively for (\textbf{E}) small lesions and (\textbf{F}) large lesions.}
\label{fig:posture}
\end{figure}

\begin{figure}
\centering
\includestandalone{figures/jumping}
\caption{Animals use different strategies for dealing with the unstable obstacles. (\textbf{A}) Example average projection of all posture images for stable (green) and unstable (red) sessions for two non-jumper and two jumper animals. (\textbf{B}) Average nose trajectories for individual animals crossing the unstable condition. The shaded area around each line represents the 95\% confidence interval. (\textbf{C}) Correlation of the probability of skipping the center two steps with the weight of the animal. The color indicates whether the animal was a control or a lesion.}
\label{fig:jumping}
\end{figure}

\begin{figure}
\centering
\includestandalone{figures/random}
\caption{Animals adjust their posture on a trial-by-trial basis to the expected state of the world. (\textbf{A}) Distribution of nose positions for all animals in stable (blue) and unstable (orange) trials of the randomized protocol (see methods) when stepping on first unstable obstacle. (\textbf{B}) Distribution of nose positions for trials in which the previous two trials were stable (blue) or unstable (orange). (\textbf{C-D}) Same data as in (\textbf{B}) split by the control and lesion groups.}
\label{fig:random}
\end{figure}

\begin{figure}
\centering
\includestandalone{figures/speed}
\caption{Encountering different states of the randomized obstacles causes the animals to quickly adjust their movement trajectory. (\textbf{A}) Example average speed profile across the obstacles for stable (blue) and unstable (orange) trials in the randomized sessions of a control animal (see text). The shaded area around each line represents the 95\% confidence interval. (\textbf{B}) Respectively for one of the largest lesions. (\textbf{C}) Summary of the average difference between the speed profiles for stable and unstable trials across the two groups of animals. Error bars show standard error of the mean.} (add p value)
\label{fig:speed}
\end{figure}

\begin{figure}
\centering
\includestandalone{figures/ethogram}
\caption{Responses to an unexpected change in the environment. (\textbf{A}) Response types observed across individuals upon first encountering an unpredicted instability in the state of the centre obstacles. (\textbf{B}) Ethogram of behavioural responses classified according to the three criteria described in (\textbf{A}) and aligned (0.0) on first contact with the newly manipulated obstacle. Black dashes indicate when the animal exhibits a pronounced ear flick. White indicates that the animal has crossed the obstacle course.}
\label{fig:ethogram}
\end{figure}
