% Motor cortex paper draft (mc-paper.tex)
% Copyright (c) Kampff Lab 2016
% See README.md for redistribution license

\documentclass[12pt]{article}
\usepackage[T1]{fontenc}
\usepackage[utf8]{inputenc}
\usepackage{hyphenat}
\usepackage{siunitx}

%%% Preprint settings
\usepackage{setspace}
\usepackage{lineno}
\doublespacing

% For author lists
\usepackage{authblk}
\renewcommand\Authsep{ \qquad }
\renewcommand\Authand{ \qquad }
\renewcommand\Authands{ \qquad }

%%% For figures
\usepackage{tikz}
\usepackage[mode=buildnew]{standalone}

%%% For video references
\newcounter{video}
\newcommand{\videolabel}[1]{%
  \refstepcounter{video}\label{#1}%
}

\renewcommand\Affilfont{\small}
\title{A robust role for motor cortex}
\author[1,2,*]{Gonçalo Lopes}
\author[1,2]{Joana Nogueira}
\author[1]{Joseph J. Paton}
\author[1,2]{Adam R. Kampff}
\affil[1]{Champalimaud Neuroscience Programme, Champalimaud Centre for the Unknown, Lisbon, PT}
\affil[2]{Sainsbury Wellcome Centre, University College London, London, UK}
\affil[*]{Correspondence: Gonçalo Lopes, Champalimaud Neuroscience Programme, Champalimaud Centre for the Unknown, Av. de Brasília s/n, Doca de Pedrouços, 1400-038, Lisbon, Portugal. \emph{email}: goncalo.lopes@neuro.fchampalimaud.org}

% bibliography
\usepackage{doi}
\usepackage[backend=bibtex,url=false,isbn=false,eprint=false,sorting=none,giveninits=true]{biblatex}
\makeatletter
\def\blx@maxline{77}
\makeatother
\addbibresource{references.bib}

\begin{document}
\maketitle

\begin{linenumbers}
\begin{abstract}
The role of motor cortex in the direct control of movement remains unclear, particularly in non-primate mammals. More than a century of research using stimulation, anatomical and electrophysiological studies has implicated neural activity in this region with all kinds of movement. However, following the removal of motor cortex, or even the entire cortex, rats retain the ability to execute a surprisingly large range of adaptive behaviours, including previously learned skilled movements. In this work we revisit these two conflicting views of motor cortical control by asking what the primordial role of motor cortex is in non-primate mammals, and how it can be effectively assayed. In order to motivate the discussion we present a new assay of behaviour in the rat, challenging animals to produce robust responses to unexpected and unpredictable situations while navigating a dynamic obstacle course. Surprisingly, we found that rats with motor cortical lesions show clear impairments in dealing with an unexpected collapse of the obstacles, while showing virtually no impairment with repeated trials in many other motor and cognitive metrics of performance. We propose a new role for motor cortex: extending the robustness of sub-cortical movement systems, specifically to unexpected situations demanding rapid motor responses adapted to environmental context. The implications of this idea for current and future research are discussed.
\end{abstract}


\videolabel{vid:learning}
\videolabel{vid:learning-matrix}
\videolabel{vid:decorticate-habituation}

\videolabel{vid:conditions}
\videolabel{vid:jpak20}

\videolabel{vid:manipulation-strategies}
\videolabel{vid:manipulation-small}
\videolabel{vid:manipulation-large}

\section{Introduction}

Since its discovery 150 years ago, the role of the motor cortex has been a topic of controversy and confusion \cite{Lashley1924}. Here we report our efforts to piece together a teleology for cortical motor control. It is a story about what motor cortex may be good for, rather than how it works. Motor cortex has been implied in ``understanding'' the observed movements of others \cite{Rizzolatti2004}, imagining one's own movements, or in learning new movements, but here we will focus on its role in directly controlling movement.

Motor cortex is still broadly defined as the region of the cerebral hemispheres from which movements can be evoked by low-current surface stimulation, following Fritsch and Hitzig's original experiments in 1870 \cite{Fritsch1870}. Stimulation of different parts of the motor cortex elicit movements in different parts of the body, establishing a topographical representation covering the entire skeletal musculature across cortical surface \cite{Leyton1917,Penfield1937,Neafsey1986}. Electrophysiological recordings have found correlations of motor cortical activity with all kinds of movement parameters including muscle force, speed or target endpoint, at both single neuron and population levels \cite{Georgopoulos1986}. There is a long standing debate regarding what, exactly, does motor cortex control, muscles or movements \cite{Todorov2000}. In fact, recent stimulation studies have even suggested that the entire natural repertoire of animal behaviour is represented across the surface of motor cortex \cite{Graziano2002,Aflalo2006}. Together, these observations have led to the long standing idea that motor cortex is the part of the brain responsible for the control of voluntary movements.

However, there is a nagging dilemma in the field of cortical motor control which is equally old. In humans, the result of motor cortical stroke can be devastating, resulting in permanent loss of control of muscle movements or even paralysis; motor control is permanently and obviously impaired. In non-human primates, the primary reported effect of motor cortical lesion is a clear loss of motility in distal forelimb, especially in the control of fine, individual finger movements, which are required for the execution of precision skills \cite{Leyton1917,Darling2011}. But equally impressive is the extent to which other behaviours fully recover, including the ability to sit, stand, walk, climb and even reach accurately for food, as long as precise finger movements are not required \cite{Leyton1917,Darling2011}. In other mammals, the absence of lasting effects following motor cortical lesion is even more dramatic. Careful studies of skilled reaching in the rodent have revealed subtle impairments in grasping behaviours \cite{Alaverdashvili2008a}, comparable to the loss of precision finger movements in primates, but this loss pales in comparison to the range of movements that \emph{are} preserved. In fact, even following complete decortication, rats and cats retain a shocking amount of their original behavioural repertoire \cite{Bjursten1976,Terry1989}. If we accept the simple hypothesis that motor cortex is the structure responsible for ``voluntary movement production'', then why is there such a huge difference in the effects of motor cortical lesions between humans and other mammals? Do non-human mammals only utilize cortical motor control for precision movements involving the distal musculature? The results of the numerous stimulation and electrophysiology studies do seem to suggest the involvement of this structure with all kinds of movements, across all mammalian species, so how can these two disparate views be rectified?

A partial explanation for this dilemma has been found in the anatomy of the descending cortical motor pathways. In primates, the conspicuous effects of motor cortical lesion can be reproduced by complete section of the pyramidal tract, the direct monosynaptic projection that links motor cortex, and other cortical regions, to the spinal cord \cite{Tower1940,Lawrence1968}. This corticospinal tract is thought to support the low-current movement responses caused by electrical stimulation in the cortex, as observed by the increased difficulty in obtaining stimulation effects following section of the tract at the level of the medulla \cite{Woolsey1972}. In the monkey, and similarly in man, this fibre system has been found to directly terminate on spinal motor neurons responsible for the control of distal musculature \cite{Leyton1917,Bernhard1954}. However, in all other mammals, including cats and rats, the termination pattern of the pyramidal tract in the cord largely avoids these ventral motor neuron pools and concentrates instead on intermediate zone interneurons \cite{Kuypers1981,Yang2003}. Furthermore, the rubrospinal tract, an important descending pathway that overlaps with the corticospinal terminations in the intermediate zone, is much degenerated in humans when compared to primates and other mammals \cite{Square1982}, and is thought to play a role in compensating for the loss of the pyramidal tract \cite{Lawrence1968a}.

These differences in anatomy might explain the lack of noticeable, lasting movement deficits in non-primates, but begs the question of what motor cortex is good for in all these other mammals, including our oldest known common ancestors. In the rat, a large swath (what percentage?) of cortex is considered ``motor'' based on anatomical \cite{Donoghue1982}, stimulation \cite{Donoghue1982,Neafsey1986} and electrophysiological grounds. However, the most consistently observed long-term effect following motor cortical lesion has been an impairment in supination of the wrist which impairs reaching for food pellets through a narrow vertical slit. Are we to believe that this massive high-level motor control structure, with dense efferent projections not only to the spinal cord but to basal ganglia, cerebellum and other brainstem movement related areas, as well as to most primary sensory areas, has evolved simply to allow more precise wrist rotations? What did this cortical structure influence about movement and behaviour in our primordial relatives? If nothing, then why did it evolve? Are there other functions in the control of movement underlying motor cortical circuits and its descending projections that we may be missing with our current assays?

A different perspective on this issue has emerged from studies in the neural control of locomotion, particularly in cats, which suggest that the corticospinal tract may play a role in the adjustment of ongoing movements that are themselves generated by lower motor systems. In this view, rather than motor cortex assuming direct control over muscle movement, it might instead modulate the activity and sensory feedback in spinal circuits in order to adapt the movement to changing conditions. The idea that the descending pathways that make up the lateral system can superimpose speed and precision on an existing baseline of behaviour was also suggested by the work in primates, but has been investigated most thoroughly in the context of cat locomotion.

It has been known for more than a century that completely decerebrate cats are still capable of sustaining the basic locomotor rhythm necessary for walking on a flat treadmill utilizing only spinal circuits \cite{GrahamBrown1911}. Brainstem circuits are sufficient to initiate the activity of these central patttern generators \cite{Grillner1973}, so what are the contributions of motor cortex to the control of locomotion? Single-unit recordings of pyramidal tract neurons (PTNs) during treadmill locomotion have shown that a large proportion of these neurons are locked to the step cycle \cite{Armstrong1984a}. However, we know from the decerebrate studies that this activity is not required for the basic movement pattern, so what is its functional role?

Lesions of the lateral descending pathways have revealed a long term impairment in the ability to step over obstacles \cite{Drew2002}. Recordings of PTN neurons during locomotion show increased activity during execution of these visually guided modifications to the step cycle \cite{Drew1996}. This led to the suggestion that motor cortex neurons are necessary for precise stepping and for superimposing movement modifications on top of ongoing muscle synergies. However, long-term effects seem to require complete lesion of \emph{both} the corticospinal and rubrospinal tracts \cite{Drew2002}. Even in these animals, the voluntary act of stepping over an obstacle does not disappear entirely and in fact it can even be adapted to changes in the height of the obstacles \cite{Drew2002}. Specifically, even though animals never regain the ability to gracefully clear an obstacle without touching it, they seem to adjust their stepping height to a higher obstacle in such a way that would allow them to clear the lower obstacles. On the other hand, effects of lesions restricted to the pyramidal tract seem to disappear over time \cite{Liddell1944}, and are more clearly visible only the first time an animal encounters the obstacles \cite{Liddell1944}.

What can we take away from all these studies? It is becoming clear that the involvement of motor cortex in the control of all ``voluntary movement'' is human-specific. There is a clear role of motor cortex across mammals in the control of precise movements of the extremities, especially those requiring breaking apart lower movement patterns and synergies such as individual movements of the fingers, but these effects are subtle in non-primate mammals. Is this really the only reason why we have a motor cortex? In other mammals many other precise movements do not depend on motor cortex for their execution. Thus, generalizing this specific role of motor cortex from humans to all other mammals can be misleading. We may be missing some other more primordial roles for this structure that predominate in other mammals, and by doing so, we may also be missing something important for humans.

The suggestion that motor cortex produces modifications to ongoing movement patterns, as suggested by the electrophysiological studies of cat locomotion, definitely points to a satisfying role that could potentially rectify the results of lesions. However, it seems that even these modifications can generally recover from a motor cortical lesion. What exactly does cortex add to movement? When would such cortical control be required?

Cortex has long been thought to be the fundamental structure for integrating a representation about the contextual state of the world and for understanding how it works from experience. If motor cortex is the means by which these representations can gain influence over the body, can we find situations where they would be required for adaptive behaviour? Cortical control over behaviour has been actively investigated in experimental psychology by the foundational work of Karl Lashley and many of his students. In the rat, large cortical lesions were found to produce only slight impairments in movement control, and even deficits in learning and decision making abilities were hard to demonstrate consistently over repeated trials. There is some evidence that cortical control may be involved in postural adaptations to unexpected situations \cite{Lashley1921a}, but we may need to expand the scope of laboratory tasks to address a broader range of ``natural'' problems that brains would normally encounter in order to clarify the cortical contributions to movement in non-primate mammals. Specifically, we may just be missing some obvious roles because our assessment is focused on tasks in which humans are impaired following motor cortical lesion.

In the following, we report an experiment that was designed precisely to simultaneously assay the role of motor cortex in the direct control of movement as well as provide opportunities to expose the animals to more natural and challenging situations. The results of this experiment have led us to reformulate a new hypothesis for motor cortical control that we outline and address in the discussion.

\section{Experiment Introduction}

In the natural world it is important to be able to adapt our locomotion to any given surface, not only in anticipation of the movement conditions that it will afford, but also in response to both expected and unexpected perturbations that may happen in the course of movement. This allows animals to move robustly even in the face of changing environmental conditions. Testing this kind of robust control during locomotion can be difficult as it requires introducing a mechanical perturbation without queueing the animal about the altered state of the world. Marple-Horvat and colleagues built a circular ladder specifically designed to record from motor cortex during exactly such conditions \cite{Marple-Horvat1993}. One of the modifications they introduced was to make one of the rungs of the ladder fall unexpectedly under the weight of the animal. When they now looked at the activity of motor cortical neurons during the rung drop, they noticed an increased pattern of activity, above the recorded baseline from normal stepping, as the animal recovered from the fall and resumed walking. However, whether this increased activity of motor cortex was necessary for the recovery response has never been assayed.

To test whether the intact motor cortex is required for robust control, we designed a reconfigurable dynamic obstacle course where individual steps can be made stable or unstable on a trial-by-trial basis. In this assay, rats are motivated to go back and forth across the obstacles, in the dark, in order to collect water rewards. We specifically designed the assay such that modifications to the physics of the obstacles could be made covertly. In this way, the animal has no explicit information about the state of the steps until he actually walks over them. Our goal was to characterize precisely the conditions under which motor cortex becomes necessary for the control of movement, which motivated us to introduce an environment with graded levels of uncertainty.

In this experiment, we assessed the role of motor cortical structures by making targeted lesions to areas responsible for forepaw control. Assaying the relevance of a brain structure has long involved surgical lesions of that structure and subsequent detailed analysis of any behavioural deficits. This method is not without its difficulties. The problems of plasticity and diaschisis seem to forever plague our conclusions based on injury and manipulations of nervous tissue. Many recent methods for reversible chemical or optogenetic inactivation of the cortex have been proposed to improve statistical power of behavioural assessments. Unfortunately, given that the cortex maintains a tight balance of excitation and inhibition during its functioning, the effects of such transient manipulations are prone to cause multiple downstream effects that can confound inferences about behavioural relevance \cite{Otchy2015}. In this respect, they are similar to stimulation experiments in that they are very useful in determining that two areas are connected in a circuit, but not necessarily what the connection means. Of course, permanent lesions themselves can induce plasticity changes in the function of downstream and upstream circuits but such changes should represent a homeostatically stable state of the system. In this way, we hope to simultaneously probe the limits of recovery, as well as the kinds of problems for which a fully intact structure is definitely required.

We first describe the assay and our detailed methods, and proceed to describe the behaviour of animals faced with different degrees of uncertainty in this motor control problem. We then describe and discuss the differences in the behaviour of animals with and without motor cortex under different situations and conclude with a discussion of the implications of the results in the context of our understanding of motor cortex, ultimately proposing a new (more robust) role for motor cortex in non-primate mammals.

\section{Results}

Histological examination of the lesions revealed significant variability in the extent of damaged areas, which were in some cases traced back to spurious mechanical blockage of the injection pipette. Nevertheless, volume reconstruction of the serial sections allowed us to accurately quantify the size of each lesion and use these values to compare observed behavioural effects as a function of lesion size.

During the first training sessions in the ``stable'' environment, all animals, both lesions and controls, quickly learned to shuttle across the obstacles and decreased both the time between collected rewards, as well as time taken to cross the obstacles, achieving stable, skilled performance after a few days of training. After only a few encounters with the obstacles, animals quickly adapted their stride length in order to facilitate crossing. Even though the distance between steps was fixed for all animals, the time taken to adapt the crossing strategy was similar irrespective of body size.

When first encountering the obstacles, animals adopted a cautious gait, investigating the location of the subsequent obstacle with their whiskers, stepping with the leading forepaw followed by a step to the same position with the trailing paw. However, over the course of a couple of trials, all animals exhibited a new strategy of ``stepping over'' the planted forepaw to the next obstacle, suggesting an increased confidence in their movement strategy in this new environment.

This more confident gait developed into a coordinated locomotion sequence after a few additional training sessions. The development of the ability to move confidently and quickly over the obstacle course was observed in both lesions and control animals. Surprisingly, we did not observe noticeable deficits in paw placement in the lesioned animals compared to controls as previously reported following pyramidal tract section. (in what other animals, in addition to Whishaw’s studies?)

In order to evaluate whether the ability to navigate this stable environment depended on motor cortical representations of trunk and hindlimbs, we included five additional animals, three with larger frontal thermocoagulation lesions and two with much more extensive cortical aspiration lesions. In these animals, there were immediately observable effects on paw placement upon the first encounter with the obstacles. However, over the course of ten repeated trials, this impairment recovered dramatically, up to the point where animals were moving as efficiently, and in fact in many cases even more efficiently, than control animals. These observations seem to indicate the presence of motor learning and not simply an increase in confidence with the new environment. The initial movement defects seem to be consistent with results from sectioning the entire pyramidal tract in cats, and as observed in these studies, quickly disappeared over the course of a few trials (how many?)

In subsequent training sessions we progressively increased the difficulty of the obstacle course, by progressively making more steps in the course unstable. The goal was to compare the performance of the two groups as a function of difficulty. Surprisingly, both lesion and control animals were able to improve their performance by the end of each training stage even for the most extreme condition where all steps were made unstable. This seems to indicate that the ability of these animals to fine-tune their motor performance in a challenging environment remained intact.

One noticeable exception was the animal with the largest ibotenic acid lesion. This animal, following exposure to the first unstable protocol, was unable to bring himself to cross the obstacle course. Some control animals experienced a similar form of distress following exposure to the first manipulation, but eventually all managed to start crossing the obstacles again over the course of a single day. In order to test whether this was due to some kind of motor disability, we lowered the difficulty of the protocol for this one animal until he was able to cross again. Following the random permutation protocol, this animal was able to get itself to cross a single released obstacle placed in any location of the assay. After this success, he eventually learned to cross the highest difficulty level in the assay in about the same time as all the other animals, which convinced us that there was indeed no lasting motor execution or learning deficit, and that the disability must have been due to some other unknown cognitive factor. 

Having established that the overall motor performance of these animals was the similar across all conditions, we next asked whether there was any difference in the strategy used by the two groups of animals to cross the obstacles. We noticed that during the first week of training, the posture of the animals when stepping on the obstacles changed significantly over time. Specifically, the centre of gravity of the body was shifted forward and higher in latter days, in a manner proportional to performance. However, after the first manipulation, we observed an immediate and persistent adjustment of this posture, with animals assuming a lowered centre of gravity as they resumed shuttling across the obstacle course. Interestingly, we noticed that a group of animals adopted a different strategy. Instead of lowering their centre of gravity, they either kept it unchanged or shifted it even more forward and performed a jump over the affected obstacles. These two strategies were remarkably consistent across the two groups of animals but there was no correlation between the strategy used and the degree of motor cortical lesion. In fact, we found the best predictor of whether an animal was a ``jumper'' to be the body weight of the animal.

During the two days where the stable state of the environment was reinstated before the randomized protocol, the posture of the animals was gradually restored to pre-manipulation levels, although at a much slower rate than the transition from stable to unstable. Again, this postural adaptation was independent of the presence or absence of forepaw motor cortex.

We next turned to look in detail at the days where the state of the obstacle course was randomized on a trial-by-trial basis. This stage of the protocol is particularly interesting as it reflects a situation where the environment has a persistent degree of uncertainty. For this analysis, we were forced to exclude the animals that employed a jumping strategy, as their approach to the obstacles was the same irrespective of the state of the world. First, we repeated the same posture analysis comparing all the stable and unstable trials in the random protocol in order to control for whether there was any cue in our motorized setup that the animals might be using to gain information about the state of the world. There was no significant effect between stable and unstable trials on the posture of the animal. However, by classifying the trials on the basis of past trial history, a significant effect on posture was obtained, which became more pronounced the more consistent the history was. This suggested to us that the animals might be adjusting their body posture when stepping on the affected obstacles on the basis of their current expectation about the state of the world. Surprisingly, this effect again did not depend on the presence or absence of frontal motor cortical structures.

Finally, we decided to test whether motor strategies were differentially affected by the state of the obstacles. If the animals do not know beforehand what state the world will be in, this means that there might be an increased challenge to the stability of the motor plant when they cross over the affected steps that might demand a quick change in strategy. In order to evaluate the dynamics of crossing, we measured the speed at which the animals were moving at each point in the assay, aligned on the first manipulated step. Consistent with the past history effect discussed above, there was also a consistent difference in the speed with which the animals approached the obstacles, depending on the state of the world in the previous trial. In order to normalize for this absolute change in speed we used the average speed with which the animal approached the step as a baseline and subtracted it from the overall profile for each trial. In order to summarize the difference between stable and unstable trials for each animal, we subtracted the speed in unstable trials from the speed in stable trials at each position in the assay and computed the sum of this measure for all trials. Interestingly, two of the largest lesions appeared to be significantly slowed down on unstable trials, while controls or the smallest lesions instead tended to speed up in response to a perturbation. However, the overall effect is not statistically significant and both animals clearly display the same comparable overall performance in both situations.

Nevertheless, we were intrigued by this trend and decided to investigate in detail the single moment in the assay where the effect of a perturbation should be the largest and which was experienced by all animals. This is the very first time when the obstacles become unstable. In contrast with the random protocol where the uncertainty in the state of the world is expected, in this case the collapse of the obstacles is completely unexpected and could trigger an entirely different response from the animals.

A detailed analysis of the responses to the first collapse of the steps revealed an impressive difference in the strategies deployed by the two groups of animals. Upon the first encounter with the manipulated steps, we observed three types of behavioural responses from the animals: investigation, in which the animals immediately stop their progression and orient towards, whisk and physically manipulate the altered obstacle; compensation, in which the animals rapidly adjust their behaviour to negotiate the unexpected obstacle; and halting, in which the ongoing motor program ceases and the animals' behaviour simply comes to a sudden stop for several seconds. Remarkably, these responses depended dramatically on the presence or absence of motor cortex. All animals with significant motor cortical lesions, upon their first encounter with the novel environmental obstacle, halted for several seconds, whereas animals with an intact motor cortex, or with vestigial lesions, were able to rapidly react with either an investigatory or compensatory response.

\section{Experiment Discussion}

In this experiment, we assessed the role of motor cortical structures by making targeted lesions to areas responsible for forelimb control \cite{Kawai2015,Otchy2015}. Consistent with previous studies, we did not observe any conspicuous deficits in movement execution for rats with bilateral motor cortex lesions when negotiating a stable environment. Even when exposed to a sequence of unstable obstacles, animals were able to learn an efficient strategy for crossing these more challenging environments, with or without motor cortex. These movement strategies also include a preparatory component that might reflect the state of the world an animal expected to encounter. Surprisingly, these preparatory responses also did not require the presence of motor cortex.

It was only when the environment did not conform to expectation, and demanded a rapid adjustment, that a difference between the lesion and control groups was obvious. Animals with extensive damage to the motor cortex did not deploy a change in strategy. Rather, they halted their progression for several seconds, unable to robustly respond to the new motor challenge. In an ecological setting, such hesitation could easily prove fatal.

\section{Discussion}

Is ``robust control'' a problem worthy of high level cortical input? Recovering from a perturbation, to maintain balance or minimize the impact of a fall, is a role normally assigned to our lower level postural control systems. The corrective responses embedded in our spinal cord \cite{Sherrington1893b,Sherrington1910}, brainstem \cite{Arshian2014} and midbrain \cite{Grillner1973} are clearly important components of this stabilizing network, but are they sufficient to maintain robust movement in the dynamic environments that we encounter on a daily basis? Some insight into the requirements for a robust control system can be gained from engineering attempts to build robots that navigate in natural environments.

In the field of robotics, feats of precision and fine movement control (the most commonly prescribed role for motor cortex), are not a major source of difficulty. Industrial robots have long since exceeded human performance in both accuracy and execution speed \cite{Senoo2009}. More recently, using reinforcement learning methods, they are now able to automatically learn efficient movement strategies, given a human-defined goal and many repeated trials for fine-tuning \cite{Coates2008}. What then are the hard problems in robotic motor control? Why are most robots still confined to factories, i.e. controlled, predictable environments? The reason is that as soon as a robot encounters natural terrain, a vast number of previously unknown situations arise. The resulting ``perturbations'' are dealt with poorly by the statistical machine learning models that are currently used to train robots in controlled settings.

Let’s consider a familiar example: You are up early on a Sunday morning and head outside to collect the newspaper. It is cold out, so you put on a robe and some slippers, open the front door, and descend the steps leading down to the street in front of your house. Unbeknownst to you, a thin layer of ice has formed overnight and your foot is now quickly sliding out from underneath you. You are about to fall. What do you do? Well, this depends. Is there a railing you can grab to catch yourself? Were you carrying a cup of coffee? Did you notice the frost on the lawn and step cautiously, anticipating a slippery surface? Avoiding a dangerous fall, or recovering gracefully, requires a rich knowledge of the world, knowledge that is not immediately available to spinal or even brainstem circuits. This rich context relevant for robust movement is readily available in cortex, and cortex alone.

Imagine now that you were tasked with building a robot to collect your morning newspaper. This robot, in order to avoid a catastrophic and costly failure, would need to have all of this contextual knowledge as well. It would need to know about the structure of the local environment (hand railings that can support its weight), hot liquids and their viscosities, and even the correlation of frozen dew with icy surfaces. To be a truly robust movement machine, a robot must \emph{understand} the physical structure of the world.

Reaching to stop a fall while holding a cup of coffee is not exactly the kind of feat for which we praise our athletes and sports champions, and this might explain why the difficulty of such ``feats of robustness'' are often overlooked. However, it would not be the first time that we find ourselves humbled by the daunting complexity of a problem that we naively assumed was ``trivial''. Vision, for example, has remained an impressively hard task for a machine to solve at human-level performance, yet it was originally proposed as an undergraduate summer project \cite{Papert1966}. Perhaps a similar misestimate has clouded our designation of the hard motor control problems worthy of cortical input.

Inspired by the challenges confronting roboticists, as well as our rodent behavioural results, we are now in a position to posit a new role for motor cortex.

\subsubsection*{A primordial role for motor cortex}

We are seeking a role for motor cortex in non-primate mammals, animals that do not require this structure for overt movement production. The struggles of roboticists highlight the difficulty of building movement systems that robustly adapt to unexpected perturbations and the results we report in this study suggest that this is, indeed, the most conspicuous deficit for rats lacking motor cortex. So let us propose that, in rodents, motor cortex is primarily responsible for extending the robustness of the subcortical movement systems. It is not required for control in stable, predictable, non-perturbing environments, but instead specifically exerts its influence when unexpected challenges arise. This, we propose, was the original selective pressure for evolving a motor cortex, and thus its primordial role. This role persists in all mammals, mediated via a modulation of the subcortical motor system as is emphasized in studies of cat locomotion, and has been elaborated upon in primates to include direct control over much of the skeletal musculature. Our proposal of a ``robust'' teleology for motor cortex has a number of interesting implications.

\subsubsection*{Implications for non-primate mammals}

One of the most impressive traits of mammals is the vast range of environmental niches that they occupy. While most other animals adapt to change over evolutionary time scales, mammals excel in their flexibility, quickly evaluating and responding to unexpected situations, taking risks even when faced with challenges that have never been previously encountered \cite{Spinka2001}. This success requires more than precision, it requires resourcefulness, the ability to quickly execute a motor solution for any situation and in any condition \cite{Bernstein1996}. The Russian neurophysiologist Bernstein referred to this ability with an unconventional definition of ``dexterity'', which he considered to be distinct from a simple harmony and precision of movements. In his words, dexterity is required only when there is \enquote{a conglomerate of unexpected, unique complications in the external situations, in a quick succession of motor tasks that are all unlike each other} \cite{Bernstein1996}.

If Bernstein’s ``robust dexterity'' is the primary role for motor cortex, then it becomes clear why the effects of lesions have thus far been so hard to characterize: assays of motor behaviour typically evaluate situations that are repeated over many trials in a stable environment. Such repeated tasks were useful, as they offer improved statistical power for quantification and comparison. However, we propose that these conditions specifically exclude the scenarios for which motor cortex originally evolved. It is not easy to repeatedly produce conditions that animals have not previously encountered, and the challenges in analysing these unique situations are considerable.

The assay reported here represents our first attempt at such an experiment, and it has already revealed that such conditions may indeed be necessary to isolate the role of motor cortex in rodents. We thus propose that neuroscience should pursue similar assays, emphasizing unexpected perturbations and novel challenges, and we have developed new hardware and software tools to make their design and implementation much easier \cite{Lopes2015a}.

\subsubsection*{Implications for primate studies}

In contrast to other mammals, primates require motor cortex for the direct control of movement. However, do they also retain its role in generating robust responses? The general paresis, or even paralysis, that results from motor cortical lesions in primates will trivially obscure its involvement in directing rapid responses to perturbations. Yet there is evidence that a role in robust control is still present in primates, including humans. For example, stroke patients with partial lesions to the distributed motor cortical system will often recover the ability to move the affected musculature. However, even after recovering movement, stroke patients are still prone to severe impairments in robust control: unsupported falls are one of the leading causes of injury and death in patients surviving motor cortical stroke \cite{Jacobs2014}. We thus suggest that stroke therapy, currently focused on regaining direct movement control, should also consider strategies for improving robust responses.

Even if we acknowledge that a primordial role of motor cortex is still apparent in primate movement control, it remains to be explained why the motor cortex of these species acquired direct control of basic movements in the first place. This is an open question, but the consequences are intriguing.

What happens when cortex acquires direct control of movement? First, it must learn how to use this influence, bypassing or modifying lower movement controllers. While functional corticospinal tract connections may be established prenatally \cite{Eyre2000}, effective development of corticospinal dependent movements that override the lower motor system can take much longer in primates and seems to follow the maturation period of corticospinal termination patterns \cite{Lawrence1976}. Humans require years of practice to produce and refine simple locomotion and grasping \cite{Thelen1985,VonHofsten1989}, motor behaviours that are available to other mammals almost immediately after birth. This may be the cost of giving cortex direct control of movement, it takes more time to figure out how to move the body, but what is the benefit? 

A robust control system (i.e. cortex), in full command of the body, is capable of negotiating ever more difficult environmental niches. Primates may have been able to ascend trees and negotiate their precious branches, and thus avoid less ``dexterous'' predators, by relying on more direct motor cortical control. However, the implications of this cortical ``take-over'' are potentially even more profound. 

If cortex has direct control of overt movements, then observing the behaviour of a primate constitutes direct observation of commands arising in cortex (vs. subcortical pattern generators). In other words, when you watch a primate move you are directly observing cortical commands. If a species reliant on direct cortical control happened to live in social groups, then members of this group would have a uniquely efficient means of observing the state of cortex in a conspecific; what was happening in cortex would be directly observable in the movements of each individual. The implications of establishing an efficient means of ``communicating'' the state of one cortex to another are simply exhilarating to imagine...

\subsubsection*{Some preliminary conclusions}

Clearly our results are insufficient to draw any final conclusion, but that is not our main goal. We present these experiments to support and motivate our attempt to distil a long history of research, and ultimately suggest a new approach to investigating the role of motor cortex. This approach most directly applies to studies of non-primate mammals. There is now a host of techniques to monitor and manipulate cortical activity during behaviour in these species, but we propose that we should be monitoring and manipulating activity during behaviours that actually require motor cortex.

This synthesis also has implications for engineers and clinicians. We suggest that acknowledging a primary role for motor cortex in robust control, a problem still daunting to robotics engineers, can guide the development of new approaches for building intelligent machines, as well as new strategies to assess and treat patients with motor cortical damage. We concede that our results are still naïve, but propose that the implications are worthy of further consideration.


\section{Methods}

\textbf{Lesions:} Ibotenic acid was injected bilaterally in 11 Long-Evans rats (ages from 83 to 141 days; 9 females, 2 males), on 3 injection sites at 2 depths (\SI{-1.5}{\milli\meter} and \SI{-0.75}{\milli\meter} from the surface of the brain). At each depth we injected a total amount of \SI{82.8}{\nano\liter} using a microinjector (Drummond Nanoject II, \SI{9.2}{\nano\liter} per injection, 9 injections per depth). The coordinates for each site, in \si{\milli\meter} with respect to Bregma, were: +1.0 AP / 2.0 ML; +1.0 AP / 4.0 ML; +3.0 AP / 2.0 ML. Five other animals were used as sham controls (age-matched controls; 3 females, 2 males), subject to the same intervention, but where ibotenic acid was replaced with physiological saline. Six additional animals were left as wildtype, no-surgery, controls (age-matched controls; 6 females).

\textbf{Recovery period:} After the surgeries, animals were given a minimum of one week (up to two weeks) recovery period in isolation. After this period, animals were handled every day for a week, after which they were paired again with their age-matched control to allow for social interaction during the remainder of the recovery period. In total, all animals were allowed at least one full month of recovery before they were first exposed to the behaviour assay.

\textbf{Histology:} Animals were perfused intracardially with 4\% paraformaldehyde in phosphate buffer saline (PBS) and brains were post-fixed for at least \SI{24}{\hour} in the same fixative. Serial coronal sections (\SI{100}{\micro\meter}) were Nissl-stained and imaged for identification of lesion boundaries.

\textbf{Behaviour:} The animals were kept in a state of water deprivation for \SI{20}{\hour} before each daily session. During the session the animal was placed inside a behaviour box for \SI{30}{\minute}, where it could collect water by shuttling back and forth between two reward ports (Island Motion Corporation, USA). To do this, animals had to cross a \SI{48}{\centi\meter} obstacle course composed of eight \SI{2}{\centi\meter} aluminum steps spaced by \SI{4}{\centi\meter} (Figure \ref{fig:assay}A). The structure of the behaviour assay and each step in the obstacle course was built out of Bosch Rexroth aluminum structural framing, \SI{20}{\milli\meter} series. For every trial, rats were delivered a \SI{20}{\micro\liter} drop of water. At the end of the day, rats were given free access to water for \SI{10}{\minute} before initiating the next deprivation period.

A motorized brake allowed us to lock or release each step in the obstacle course (Figure \ref{fig:assay}B). The shaft of each of the obstacles was coupled to an acrylic piece used to control the rotational stability of each step. In order to lock a step in a fixed position, two servo motors are actuated to press against the acrylic piece and hold it in place. Two other acrylic pieces were used as stops to ensure a maximum rotation angle of approximately +/- \ang{100}. Two small nuts were attached to the bottom of each step to work as light weights that give the obstacles a tendency to return to their original flat configuration. In order to ensure that noise from servo motor actuation cannot be used as a cue to tell the animal about the state of each step, the motors are always set to press against an acrylic piece, either the piece that keeps the step stabilized, or the acrylic stops. At the beginning of each trial, the motors go through a randomized sequence of positions in order to mask information about state transitions and also to ensure the steps are reset to their original configuration. Control of the motors was done using a Motoruino board (Artica, PT) along with a custom workflow written in the Bonsai visual programming language \cite{Lopes2015a}.

Animals were run on \SI{30}{\minute} sessions six days of the week from Monday to Saturday, with a day of free access to water on Sunday. Before the water deprivation begins, animals were run on a single habituation session where they were placed in the box for a period of \SI{15}{\minute}. Animals were run on the following sequence of conditions over the course of a month (see also Figure \ref{fig:assay}A): day 0, habituation to the box; day 1-4, all the steps were fixed in a stable configuration; day 5, 20 trials of the stable configuration, after which the two center steps are made free to rotate; day 6-10, the center two steps remain free to rotate; day 11, 20 trials of the partially unstable configuration, after which the two center steps are again fixed in a stable state; day 12, all the steps were fixed in a stable configuration; day 13-16, the state of the center two steps was randomized on a trial-by-trial basis to be either stable or free to rotate; day 17-18, 20 trials of stable configuration, followed by the randomized protocol for the center two steps; day 19-22, the state of the six center steps was randomized trial-by-trial in such a way that on each trial either two rails are picked at random to be free to rotate or with a low probability all rails are stable; day 23, after 20 trials of randomized permutations, all rails are made free to rotate; day 24, all rails are free to rotate.

\textbf{Behaviour Data:} The performance of the animals was recorded using a high-speed and high-resolution video camera (PointGrey Flea3 FL3-U3-13S2M-CS, 1280x960 @ 120 fps). Behaviour was run in the dark, under infrared illumination using super-bright LED strips. Tracking of the nose was achieved by background subtraction and connected component labeling of segmented image elements, of which we then selected the furthermost point along the major axis of the ellipse best fit to the largest segmented object.

\textbf{Video Classification:} Ethogram classification was done on a frame-by-frame analysis of the high-speed video aligned on first contact with the manipulated rail. The frame of first contact was defined as the first frame in which there is noticeable movement of the rail caused by animal contact. Three main categories of behaviour were observed to follow the first contact: compensation, investigation and halting. Behaviour sequences were first classified as belonging to one of these categories and their onsets and offsets determined by the following criteria. Compensation behaviour is defined by rapid, whole-body (?) movement that results in an adaptive postural correction to the perturbation. Onset of this behaviour is defined by the first frame in which there is visible rapid contraction of the body musculature following first contact. Investigation behaviour consists of periods of targeted interaction with the rails, often involving manipulation of the freely moving obstacle with the forepaw. The onset of this behaviour is defined by the animal orienting its head down to one of the manipulated rails, followed by subsequent interaction. Halting behaviour is characterized by a period in which the animals stop their ongoing motor program, and maintain the same body posture for several seconds, with occasional movements of the head, but without orienting specifically to the manipulated rails. They tend to look up and to the sides as if in a state of confusion (too vague?), or uncertainty about what to do next. Onset of this behaviour is defined by the moment where locomotion and other motor activities besides movement of the head come to a stop. A human classifier blind to the lesion condition was given descriptions of each of these three main categories of behaviour and asked to note onsets and offsets of each behaviour throughout the video.

\textbf{Data Analysis:} Analysis was run using the NumPy scientific computing package for the Python programming language and custom software. We need more details of specific analysis techniques. Maybe.

\section{Acknowledgements}

We thank João Frazão for invaluable help annotating the behaviour videos. G.L. is supported by the PhD Studentship SFRH/BD/51714/2011 from the Foundation for Science and Technology. The Champalimaud Neuroscience Programme is supported by the Champalimaud Foundation.

\begin{figure}
\centering
\includestandalone{figures/assay}
\caption{An obstacle course for rodents. (\textbf{A}) Schematic of the apparatus and of the different conditions in the behaviour protocol. Animals shuttle back and forth between two reward ports. (\textbf{B}) Schematic of the locking mechanism allowing each individual step to be made stable or unstable on a trial-by-trial basis. (\textbf{C}) Example video frame from the behaviour tracking system. Coloured overlays represent regions of interest and feature traces extracted automatically from the video. }
\label{fig:assay}
\end{figure}

\begin{figure}
\centering
\includestandalone{figures/histology}
\caption{Histological analysis of lesion size. (\textbf{A}) Representative example of bilateral ibotenic acid lesion of primary and secondary motor cortex. (\textbf{B}) Distribution of lesion volumes in the left and right hemispheres for individual animals. A lesion was considered ``large'' if the total lesion volume was above \SI{15}{\milli\meter\cubed}. (\textbf{C}) Super-imposed reconstruction stacks for all the small lesions. (\textbf{D}) Super-imposed reconstruction stacks for all the large lesions.}
\label{fig:histology}
\end{figure}

\begin{figure}
\centering
\includestandalone{figures/learning}
\caption{Overall performance on the obstacle course is similar for both lesion and control animals. (\textbf{A}) Average time to cross the obstacles on the different protocol stages for forelimb motor cortex lesions and control animals. Each set of three coloured bars represents the average performance on a single session. Asterisks indicate sessions where there was a change in protocol conditions (see text). In those sessions, the average performance on the 20 trials immediately preceding the change is shown to the left of the solid vertical line whereas the performance on the remainder of that session (after the change) is shown to the right. (\textbf{B}) Average time required to cross the obstacles in the stable condition for extended lesions. Performance of the other groups is shown for comparison. (\textbf{C}) Average number of slips per crossing in early versus late sessions of the stable condition.}
\label{fig:learning}
\end{figure}

\begin{figure}
\centering
\includestandalone{figures/posture}
\caption{Rats adapt their postural approach to the obstacles after a change in physics. (\textbf{A}) Schematic of postural analysis image processing. The position of the animal's nose is extracted whenever the paw activates the ROI of the first manipulated step (see methods). (\textbf{B}) The position of the nose in single trials for one of the control animals stepping across the different conditions of the shuttling protocol. (\textbf{C}) Average position of the nose across the different protocol stages for both lesion and control animals. Asterisks indicate the average nose position on the 20 trials immediately preceding a change in protocol conditions (see text). (\textbf{D}) Distribution of nose positions for control animals in the last two days of the stable (blue) and unstable (orange) protocol stages. Respectively for (\textbf{E}) small lesions and (\textbf{F}) large lesions.}
\label{fig:posture}
\end{figure}

\begin{figure}
\centering
\includestandalone{figures/jumping}
\caption{Animals use different strategies for dealing with the manipulated obstacles. (\textbf{A}) Example average projection of posture image stacks for stable (green) and unstable (red) sessions for two non-jumper and two jumper animals. (\textbf{B}) Average nose trajectories for individual animals crossing the unstable condition. The shaded area around each line represents the 95\% confidence interval. (\textbf{C}) Correlation of the probability of skipping the center two steps with the weight of the animal. The color of each dot indicates whether the animal was a control or a lesion.}
\label{fig:jumping}
\end{figure}

\begin{figure}
\centering
\includestandalone{figures/random}
\caption{Animals adjust their posture on a trial-by-trial basis to the expected state of the world. (\textbf{A}) Distribution of nose positions for all animals in stable (blue) and unstable (orange) trials of the randomized protocol (see methods). (\textbf{B}) Distribution of nose positions for trials in which the last trial was stable (blue) or unstable (orange). (\textbf{C}) Respectively for trials where the last two trials were stable or unstable. (\textbf{D-F}) Same data as in (\textbf{C}) split by the control and lesion groups.}
\label{fig:random}
\end{figure}

\begin{figure}
\centering
\includestandalone{figures/speed}
\caption{Uncertainty in the expected state of the world causes the animals to rapidly adjust their movement trajectory. (\textbf{A}) Example average speed profile for stable (blue) and unstable (orange) trials in the randomized sessions of a control animal (see text). The shaded area around each line represents the 95\% confidence interval. (\textbf{B}) Respectively for one of the largest lesions. (\textbf{C}) Summary of the average difference between the speed profiles for stable and unstable trials across the two groups of animals. Error bars show standard error of the mean.}
\label{fig:speed}
\end{figure}

\begin{figure}
\centering
\includestandalone{figures/ethogram}
\caption{First response to an unexpected change in the environment. (\textbf{A}) Response types observed across individuals upon first encountering an unpredicted shift in the state of the centre obstacles. (\textbf{B}) Ethogram of behavioural responses classified according to the three criteria described in (\textbf{A}) and aligned (0.0) on first contact with the newly manipulated obstacle. Black dashes indicate when the animal flicks its ears. White indicates the animal has left the obstacle course.}
\label{fig:ethogram}
\end{figure}

\end{linenumbers}

% Allow breaks at numbers and letters for bibliography
\setcounter{biburlnumpenalty}{100}
\setcounter{biburlucpenalty}{100}
\setcounter{biburllcpenalty}{100}
\hyphenation{Gross-hirns}
\printbibliography

\end{document}