% Motor cortex paper draft (mc-paper.tex)
% Copyright (c) Kampff Lab 2016
% See README.md for redistribution license

\documentclass[12pt]{article}
\usepackage[T1]{fontenc}
\usepackage[utf8]{inputenc}
\usepackage{hyphenat}
\usepackage{siunitx}

%%% Preprint settings
\usepackage{setspace}
\usepackage{lineno}
\doublespacing

% For author lists
\usepackage{authblk}
\renewcommand\Authsep{ \qquad }
\renewcommand\Authand{ \qquad }
\renewcommand\Authands{ \qquad }

%%% For figures
\usepackage{tikz}
\usepackage[mode=buildnew]{standalone}

%%% For video references
\newcounter{video}
\newcommand{\videolabel}[1]{%
  \refstepcounter{video}\label{#1}%
}

\title{A robust role for motor cortex}
\author{Gonçalo Lopes}
\author{Joana Nogueira}
\author{Joseph J. Paton}
\author{Adam R. Kampff}
\affil{Champalimaud Neuroscience Programme\\ Champalimaud Centre for the Unknown, Lisbon, Portugal}

% bibliography
\usepackage{doi}
\usepackage[backend=bibtex,url=false,isbn=false,eprint=false,sorting=none,giveninits=true]{biblatex}
\makeatletter
\def\blx@maxline{77}
\makeatother
\addbibresource{references.bib}

\begin{document}
\maketitle

\begin{linenumbers}
\begin{abstract}
The role of motor cortex in the direct control of movement remains unclear, particularly in non-primate mammals. More than a century of research using stimulation, anatomical and electrophysiological studies has implicated neural activity in this region with all kinds of movement. However, following the removal of motor cortex, or even the entire cortex, rats retain the ability to execute a surprisingly large range of adaptive behaviours, including previously learned skilled movements. In this work we revisit these two conflicting views of motor cortical control by asking what the primordial role of motor cortex is in non-primate mammals, and how it can be effectively assayed. In order to motivate the discussion we present a new assay of behaviour in the rat, challenging animals to produce robust responses to unexpected and unpredictable situations while navigating a dynamic obstacle course. Surprisingly, we found that rats with motor cortical lesions show clear impairments in dealing with an unexpected collapse of the obstacles, while showing virtually no impairment with repeated trials in many other motor and cognitive metrics of performance. We propose a new role for motor cortex: extending the robustness of sub-cortical movement systems, specifically to unexpected situations demanding rapid motor responses adapted to environmental context. The implications of this idea for current and future research are discussed.
\end{abstract}


\videolabel{vid:learning}
\videolabel{vid:learning-matrix}
\videolabel{vid:decorticate-habituation}

\videolabel{vid:conditions}
\videolabel{vid:jpak20}

\videolabel{vid:manipulation-strategies}
\videolabel{vid:manipulation-small}
\videolabel{vid:manipulation-large}

\videolabel{vid:manipulation-decorticate-oblivious}
\videolabel{vid:manipulation-decorticate-halting}

\section{Introduction}

Here we will discuss the role of motor cortex in controlling movement.  Motor cortex may play a role in "understanding" the observed movements of others, imaging your own movements, or in learning new movements, but here we focus on its role in controlling movement. How does neural activity in cortical motor areas influence movement of the body? Are there specific types of movements for which motor cortex is required? Are there behaviours that an animal cannot perform without motor cortex? Motor cortex is active during movement, stimulating motor cortex ellicits movevement, and lesions to motor cortex (in some mammals) cause lasting deficits in motor control. Its role seems obvious: motor cortex is the part of cortex that produces movement of the body. Yet, it is not so simple.

We were motivated to open this discussion based on the following observation: when a human motor cortex is damaged, by stroke or other lesion, then a paitient suffers severe and conspicuous deficits in movement. However, when lower mammals (rats, cats, dogs, and monkeys) suffer similar damage, then the effect on movement is much less obvious. In fact, rats without motor cortex, or even most of cortex, do not show lasting problems in movement...as assayed with a battery of standard behavioural tests. This presents a dillemma. Although it is clear that motor cortex is important for movement in humans (e.g. humans with lesions of secondary motor areas called "Broca's area" lose, and may never recover, the ability to speak), but thus is much less clear in other mammals. Is there a general "role" for motor cortex? We will try to describe such a general role, and to provide an explanation for the apparent human-rat discrepancy, but to do so, we must first review a century of evidence that has led to our current views about what motor cortex is and what it is good for. 

This discrepancy was apparent from the beginning. 

Our review will begin with the discovery of motor cortex. In the 1880s, a lovely series of experiments isolated a region of the cortex that, when stimulated, could evoke movements. However, lesioning this area produced confusing results (and they still do). Monkeys with lesions showed human-like gross motor impariments, but dogs with similar lesion extents recovered remarkably well. These seemingly conflicting results were largely ignored...primates were more closely related to humans, so the basic premise that a localized part of cortex moves "us" both was accepted, and it's just different in dogs. What does dog (or cat and rat) motor cortex do? An open question.

We all evoled from a little rodent-like creature. It had an expanded telencephalon and likely a cortical organization, which began to expand, ultimately becoming 70% of the human brain. What did this cortical structure influe nce about movement and behaviour in thes eprimoridial relatives? If nothing, then why did it evolve? 

We then survey studies of motor cortex anatomy and its output projections. It was 

We will discuss the physiology of motor cortex, both recording experiments that have identified a number

However, we will priminarily focus on experiments that disrupt the function of motor cortex and then chacraterize changes in behaviour. These experiments have involved a range of lesion size, form thos targeting "primary" motor cortical areas to complete decortications (i.e. total removal of cortex). These results reveal a clear pattern. As one transcends the "mammalian hieracrchy" the overt, conspicuous deficits in behaviour resulting from damage to motor cortex increase. Humans not only need motor cortex to spear, they are the only species that needs motor cortex to walk! Rats seem fine, cats and dogs as well, whereas primates are more obviously impaired, yet do show impresisve recovery. This graded importance of motor cortex for movement immediately suggests that motor cortex plays an increasingly "important" role in higer, more encephalized, mammals. However, as we shall see

comment on dextererity and volutnary control/


After review our past, we will revisit the dilemma of motro cortex from another perspective. What WOULD we need a motor cortex for? What are the HARD motor control probelms that we ecounter in the environment for which a higher motor control system would be needed? Traditional assays of behaviour, and the deficits that results of lesions to motor cortex, do not necessarly test the range of problems animals would have encountered in their envieoment,mes...amd would have required elegant solutions to. We will discuss the ecological dilemmas of animals by caharcateizing the nature of problems faced by those scientists trying to build new machines for the natural enviornment, robotics. Here the probelsm are clear...and they are not the ones for which we have ascribed a role to our brain's highest motor control system. Moving with precision and "fine dexterity" is not hard for modern robots. We rely on precisely this skill in all industrila robotics, many of which vastly exceed human level performance. Yet, these robots are not wandering around amongst us. Why? They fall over.

Here is the dilemma, and here is a new probelm...do they fit tgether? Have they been addrssed before. 

We will next outline the histriy of robustness and its cortical requirement. 

Then we will propose and run a simple experiment. How do rats repsond to unexpected problems. We wil present a new assay, and assess the role of motor cortex by making targeted lesions to areas responsible for forepaw control. These results suggest a ...but will require an entirely new kind of assays for motor cortical function. Thus we will conclude with some pontificatiin. What does this new framework for otor cortex (and even cortex in general) mean for futue neuoscience? What does it mean for the evoiltuon of mammalian brains?

There is a lot to say, and we will say it, but mind you, this is just the beginning. We wre likely wrong and major and minor points...but we want a more "rubust" understanding of cortex. So we start a conversation.



box: What do we mean by "Motor Cortex"?



box: What do we mean be teleology?


\section{Results}

To investigate whether the intact motor cortex is required for the robust control of movement in response to unexpected perturbations, we designed a reconfigurable dynamic obstacle course where individual steps can be made stable or unstable on a trial-by-trial basis (Figure \ref{fig:assay}, also see Methods). In this assay, rats shuttle back and forth across the obstacles, in the dark, in order to collect water rewards. We specifically designed the assay such that modifications to the physics of the obstacles could be made covertly. In this way, the animal has no explicit information about the state of the steps until it actually contacts them. Water deprived animals were trained daily for 4 weeks, throughout which they encountered increasingly challenging states of the obstacle course. Our goal was to characterize precisely the conditions under which motor cortex becomes necessary for the control of movement, and this motivated us to introduce an environment with graded levels of uncertainty.

We compared the performance of 22 animals: 11 with bilateral ibotenic acid lesions to the primary and secondary forelimb motor cortex, and 11 age and gender matched controls (5 sham surgery, 6 wild-types). Animals were given ample time to recover, 4 weeks post-surgery, in order to specifically isolate behaviours that are chronically impaired in animals lacking the functions enabled by motor cortical structures. Histological examination of serial coronal sections revealed significant variability in the extent of damaged areas (Figure \ref{fig:histology}), which was likely caused by mechanical blockage of the injection pipette during lesion induction at some sites. Nevertheless, volume reconstruction of the serial sections allowed us to accurately quantify the size of each lesion, identify each animal (from Lesion A to Lesion K; largest to smallest), and use these values to compare observed behavioural effects as a function of lesion size.

During the first sessions in the ``stable'' environment, all animals, both lesions and controls, quickly learned to shuttle across the obstacles, achieving stable, skilled performance after a few days of training (Figure \ref{fig:learning}). Even though the distance between steps was fixed for all animals, the time taken to adapt the crossing strategy was similar irrespective of body size. When first encountering the obstacles, animals adopted a cautious gait, investigating the location of the subsequent obstacle with their whiskers, stepping with the leading forepaw followed by a step to the same position with the trailing paw (Video \ref{vid:learning}: ``First Leftwards Crossing''). However, over the course of only a few trials, all animals exhibited a new strategy of ``stepping over'' the planted forepaw to the next obstacle, suggesting an increased confidence in their movement strategy in this novel environment (Video \ref{vid:learning}: ``Second Leftwards Crossing''). This more confident gait developed into a coordinated locomotion sequence after a few additional training sessions (Video \ref{vid:learning}: ``Later Crossing''). The development of the ability to move confidently and quickly over the obstacle course was observed in both lesion and control animals (Video \ref{vid:learning-matrix}).

In addition to the excitotoxic lesions, in three animals we performed larger frontal cortex aspiration lesions in order to determine whether the remaining trunk and hindlimb representations were necessary to navigate the elevated obstacle course. Also, in order to exclude the involvement of other corticospinal projecting regions in the parietal and rostral visual areas \cite{Miller1987}, we included three additional animals which underwent even more extensive cortical lesion procedures (Figure \ref{fig:extended}A,B, see Methods). These \emph{extended} lesion animals were identified following chronological order (from Extended Lesion A to Extended Lesion F; where the first three animals correspond to frontal cortex aspiration lesions and the remaining animals to the more extensive frontoparietal lesions). In these extended cortical lesions, recovery was found to be overall slower than in lesions limited to the motor cortex, and animals required isolation and more extensive care during the recovery period.

Nevertheless, when tested in the shuttling assay, the basic performance of these extended lesion animals was similar to that of controls and animals with excitotoxic motor cortical lesions (Figure \ref{fig:extended}C). Animals with large frontoparietal lesions did exhibit a very noticeable deficit in paw placement throughout the early sessions (Figure \ref{fig:extended}D). Interestingly, detailed analysis of paw placement behaviour revealed that this deficit was almost entirely explained by impaired control of the hindlimbs. Paw slips were much more frequent when stepping with a hindlimb than with a forelimb (Figure \ref{fig:extended}E,F). In addition, when a slip did occur, these animals failed to adjust the affected paw to compensate for the fall (e.g. keeping their digits closed), which significantly impacted their overall posture recovery. These deficits in paw placement are consistent with results from sectioning the entire pyramidal tract in cats \cite{Liddell1944}, and reports in ladder walking following motor cortical lesion in rodents \cite{Metz2002}, but surprisingly we did not observe deficits in paw placement in animals with ibotenic acid lesions limited to forelimb motor cortex (Figure \ref{fig:extended}D). Furthermore, despite this initial impairment, animals with extended lesions were still able to improve their motor control strategy up to the point where they were moving across the obstacles as efficiently as controls and other lesioned animals (Figure \ref{fig:extended}C, Video \ref{vid:learning-matrix}). Indeed, in the largest frontoparietal lesion, which extended all the way to rostral visual cortex, recovery of a stable locomotion pattern was evident over the course of just ten repeated trials (Video \ref{vid:decorticate-habituation}). The ability of this animal to improve its motor control strategy in such a short period of time seems to indicate the presence of motor learning, not simply an increase in confidence with the new environment.

In subsequent training sessions we progressively increased the difficulty of the obstacle course, by making more steps unstable. The goal was to compare the performance of the two groups as a function of difficulty. Surprisingly, both lesion and control animals were able to improve their performance by the end of each training stage even for the most extreme condition where all steps were unstable (Figure \ref{fig:learning}, Video \ref{vid:conditions}). This seems to indicate that the ability of these animals to fine-tune their motor performance in a challenging environment remained intact.

One noticeable exception was the animal with the largest ibotenic acid lesion. This animal, following exposure to the first unstable protocol, was unable to bring itself to cross the obstacle course (Video \ref{vid:jpak20}). Some other control and lesioned animals also experienced a similar form of distress following exposure to the unstable obstacles, but eventually all these animals managed to start crossing over the course of a single session. In order to test whether this was due to some kind of motor disability, we lowered the difficulty of the protocol for this one animal until it was able to cross again. Following a random permutation protocol, where any two single steps were released randomly, this animal was then able to cross a single released obstacle placed in any location of the assay. After this success, it eventually learned to cross the highest difficulty level in the assay in about the same time as all the other animals, suggesting that there was indeed no lasting motor execution or learning deficit, and that the disability must have been due to some other unknown, yet intriguing, (cognitive) factor.

Having established that the overall motor performance of these animals was similar across all conditions, we next asked whether there was any difference in the strategy used by the two groups of animals to cross the unstable obstacles. We noticed that during the first week of training, the posture of the animals when stepping on the obstacles changed significantly over time (Figure \ref{fig:posture}B,C). Specifically, the centre of gravity of the body was shifted further forward and higher during later sessions, in a manner proportional to performance. However, after the obstacles changed to the unstable state, we observed an immediate and persistent adjustment of this crossing posture, with animals assuming a lower centre of gravity and reducing their speed as they approached the unstable obstacles (Figure \ref{fig:posture}C,D). Interestingly, we also noticed that a group of animals adopted a different strategy. Instead of lowering their centre of gravity, they either kept it unchanged or shifted it even more forward and performed a jump over the unstable obstacles (Figure \ref{fig:jumping}A,B). These two strategies were remarkably consistent across the two groups, but there was no correlation between the strategy used and the degree of motor cortical lesion (Figure \ref{fig:posture}E,F, \ref{fig:jumping}C). In fact, we found that the use of a jumping strategy was best predicted by the body weight of the animal (Figure \ref{fig:jumping}C).

During the two days where the stable state of the environment was reinstated, the posture of the animals was gradually restored to pre-manipulation levels (Figure \ref{fig:posture}B,C), although in many cases this adjustment happened at a slower rate than the transition from stable to unstable. Again, this postural adaptation was independent of the presence or absence of forepaw motor cortex.

We next looked in detail at the days where the state of the obstacle course was randomized on a trial-by-trial basis. This stage of the protocol is particularly interesting as it reflects a situation where the environment has a persistent degree of uncertainty. For this analysis, we were forced to exclude the animals that employed a jumping strategy, as their experience with the manipulated obstacles was the same irrespective of the state of the world. First, we repeated the same posture analysis comparing all the stable and unstable trials in the random protocol in order to control for whether there was any subtle cue in our motorized setup that the animals might be using to gain information about the current state of the world. There was no significant difference between randomly presented stable and unstable trials on the approach posture of the animal (Figure \ref{fig:random}A). However, classifying the trials on the basis of past trial history revealed a significant effect on posture (Figure \ref{fig:random}B). This suggested that the animals were adjusting their body posture when stepping on the affected obstacles on the basis of their current expectation about the state of the world, which is updated by the previously experienced state. Surprisingly, this effect again did not depend on the presence or absence of frontal motor cortical structures (Figure \ref{fig:random}C,D).

Finally, we decided to test whether general motor performance was affected by the randomized state of the obstacles. If the animals do not know what state the world will be in, then there will be an increased challenge to their stability when they cross over the unstable obstacles, possibly demanding a quick change in strategy when they learn whether the world is stable or unstable. In order to evaluate the dynamics of crossing, we compared the animals' speed profiles across these different conditions (Figure \ref{fig:speed}, see Methods). Interestingly, two of the animals with the largest lesions appeared to be significantly slowed down on unstable trials, while controls and the animals with the smallest lesions instead tended to accelerate after encountering an unstable obstacle. However, the overall effect for lesions versus controls was not statistically significant (Figure \ref{fig:speed}C).

Nevertheless, we were intrigued by this observation and decided to investigate, in detail, the first moment in the assay when a perturbation is encountered. In the random protocol, even though the state of the world is unpredictable, the animals know that the obstacles might become unstable. However, the very first time the environment becomes unstable, the collapse of the obstacles is completely unexpected and demands an entirely novel motor response.

A detailed analysis of the responses to the first collapse of the steps revealed a striking difference in the strategies deployed by the lesion and control animals. Upon the first encounter with the manipulated steps, we observed three types of behavioural responses from the animals (Video \ref{vid:manipulation-strategies}): investigation, in which the animals immediately stop their progression and orient towards, whisk, and physically manipulate the altered obstacle; compensation, in which the animals rapidly adjust their behaviour to negotiate the unexpected instability; and halting, in which the ongoing motor program ceases and the animals' behaviour simply comes to a stop for several seconds. Remarkably, these responses depended on the presence or absence of motor cortex (Figure \ref{fig:ethogram}). Animals with the largest motor cortical lesions, upon their first encounter with the novel environmental obstacle, halted for several seconds, whereas animals with an intact motor cortex, and those with the smallest lesions, were able to rapidly react with either an investigatory or compensatory response (Video \ref{vid:manipulation-small},\ref{vid:manipulation-large}).

The response of animals with extended lesions was even more striking -- in two of these animals, there was a failure to recognize that a change had occurred at all (Video \ref{vid:manipulation-decorticate-oblivious}). Instead, they kept walking across the now unstable steps for several trials, never stopping to assess the new situation. One of them gradually noticed the manipulation and stopped his progression, while the other one only fully realized the change after inadvertently hitting the steps with its snout (Video \ref{vid:manipulation-decorticate-oblivious}: Extended Lesion A). This was the first time we ever observed this behaviour, as all animals with or without cortical lesions always displayed a clear switch in behavioural state following the first encounter with the manipulation. In the remaining animals with extended lesions, two of them clearly halted their progression following the collapse of the obstacles, in a way similar to the large motor cortex ibotenic lesions (Video \ref{vid:manipulation-decorticate-halting}). The third animal (Extended Lesion B) actually collapsed upon contact with the manipulated step, falling over its paw and digits awkwardly and hitting the obstacles with its snout. Shortly after this there was a switch to an exploratory behaviour state, in a way similar to Extended Lesion A.

\section{Experiment Discussion}

Consistent with previous studies, we did not observe any conspicuous deficits in movement execution for rats with bilateral motor cortex lesions when negotiating a stable environment. Even when exposed to much unstable obstacles, all animals were able to learn an efficient strategy for crossing these more challenging environments, with or without motor cortex. These movement strategies include a preparatory component that seems to reflect the state of the world an animal expects to encounter. Surprisingly, these preparatory responses also did not require the presence of motor cortex.

It was only when the environment did not conform to expectation, and demanded a rapid adjustment, that a difference between the lesion and control groups was obvious. Animals with extensive damage to the motor cortex did not deploy a change in strategy. Rather, they halted their progression for several seconds, unable to robustly respond to the new motor challenge. In an ecological setting, such hesitation could easily prove fatal.

\section{Discussion}

Is “robust control” a problem worthy of high level motor cortical input? Recovering from a perturbation, to maintain balance or minimize the impact of a fall, is a role normally assigned to our low level posture control system. The corrective responses embedded in our spinal cord networks (cite: knee jerk) and brainstem (cite: vestibular) are clearly important components of this stabilizing process, but are they sufficient to maintain robust movement within the complex environments we encounter, and elegantly negotiate, on a daily basis? Some insight into the requirements for designing a robust control system can be gained from engineers’ trying to build robots that can navigate in natural environments.

In the field of robotics, feats of precision and fine movement control (the most commonly prescribed role for motor cortex), are not a major source of difficulty. Industrial robots have long since exceeded human performance in both accuracy and execution speed. More recently, using reinforcement learning methods, they are now able to learn efficient movement strategies, given a human-defined goal and many repeated trials for fine-tuning. What then are the hard problems in robotic motor control? Most robots are still confined to factories, i.e. controlled, predictable environments. However, as soon as a mobile robot needs to negotiate natural terrain, a vast number of previously unknown situations arise. The non-linearities inherent to these unexpected states and resulting “perturbations” are dealt with poorly by the statistical machine learning models that are currently used to train robots in controlled settings.

Let’s consider a familiar example: You are up early on a Sunday morning and head outside to collect the newspaper. It’s cold out, so you put on a robe and some slippers, open the front door, and descend the steps leading down to the street in front of your house. Unbeknownst to you, a thin sheet of ice has formed overnight and your foot is now quickly sliding out from underneath you. You are about to fall. What do you do? Well, this depends. Is there a railing you can grab to catch yourself? Were you carrying a cup of coffee? Did you notice the frost on the lawn and step more cautiously, anticipating a slippery surface? Avoiding a dangerous fall, or recovering gracefully, requires a rich knowledge of the world, knowledge that is not immediately available to spinal or even brainstem circuits. This rich context relevant for movement is available to cortex, and cortex alone.

Imagine you were tasked with building a robot to collect your morning newspaper. This robot, in order to avoid a catastrophic and costly failure would need to have all of this contextual knowledge as well. It would ned to know about the structure of the local environment (hand railings that can support its weight), hot liquids and their viscosities, and even the correlation of frozen dew with icy surfaces. To be a truly robust movement machine, a robot must understand the world. This remains the hard problem for artificial intelligence, not playing chess or go [cite], and this may be a fitting role for our highest movement controller, motor cortex.

Given that clever posture corrections are not the sort of feats for which we praise our athletes and sports champions, it is not surprising that the difficulty of this problem has been under appreciated. It also would not be the first time that we find ourselves humbled by a problem that we initially considered trivial. Vision, for example, remains an impressively hard task for a machine to solve at human-level performance, yet it was originally proposed as an undergraduate summer project (cite: Minsky).

A primordial role for motor cortex:

We are seeking a role for motor cortex in lower (non-primate) mammals, animals that do not appear to require this structure for overt movement production. The struggles of roboticists highlight the challenge of building movement systems that robustly adapt to unexpected perturbations and the results we report in this study suggest that this is, indeed, the most conspicuous deficit of rats lacking motor cortex. So let us propose that, in rodents, motor cortex is primarily responsible for extending the robustness of the sub-cortical movement systems. It is not required for control in stable, predictable, non-perturbing environments, but instead specifically exerts its influence when unexpected challenges arise. This, we propose, was the original selective pressure for evolving a motor cortex, and thus its primordial role. This role persists in lower mammals and has been elaborated upon in primates. Our proposal of a “robust” teleology for motor cortex has a number of interesting implications.


Implications for lower mammals:

One of the most impressive traits of mammals is the vast range of environmental niches that they occupy [cite]. This flexibility requires the ability to quickly evaluate and respond to unexpected situations, many of which have never been previously encountered. Nonetheless, mammals are able to thrive in such environments. This success requires more than precision, it requires robustness, the ability to quickly come up with a passable motor solution for any situation, in any condition \cite{Bernstein1996}. Bernstein referred to this ability with his unconventional definition of “dexterity”, distinct from a simple harmony in precise hand movements. Such dexterity is required when there is \enquote{a conglomerate of unexpected, unique complications in the external situations, in a quick succession of motor tasks that are all unlike each other} \cite{Bernstein1996}.

If “robust dexterity” is the primary role for motor cortex, then it is clear why the effects of lesions have thus far been so hard to characterize: assays of motor behaviour typically rely on assessment in situations that are repeated over many trials, in the same environment. Such repeated tasks were necessary, as they offer improved statistics for quantification and comparison. However, these conditions may specifically exclude the scenarios for which motor cortex originally evolved. However, it is not easy to repeatedly produce situations that animals have not previously encountered, and even if it was, the challenges in analysing these unique situations are considerable.

The assay reported here represents our first attempt at such an experiment, and it has already revealed that such conditions may indeed by necessary to isolate the role of motor cortex. We thus propose that neuroscience should pursue similar assays, emphasizing perturbations and novel challenges, and we have worked on developing new hardware and software tools to make their design much easier (cite: Bonsai).



Implications for primates:

In contrast to lower mammals, primates rely on motor cortex for the direct control of movement. However, do they also retain its role in generating robust responses? The general paresis, or even paralysis, that results from motor cortical lesions in primates will trivially obscure any of its involvement in directing rapid responses to perturbations. Yet there is still evidence that a role in robust control is also present in primates, including humans. For example, stroke patients with partial lesions to the distributed motor cortical system will often recover the ability to move the affected bid parts. However, even after recovering movement, stroke patients are still prone to devastating errors in robust control. In fact, unsupported falls are one of the leading causes of injury and death in patients surviving motor cortical stroke \cite{Jacobs2014}. We thus suggest that stroke therapy, currently focused on regaining direct movement control, should also consider strategies for improving robustness.

Even if we acknowledge that a primordial role of motor cortex is still apparent in primate movement control, it remains to be explained why the motor cortex of these species has acquired control of basic movements. This remains an open question, but the consequences are intriguing.


What happens when cortex acquires direct control of movement? First, it must learn how to use this influence, which may explain the extended developmental times required for primates to produce basic motor control. Humans babies require years of practice to produce simple locomotion and grasping [cite], motor behaviours that are available to lower mammals almost immediately after birth. This may be the cost of giving cortex control, it takes time to figure out how to move the body, so what is the benefit? 

A robust control system (i.e. cortex), in full command of the body, is capable of negotiating more difficult environmental niches. Primates may have been able to ascend trees, and avoid their less “dexterous” predators, with a feat of motor cortical control. However, the implications of this cortical “take-over” are potentially even more profound. 

If cortex has direct control of overt movements, then observing the behaviour of a primate constitutes direct observation of commands arising in cortex(vs. sub-cortical pattern generators). In other words, when you watch a primate move you are directly observing cortical commands. Now, if a species reliant on direct cortical control happened to live in social groups, then members of this group would have a uniquely efficient means of observing the state of cortex in a conspecific; what was going on in cortex would be directly observable in the movements of that individual. The implications of establishing an efficient means of “communicating” the state of one cortex to another are simply exhilarating to imagine…



Some Conclusions:

Clearly our results are insufficient to draft a final conclusion, but that is not our goal. We present our experiments to support and motivate our effort to distil a long history of research, and ultimately suggest a new approach to investigating the role of motor cortex. This approach most directly applies to studies of lower mammals, for which we now have a host of techniques to monitor and manipulate cortical activity during behaviour. However, we propose that we should be monitoring and manipulating activity during behaviours that actually require motor cortex. 

This synthesis also has implications for human studies. We suggest that acknowledging a primary role for motor cortex in robust control, a problem still daunting to robotics engineers, can guide the development of new approaches to building intelligent machines, as well as new strategies to assess and treat patients with motor cortical damage. We concede that our results are still naïf, but propose that the implications our worthy of your consideration.


\section{Methods}

All experiments were approved by the Champalimaud Foundation Bioethics Committee and the Portuguese National Authority for Animal Health, Direcção\hyp{}Geral de Alimentação e Veterinária (DGAV).

\subsection{Permanent lesions}

Ibotenic acid was injected bilaterally in 11 Long-Evans rats (ages from 83 to 141 days; 9 females, 2 males), at 3 injection sites with 2 depths per site (\SI{-1.5}{\milli\meter} and \SI{-0.75}{\milli\meter} from the surface of the brain). At each depth we injected a total amount of \SI{82.8}{\nano\liter} using a microinjector (Drummond Nanoject II, \SI{9.2}{\nano\liter} per injection, 9 injections per depth). The coordinates for each site, in \si{\milli\meter} with respect to Bregma, were: +1.0 AP / 2.0 ML; +1.0 AP / 4.0 ML; +3.0 AP / 2.0 ML, following the protocol reported by Kawai et al. for targeting forelimb motor cortex \citep{Kawai2015}. Five other animals were used as sham controls (age-matched controls; 3 females, 2 males), subject to the same intervention, but where ibotenic acid was replaced with physiological saline. Six additional animals were used as wildtype, no-surgery, controls (age-matched controls; 6 females).

For the frontal cortex aspiration lesions, the margins of the craniotomy were extended to cover from -2.0 to +5.0 \si{\milli\meter} AP relative to Bregma and laterally from 0.5 \si{\milli\meter} up to the temporal ridge of the skull. After removal of the skull, the exposed dura was cut and removed, and the underlying tissue aspirated to a depth of 2 to 3 \si{\milli\meter} with a fine pipette \citep{Whishaw2000}. For the frontoparietal cortical lesions, the craniotomy extended from -6.0 to +4.0 \si{\milli\meter} AP relative to Bregma and laterally from 0.5 \si{\milli\meter} up to the temporal ridge. Two of these animals underwent aspiration lesions as described above. In the remaining animal, the lesion was induced by pial stripping in order to further restrict the damage to cortical areas. After removal of the dura, the underlying pia, arachnoid and vasculature were wiped with a sterile cotton swab until no vasculature was visible \citep{Farr2002}.

\subsection{Recovery period}

After the surgeries, animals were given a minimum of one week (up to two weeks) recovery period in isolation. After this period, animals were handled every day for a week, after which they were paired again with their age-matched control to allow for social interaction during the remainder of the recovery period. In total, all animals were allowed at least one full month of recovery before they were first exposed to the behaviour assay.

The three largest frontoparietal lesioned animals were originally prepared for a study of behaviour in a dynamic visual foraging task, which they were exposed to for one month in addition to the recovery period described above. This task did not, however, require any challenging motor behaviours besides locomotion over a completely flat surface. This period was also used to monitor the overall health condition of the animals and to facilitate sensorimotor recovery as much as possible. The animal with the largest lesion (Extended Lesion F) was prevented from completing the behaviour protocol due to deteriorating health conditions following the first two days of testing.

\subsection{Histology}

All animals were perfused intracardially with 4\% paraformal\-dehyde in phosphate buffer saline (PBS) and brains were post-fixed for at least \SI{24}{\hour} in the same fixative. Serial coronal sections (\SI{100}{\micro\meter}) were Nissl-stained and imaged for identification of lesion boundaries. In two of the largest frontoparietal lesions (Extended Lesions D and E), serial sections were taken sagittally.

In order to reconstruct lesion volumes, the images of coronal sections were aligned and the outlines of both brain and lesions were manually traced in Fiji \citep{Schindelin2012} and stored as two-dimensional regions of interest. Lesion volumes were calculated by summing the area of each region of interest multiplied by the thickness of each slice. The stored regions were also used to reconstruct a 3D polygon mesh for visualization of lesion boundaries.

\subsection{Electrocorticography}

Recording of electrophysiological signals from the intact rodent cortex was performed using two high-density 64-channel micro-electrocorticography (micro-ECoG) grids using the method reported by Dimitriadis et al. for freely moving animals \citep{Dimitriadis2014}. The particular grids used in these experiments were fabricated at the International Iberian Nanotechnology Laboratory by depositing microelectrode gold contacts through a custom designed layout mask on a flexible thin-film polyimide substrate (Figure \ref{fig:ecog}A). The soft connectors at the end of each grid are inserted during the implantation surgery into a custom made breakout board in the recording chamber, which exposes groups of 32-channels via Omnetics connectors to the recording amplifier (see data acquisition section).

The microelectrode grids were implanted epidurally into the right hemisphere of three male Long-Evans rats at almost two years of age. The margins of the craniotomy for implantation extended from -3.3 to +5.0 \si{\milli\meter} AP relative to Bregma and laterally from 1.5 to 4.0 \si{\milli\meter}. The anterior grid was first placed carefully on top of the brain, and then slowly inserted below the anterior and medial margins of the craniotomy until the first rows of electrodes were fully covered. The second grid was placed posterior to the first one and inserted below the medial margin of the craniotomy, taking care that the first rows of electrodes were kept equidistant from the last row of electrodes in the anterior grid. Two zirconium hooks were inserted in the anterior and posterior margins of the craniotomy and fixed to the recording chamber in order to hold it firmly in place relative to the skull. With the aid of a micromanipulator and video feedback system, the coordinates of different electrodes in each quadrant of both grids were measured relative to Bregma, and later used to reconstruct the precise placement of all grid electrodes in the brain. At the end of the surgery, a titanium screw was inserted posteriorly to the craniotomy in contact with the brain in order to be used as reference for the recording system. The stability of the implant depends critically on the absence of movement in the bony plates of the skull during development, which can compromise the mechanical fixation of the recording chamber to the head \citep{Dimitriadis2014}. For this reason, it is recommended that rats undergoing this procedure should be older than 7 months \citep{Dimitriadis2014}.

\subsection{Behaviour assay}

During each session the animal was placed inside a behaviour box for \SI{30}{\minute}, where it could collect water rewards by shuttling back and forth between two nose pokes (Island Motion Corporation, USA). To do this, animals had to cross a \SI{48}{\centi\meter} obstacle course composed of eight \SI{2}{\centi\meter} aluminium steps spaced by \SI{4}{\centi\meter} (Figure \ref{fig:assay}A). The structure of the assay and each step in the obstacle course was built out of aluminium structural framing (Bosch Rexroth, DE, \SI{20}{\milli\meter} series). The walls of the arena were fabricated with a laser-cutter from \SI{5}{\milli\meter} thick opaque black acrylic and fixed to the structural framing. A transparent acrylic window partition was positioned in front of the obstacle course in order to provide a clear view of the animal. All experiments were run in the dark by having the behavioural apparatus enclosed in a light tight box.

A motorized brake allowed us to lock or release each step in the obstacle course (Figure \ref{fig:assay}B). The shaft of each of the obstacles was coupled to an acrylic piece used to control the rotational stability of each step. In order to lock a step in a fixed position, two servo motors are actuated to press against the acrylic piece and hold it in place. Two other acrylic pieces were used as stops to ensure a maximum rotation angle of approximately +/- \ang{100}. Two small nuts were attached to the bottom of each step to work as a counterweight that gives the obstacles a tendency to return to their original flat configuration. In order to ensure that noise from servo motor actuation could not be used as a cue to tell the animal about the state of each step, the motors were always set to press against an acrylic piece, either the piece that keeps the step stabilized, or the acrylic stops. At the beginning of each trial, the motors were run through a randomized sequence of positions in order to mask information about state transitions and also to ensure the steps were reset to their original configuration. Control of the motors was done using a Motoruino board (Artica, PT) along with a custom workflow written in the Bonsai visual programming language \citep{Lopes2015a}.

Prior to the micro-ECoG recordings, each step in the obstacle course was outfitted with a micro load cell (CZL616C, Phidgets, CA) secured between the step front holder and the base (Figure \ref{fig:assay}B). This allowed us to record a varying voltage signal proportional to the load applied by the animal on each step. This load signal was acquired simultaneously on all eight steps and digitized synchronously with the ECoG data acquisition system.

\subsection{Data acquisition}

The behaviour of the animals was recorded with a high-speed and high-resolution videography system (1280x680 @ \SI{120}{\hertz}) using an infrared camera (Flea3, PointGrey, CA), super-bright infrared LED front lights (SMD5050, 850 nm) and a vari-focal lens (Fujinon, JP) positioned in front of the transparent window partition. A top view of the assay was simultaneously recorded with the same system at a lower frame-rate (\SI{30}{\hertz}) for monitoring purposes. All video data was encoded with MPEG-4 compression for subsequent offline analysis. Behaviour data acquisition for the nose poke beam breaks was done using an Arduino board (Uno, Arduino, USA) and streamed to the computer via USB. All video and sensor data acquisition was recorded in parallel using the same Bonsai workflow used to control the behaviour assay.

For the micro-ECoG recordings, all electrophysiological signals were amplified, digitized and multiplexed using two 64-channel amplifier boards (RHD2164, Intan Technologies, US) connected to the electrode interface board (EIB) on the recording chamber. The amplifier boards were then connected through a dual headstage adapter (C3440, Intan Technologies, US) to the main data acquisition USB interface board (RHD2000-Eval, Intan Technologies, US). In order to facilitate the free movement of the animal in the behaviour box, the single cable connecting the head of the animal to the USB interface board was passed through a slip ring (MMC235, Moflon, CN) and hooked into a nylon string crossing the top of the assay. In this way, movement and rotation of the tethered animal were compensated to avoid unwanted strain and twisting on the cables during the entire recording period.

In order to synchronize the videography and ECoG recording systems, we connected the strobe output of the camera to a digital input in the Intan USB interface board using a GPIO cable (ACC-01-3000, PointGrey, CA). The camera strobe output is electronically coupled to individual frame exposures (i.e. shutter opening and closing events), and can be used for sub-millisecond readout of individual frame acquisition times. The strobe signal was acquired and digitized synchronously with ECoG data acquisition, and used for \emph{post-hoc} reconstruction of precise frame timing. Data acquisition from the USB interface board was recorded using a Bonsai workflow and care was taken that it was always started first and terminated last in order to ensure that no external synchronization events were lost.

\subsection{Behaviour protocol}

The animals were kept in a state of water deprivation for \SI{20}{\hour} prior to each daily session. For every trial, rats were delivered a \SI{20}{\micro\liter} drop of water. At the end of each day, they were given free access to water for \SI{10}{\minute} before initiating the next deprivation period. Sessions lasted for six days of the week from Monday to Saturday, with a day of free access to water on Sunday. Before the start of the water deprivation protocol, animals were run on a single habituation session where they were placed in the box for a period of \SI{15}{\minute}.

The following sequence of conditions were presented to the animals over the course of a month (see also Figure \ref{fig:assay}A): day 0, habituation to the box; day 1-4, all the steps were fixed in a stable configuration; day 5, 20 trials of the stable configuration, after which the two centre steps were made unstable (i.e. free to rotate); day 6-10, the centre two steps remained unstable; day 11, 20 trials of the unstable configuration, after which the two centre steps were again fixed in a stable state; day 12, all the steps were fixed in a stable configuration; day 13-16, the state of the centre two steps was randomized on a trial-by-trial basis to be either stable or unstable. Following the end of the random protocol, animals continued to be tested in the assay for a variable number of days (up to one week) in different conditions. At the end of the testing period, all animals were exposed to a final session where all steps were made free to rotate in order to assay locomotion performance under challenging conditions.

For the micro-ECoG recordings, the basic behaviour protocol was adjusted to allow for extra recording time during conditions of interest. First, all session times were doubled for the recordings (e.g. \SI{30}{\minute} for the habituation session, and \SI{60}{\minute} for all other sessions). Second, the number of days on each condition was also extended to allow extracting more trials from each animal for analysis. Finally, the condition where the centre two steps were reliably unstable was replaced with a condition of rare instability. In this condition, after the animal is exposed to an unstable configuration, the steps are reverted back to being stable for another 20 trials, after which they become again unstable for one trial, and so on.

\subsection{Data analysis}

All scripts and custom code used for data analysis are available online\footnote{https://bitbucket.org/kampff-lab/shuttling-analysis}. The raw video data was first pre-processed using a custom Bonsai workflow in order to extract features of interest (Figure \ref{fig:assay}C). Tracking of the nose was achieved by background subtraction and connected component labelling of segmented image elements. First we compute the ellipse best-fit to the largest object in the image. We then mark the tip of the nose as the furthermost point, in the segmented shape of the animal, along the major axis of the ellipse. In order to analyse stepping performance, regions of interest were defined around the surface of each step and in the gaps between the steps. Background subtracted activity over these regions was recorded for every frame for subsequent detection and classification of steps and slips.

Analysis routines were run using the NumPy scientific computing package \citep{Walt2011} and the Pandas data analysis library \citep{McKinney2010} for the Python programming language. Crossings were automatically extracted from the nose trajectory data by first detecting consecutive time points where the nose was positively identified in the video. In order for these periods to be successfully marked as crossings, the starting position of the nose must be located on the opposite side of the ending position. Inside each crossing, the moment of stepping with the forelimb on the centre steps was extracted by looking at the first peak above a threshold in the first derivative of the activation signal in the corresponding region of interest. False positive classifications due to hindlimb or tail activations were eliminated by enforcing the constraint that the position of the head must be located before the next step. Visual confirmation of the classified timepoints showed that spurious activations were all but eliminated by this procedure as stepping with the hindlimb or tail requires the head to be further ahead in space unless the animal turned around (in which case the trajectory would not be marked as a crossing anyway). The position of the nose at the moment of each step was extracted and found to be normally distributed, so statistical analysis of the step posture in the random condition used an unpaired t-test to check for independence of different measurement groups.

In order to evaluate the dynamics of crossing in the random condition, we first measured for every trial the speed at which the animals were moving on each spatial segment of the assay. To minimize overall trial-by-trial variation in individual animal performance, we used the average speed at which the animal approached the manipulated step as a baseline and subtracted it from the speed at each individual segment. To summarize differences in performance between stable and unstable trials, we then computed the average speed profile for each condition, and then subtracted the average speed profile for unstable trials from the average speed profile for stable trials. Finally, we computed the sum of all these speed differences at every segment in order to obtain the speedup index for each animal, i.e. an index of whether the animal tends to accelerate or decelerate across the assay on stable versus unstable trials.

For the micro-ECoG experiments, evoked potentials were analysed by splitting the raw physiological voltage traces into \SI{750}{\milli\second} windows, where time zero was aligned to the moment of stepping with the forelimb on one of the obstacles in the course (see below). Each individual time series was low-pass filtered at \SI{50}{\hertz} (4th order Butterworth filter, two-pass) and baselined by subtracting the average of the first \SI{250}{\milli\second} before event onset in order to compensate for constant voltage shifts between the two grids. Some of the channels in each grid were entirely excluded from the analysis due to potentially damaged surface contacts, as evidenced by wide amplitude, random oscillatory behaviour, which was often matched by the presence of high impedance measurements extracted from the electrode site in vivo. In one of the sessions, the cable connecting the headstage to the interface board was accidentally removed by the animal, and all the trials falling during this period had to be excluded from analysis. Correspondence between individual ECoG samples and video frames was computed by matching the individual hardware frame counter with the sequence of falling edges detected in the shutter strobe signal acquired from the infrared camera.

\subsection{Video classification}

Classification of paw placement faults (i.e. slips) was performed in semi-automated fashion. First, possible slip timepoints were detected automatically using the peak detection method outlined above. All constraints on head position were relaxed for this analysis in order to exclude the possibility of false negatives. A human classifier then proceeded to manually go through each of the slip candidates and inspect the video around that timepoint in order to assess whether the activation peak was a genuine paw placement fault. Examples of false positives include tail and head activations as well as paw activations that occur while the animal is actively engaged in exploration, rearing, or other activities that are unrelated to crossing the obstacles.

A similar technique was used to detect and classify the event onsets for the analysis of evoked potentials in the micro-ECoG experiments. In this case, a preliminary classification of each video frame into left and right forelimb was achieved by first computing the brightness histogram of each frame, which was used to encode the image as a lower-dimensional vector. The vectors for all step frames were subsequently clustered using K-means and then manually inspected for label correction.

Classification of behaviour responses following first exposure to the unstable condition was done on a frame-by-frame analysis of the high-speed video aligned on first contact with the manipulated step. The frame of first contact was defined as the first frame in which there is noticeable movement of the step caused by animal contact. Three main categories of behaviour were observed to follow the first contact: compensation, investigation and halting. Behaviour sequences were first classified as belonging to one of these categories and their onsets and offsets determined by the following criteria. Compensation behaviour is defined by a rapid and adaptive postural correction to the locomotion pattern in response to the perturbation. Onset of this behaviour is defined by the first frame in which there is visible rapid contraction of the body musculature following first contact. Investigation behaviour consists of periods of targeted interaction with the steps, often involving manipulation of the freely moving obstacle with the forepaws. The onset of this behaviour is defined by the animal orienting its head down to one of the manipulated steps, followed by subsequent interaction. Halting behaviour is characterized by a period in which the animal stops its ongoing motor program, and maintains the same body posture for several seconds, without switching to a new behaviour or orienting specifically to the manipulated steps. This behaviour is distinct from a freezing response, as occasional movements of the head are seen. Onset of this behaviour is defined by the moment where locomotion and other motor activities besides movement of the head come to a stop. A human classifier blind to the lesion condition was given descriptions of each of these three main categories of behaviour and asked to note onsets and offsets of each behaviour throughout the videos. These classifications provide a visual summary of the first response videos; the complete dataset used for this classification is included as supplementary movies.

\section{Author contributions}

Conceived and designed the lesion experiments: G.L., A.R.K., J.J.P.; Performed the lesion experiments: G.L., J.N.; Analysed the lesion data: G.L., A.R.K.; Conceived and designed the ECoG experiments: G.L., G.D., A.R.K., J.J.P.; Performed the ECoG implantation surgeries: G.D., J.N.; Performed the ECoG behaviour experiments: G.L.; Analysed the ECoG data: G.L., G.D., J.A.M., A.R.K.; Wrote the manuscript: G.L., A.R.K.

\section{Competing interests}

The authors declare that the research was conducted in the absence of any commercial or financial relationships that could be construed as a potential conflict of interest.

\section{Acknowledgements}

We thank Lorenza Calcaterra for the extended frontoparietal cortical lesion preparations; João Gaspar of the International Iberian Nanotechnology Laboratory for kindly providing the fabrication process for the micro-ECoG grids; João Frazão, Pedro Lacerda and Tiago Monteiro for invaluable help in extending the behaviour assay for the micro-ECOG recordings and all the members of the Intelligent Systems Lab for constant feedback on the ideas, experiments and manuscript as well as help annotating behaviour videos. The research leading to these results has received funding from the European Union's Seventh Framework Programme (FP7/2007-2013) under grant agreement no. 600925. G.L. was supported by the PhD Studentship SFRH/BD/51714/2011 from the Foundation for Science and Technology. The Champalimaud Neuroscience Programme is supported by the Champalimaud Foundation.

\begin{figure}
\centering
\includestandalone{figures/assay}
\caption{An obstacle course for rodents. (\textbf{A}) Schematic of the apparatus and summary of the different conditions in the behaviour protocol. Animals shuttle back and forth between two reward ports at either end of the enclosure. (\textbf{B}) Schematic of the locking mechanism that allows each individual step to be made stable or unstable on a trial-by-trial basis. (\textbf{C}) Example video frame from the behaviour tracking system. Coloured overlays represent regions of interest and feature traces extracted automatically from the video.}
\label{fig:assay}
\end{figure}

\begin{figure}
\centering
\includestandalone{figures/histology}
\caption{Histological analysis of lesion size. (\textbf{A}) Representative example of bilateral ibotenic acid lesion of primary and secondary forelimb motor cortex. (\textbf{B}) Distribution of lesion volumes in the left and right hemispheres for individual animals. A lesion was considered ``large'' if the total lesion volume was above \SI{15}{\milli\meter\cubed}. (\textbf{C}) Super-imposed reconstruction stacks for all the small lesions. (\textbf{D}) Super-imposed reconstruction stacks for all the large lesions.}
\label{fig:histology}
\end{figure}

\begin{figure}
\centering
\includestandalone{figures/learning}
\caption{Overall performance on the obstacle course is similar for both lesion and control animals. (\textbf{A}) Average time to cross the obstacles on the different protocol stages for forelimb motor cortex lesions and control animals. Each set of three coloured bars represents the average performance on a single session. Asterisks indicate sessions where there was a change in assay conditions during the session (see text). In these transition sessions, the average performance on the 20 trials immediately preceding the change is shown to the left of the solid vertical line whereas the performance on the remainder of that session (after the change) is shown to the right. 


(New Figure!)
(\textbf{B}) Average time required to cross the obstacles in the stable condition for extended lesions. Performance of the other groups is shown for comparison. (\textbf{C}) Average number of slips per crossing in early versus late sessions of the stable condition. Animals adopting a jumping strategy (both lesions and controls; see text) were not included in this analysis as jumping increased the probability of incorrect paw placement independently of proficiency.}
\label{fig:learning}
\end{figure}

\begin{figure}
\centering
\includestandalone{figures/posture}
\caption{Rats adapt their postural approach to the obstacles after a change in physics. (\textbf{A}) Schematic of postural analysis image processing. The position of the animal's nose is extracted whenever the paw activates the ROI of the first manipulated step (see methods). (\textbf{B}) The horizontal position, i.e. progression, of the nose in single trials for one of the control animals stepping across the different conditions of the shuttling protocol. (\textbf{C}) Average horizontal position of the nose across the different protocol stages for both lesion and control animals. Asterisks indicate the average nose position on the 20 trials immediately preceding a change in protocol conditions (see text). (\textbf{D}) Distribution of nose positions for control animals in the last two days of the stable (blue) and unstable (orange) protocol stages, respectively for (\textbf{E}) small lesions and (\textbf{F}) large lesions.}
\label{fig:posture}
\end{figure}

\begin{figure}
\centering
\includestandalone{figures/jumping}
\caption{Animals use different strategies for dealing with the unstable obstacles. (\textbf{A}) Example average projection of all posture images for stable (green) and unstable (red) sessions for two non-jumper and two jumper animals. (\textbf{B}) Average nose trajectories for individual animals crossing the unstable condition. The shaded area around each line represents the 95\% confidence interval. (\textbf{C}) Correlation of the probability of skipping the center two steps with the weight of the animal. The color indicates whether the animal was a control or a lesion.}
\label{fig:jumping}
\end{figure}

\begin{figure}
\centering
\includestandalone{figures/random}
\caption{Animals adjust their posture on a trial-by-trial basis to the expected state of the world. (\textbf{A}) Distribution of nose positions for all animals in stable (blue) and unstable (orange) trials of the randomized protocol (see methods) when stepping on first unstable obstacle. (\textbf{B}) Distribution of nose positions for trials in which the previous two trials were stable (blue) or unstable (orange). (\textbf{C-D}) Same data as in (\textbf{B}) split by the control and lesion groups.}
\label{fig:random}
\end{figure}

\begin{figure}
\centering
\includestandalone{figures/speed}
\caption{Encountering different states of the randomized obstacles causes the animals to quickly adjust their movement trajectory. (\textbf{A}) Example average speed profile across the obstacles for stable (blue) and unstable (orange) trials in the randomized sessions of a control animal (see text). The shaded area around each line represents the 95\% confidence interval. (\textbf{B}) Respectively for one of the largest lesions. (\textbf{C}) Summary of the average difference between the speed profiles for stable and unstable trials across the two groups of animals. Error bars show standard error of the mean.} (add p value)
\label{fig:speed}
\end{figure}

\begin{figure}
\centering
\includestandalone{figures/ethogram}
\caption{Responses to an unexpected change in the environment. (\textbf{A}) Response types observed across individuals upon first encountering an unpredicted instability in the state of the centre obstacles. (\textbf{B}) Ethogram of behavioural responses classified according to the three criteria described in (\textbf{A}) and aligned (0.0) on first contact with the newly manipulated obstacle. Black dashes indicate when the animal exhibits a pronounced ear flick. White indicates that the animal has crossed the obstacle course.}
\label{fig:ethogram}
\end{figure}

\end{linenumbers}

% Allow breaks at numbers and letters for bibliography
\setcounter{biburlnumpenalty}{100}
\setcounter{biburlucpenalty}{100}
\setcounter{biburllcpenalty}{100}
\hyphenation{Gross-hirns}
\printbibliography

\end{document}